\documentclass{ximera}

\input{../preamble.tex}

\title[Dig-In:]{A review of integration}

\begin{document}
\begin{abstract}
  We review differentiation and integration.
\end{abstract}
\maketitle

Let's get started. Recall the derivative rules:

\begin{theorem}[Basic Derivative Rules]\index{derivative rules}\hfil
\begin{itemize}
\item $\ddx k =0$
\item $\ddx x^n  = n x^{n-1}$
\item $\ddx e^x = e^x$
\item $\ddx a^x = a^x\ln(a)$
\item $\ddx \ln(x) = \frac{1}{x}$
\item $\ddx \sin(x) = \cos(x)$
\item $\ddx \cos(x) = -\sin(x)$  
\item $\ddx \tan(x) = \sec^2(x)$  
\item $\ddx \sec(x) = \sec(x)\tan(x)$ 
\item $\ddx \csc(x) = -\csc(x)\cot(x)$
\item $\ddx \cot(x) = -\csc^2(x)$
\item $\ddx \arcsin(x) = \frac{1}{\sqrt{1-x^2}}$
\item $\ddx \arccos(x) = \frac{-1}{\sqrt{1-x^2}}$
\item $\ddx \arctan(x) =\frac{1}{1+x^2}$
\item $\ddx \arcsec(x) = \frac{1}{|x|\sqrt{x^2-1}}$ for $|x|>1$
\item $\ddx \arccsc(x) = \frac{-1}{|x|\sqrt{x^2-1}}$ for $|x|>1$
\item $\ddx \arccot(x) = \frac{-1}{1+x^2}$
\item $\ddx \left( f(x) + g(x) \right) = f'(x) + g'(x)$
\item $\ddx \left( f(x) \cdot g(x) \right) = f(x)g'(x) + f'(x)g(x)$
\item $\ddx\frac{f(x)}{g(x)} = \frac{f'(x)g(x) - f(x)g'(x)}{g(x)^2}$
\item $\ddx f(g(x)) = f'(g(x)) \cdot g'(x)$
\end{itemize}
%\end{multicols}
\end{theorem}

\begin{question}
  What is the derivative of $\arcsin\left(\frac{x}{a}\right)$ with respect to $x$?
  \begin{prompt}
    \[
    \ddx \arcsin\left(\frac{x}{a}\right) = \answer{1/\sqrt{a^2 - x^2}}
    \]
  \end{prompt}
\end{question}

When computing definite integrals, the Fundamental Theorem of Calculus
says:

\begin{theorem}[Fundamental Theorem of Calculus]\index{Fundamental Theorem of Calculus}
  Let $f$ be continuous on $[a,b]$. If $F$ is \textbf{any}
  antiderivative of $f$, then
  \[
  \int_a^b f(x)\d x = F(b)-F(a).
  \]
\end{theorem}

The Fundamental Theorem of Calculus allows us to compute
integrals by ``undoing'' the derivative.



ADD MORE --- ALSO HOW TO DO IN XIMERA???

\begin{theorem}[Basic Integrals]\index{antiderivatives}\index{indefinite integral}\hfil
\begin{itemize}
\item $\int k \d x= kx+C$.
\item $\int x^n \d x= \frac{x^{n+1}}{n+1}+C\qquad(n\ne-1)$.
\item $\int e^x \d x= e^x + C$.
\item $\int a^x \d x= \frac{a^x}{\ln(a)}+C$.
\item $\int \frac{1}{x} \d x= \ln|x|+C$.
\item $\int \cos(x) \d x = \sin(x) + C$.
\item $\int \sin(x) \d x = -\cos(x) + C$.  
\item $\int \tan(x) \d x = -\ln|\cos(x)| + C$.  
\item $\int \sec^2(x) \d x = \tan(x) + C$. 
\item $\int \csc^2(x) \d x = -\cot(x) + C$.
\item $\int \sec(x)\tan(x) \d x = \sec(x) + C$.
\item $\int \csc(x)\cot(x) \d x = -\csc(x) + C$.
\item $\int \frac{1}{x^2+1}\d x = \arctan(x) + C$.
\item $\int \frac{1}{\sqrt{1-x^2}}\d x= \arcsin(x)+C$.
\end{itemize}
\end{theorem}

Whenever you compute an indefinite integral, you can always check your
work by taking the derivative. In particular, this means that if you
are given choices for the antiderivative, you can check simply by taking the derivative.

\begin{question}
  A question
  \begin{selectAll}
    \choice[correct]{hello} DO SOMETHING WITH LOG AND CONSTANTS HIDDEN IN THE +C
  \end{selectAll}
\end{question}

The substitution formula is a tool that can help compute some integrals.

\begin{theorem}[Integral Substitution Formula] 
If $g$ is differentiable on the interval $[a,b]$ and $f$ is
differentiable on the interval $[g(a),g(b)]$, then
\[
\int_a^b f'(g(x)) g'(x) \d x =\int_{g(a)}^{g(b)} f'(g) \d g.
\]
\end{theorem}


In the substitution formula allows us to change the form of a novel
integral, so that we can evaluate it using forms we know.

\begin{question}
  Letting $a$ be a constant, what is the antiderivative of
  $\frac{1}{a^2 + x^2}$?
  \begin{prompt}
  \[
  \int \frac{\d x}{a^2 + x^2}
  \]
  \end{prompt}
\end{question}

As we saw above, ... 



\end{document}
