\documentclass{ximera}

\input{../preamble.tex}

\title[Dig-In:]{Accumulated cross-sections}

\outcome{hello}

\begin{document}
\begin{abstract}
  We can also use integrals to compute volume.
\end{abstract}
\maketitle

\section{Volume}

We have seen how to compute certain areas by using integration. Now we
will see how to compute some volumes.  The volumes that we can compute
will have cross-sections that are easy to describe. Sometimes we think
of these cross-sections as being ``slabs'' that we are layering to
create a volume.


\begin{example}
Find the volume of a pyramid with a square base that is $20$ meters tall
and $20$ meters on a side at the base.
\begin{image}
  \begin{tikzpicture}
    \begin{axis}[
          xmin =-4,xmax=7,ymax=4,ymin=-4,
          axis lines=center, xlabel=$x$, ylabel=$y$,
          every axis y label/.style={at=(current axis.above origin),anchor=south},
          every axis x label/.style={at=(current axis.right of origin),anchor=west},
          width=5in,
          axis on top,
          xtick={0,6}, xticklabels={$0$, $20$},
          ytick={0,3},yticklabels={$0$,$20$},
            clip=false,
      ]
      \addplot [draw=penColor, fill = fillp, very thick] plot coordinates {(-3,-3) (3,-3) (6,0) (1.5, 3) (-3,-3)};
      \addplot [draw=penColor, very thick] plot coordinates {(-3,-3) (0,0)};
      \addplot [draw=penColor, very thick] plot coordinates {(6,0) (0,0)};
      \addplot [draw=penColor, very thick] plot coordinates {(1.5,3) (3,-3)};
      \addplot [draw=penColor, very thick] plot coordinates {(1.5,3) (0,0)};

      \addplot [->] plot coordinates {(0,0) (-4,-4)};
      \node[anchor=north east] at (axis cs:-4,-4) {$z$};       
    \end{axis}
  \end{tikzpicture}
\end{image}

\begin{explanation}
To solve this problem, we will ``sum up'' (integrate) infinitely many
``infinitesmal'' slabs which are parallel to the base of the pyramid
to obtain the volume.
\begin{image}
  \begin{tikzpicture}
    \begin{axis}[
          xmin =-4,xmax=7,ymax=4,ymin=-4,
          axis lines=center, xlabel=$x$, ylabel=$y$,
          every axis y label/.style={at=(current axis.above origin),anchor=south},
          every axis x label/.style={at=(current axis.right of origin),anchor=west},
          axis on top,
          width=5in,
          xtick={0,6}, xticklabels={$0$, $20$},
          ytick={0,3},yticklabels={$0$,$20$},
            clip=false,
      ]
      \addplot [draw=penColor, thick] plot coordinates {(-3,-3) (3,-3) (6,0) (1.5, 3) (-3,-3)};
            
      \addplot [draw=penColor, thick] plot coordinates {(-3,-3) (0,0)};
      \addplot [draw=penColor, thick] plot coordinates {(6,0) (0,0)};
      \addplot [draw=penColor, thick] plot coordinates {(1.5,3) (3,-3)};
      \addplot [draw=penColor, thick] plot coordinates {(1.5,3) (0,0)};

      %% slab
      \addplot [draw=penColor, fill=fillp,very thick] plot coordinates {(3,2) (1,2) (0,1) (2, 1) (3,2)};
      \addplot [draw=penColor, fill=fillp,very thick] plot coordinates {(0,.8) (0,1) (2,1) (2, .8) (0,.8)};
      \addplot [draw=penColor, fill=fillp,very thick] plot coordinates {(2,1) (2, .8) (3,1.8) (3,2) (2,1)};

      %\addplot [draw=penColor, fill=fillp,very thick] plot coordinates {(3,1.8) (1,1.8) (0,.8) (2, .8) (3,1.8)};
      %\addplot [draw=penColor, fill=fillp,very thick] plot coordinates {(3,2) (1,2) (0,1) (2, 1) (3,2)};

      \addplot [draw=penColor, thick] plot coordinates {(1.5,3) (3,-3)};

      \draw[decoration={brace,mirror,raise=.1cm},decorate,thin] (axis cs:0,.8)--(axis cs:2,.8);
      \draw[decoration={brace,raise=.1cm},decorate,thin] (axis cs:3.1,2.05)--(axis cs:3.1,1.75);
      
      \addplot [->] plot coordinates {(0,0) (-4,-4)};
      \node[anchor=north east] at (axis cs:-4,-4) {$z$};

      \node at (axis cs:3.6,1.9) {$\d y$};
      \node at (axis cs:1,.4) {$L(y)$};       
    \end{axis}
  \end{tikzpicture}
\end{image}
The ``height'' of each slab will be $\d y$, and we'll let $L(y)$ be the width:
\[
L(y) = \answer[given]{20-y}
\]
\begin{hint}
  Since $L(y)$ is a linear function of $y$, and $L(0) = 20$, and
  $L(20) = 0$ we see $L(y) = 20-y$ by the point-slope formula.
\end{hint}
For each slab, the infinitesmal volume at height $y$ is
\begin{align*}
  \d V = \mathrm{length} \cdot \mathrm{width}\cdot \mathrm{height}\\
&= L(y)\cdot L(y)\cdot  \d y
\end{align*}
so the total volume is given by
\begin{align*}
  \mathrm{Volume} &= \int_0^{20} (L(y))^2 \d y \\
  &= \int_0^{20} (\answer[given]{20-y})^2 \d y
\end{align*}
Making the substitution $g = 20-y $, we have $\d g = - \d y$, $g$ going from $20$ to $0$, and 
	\begin{align*}
	\int_0^{20} (20-y)^2 \d y &= -\int_{\answer[given]{20}}^{\answer[given]{0}} g^2 \d g\\
		&= \int_0^{20} g^2 \d g\\
		&= \eval{\answer[given]{\frac{1}{3} g^3}}_0^{20}\\
		&= \frac{20^3}{3}.
	\end{align*}
As you may know, the volume of a pyramid is
$(1/3)(\text{height})(\text{area of base})=(1/3)(20)(400)$, which
agrees with our answer.
\end{explanation}
\end{example}



\begin{example}
The base of a solid is the region between $f(x)=x^2-1$ and
$g(x)=-x^2+1$, as pictured below. Its cross-sections
perpendicular to the $x$-axis are equilateral triangles.  Find the volume of the
solid.
\begin{image}
\begin{tikzpicture}
  \begin{axis}[
      xmin=-1, xmax=1,domain=-1:1,
      clip=false,
      axis lines =center, xlabel=$x$, ylabel=$y$,
      every axis y label/.style={at=(current axis.above origin),anchor=south},
      every axis x label/.style={at=(current axis.right of origin),anchor=west},
      axis on top,
    ] 
    \addplot [penColor2,very thick] {x^2-1};
    \addplot [penColor,very thick] {-x^2+1};
    \node at (axis cs:1,0.8) [penColor] {$y = -x^2+1$};
    \node at (axis cs:1,-0.8) [penColor2] {$y = x^2-1$};
  \end{axis}
\end{tikzpicture}

\end{image}

\begin{explanation}




%% \begin{image}
%% % triangleRegion
%% \begin{tikzpicture}
%%  \begin{axis}[view={30}{30},colormap/\surfaceColor,
%%    xlabel=$x$, ylabel=$z$, zlabel=$y$,]

%%   \addplot3[surf,shader=faceted,opacity=.3,
%%   samples=12,
%%   samples y = 6,
%%   domain=-1:.5,y domain=0:1,
%%   z buffer=sort]
%%   (x,{(1-y)*(-x^2+1)}, {y*sqrt(3)*(1-x^2)});

%%   \addplot3[surf,shader=faceted,opacity=.7,%front
%%   samples=12,
%%   samples y = 6,
%%   domain=-1:.5,y domain=0:1,
%%   z buffer=sort]
%%   (x,{(1-y)*(x^2-1)}, {y*sqrt(3)*(1-x^2)});

%%   %tri
%%   \addplot3[surf,colormap/\sliceColor,shader=flat,
%%   samples=6,
%%   samples y = 6,
%%   domain=-1:1,y domain=0:1,
%%   z buffer=sort]
%%   (.5,{x*(1-y)*(.5^2-1)}, {abs(x)*y*sqrt(3)*(1-.5^2)});

%%   \addplot3[surf,shader=faceted,opacity=.3, 
%%   samples=4,
%%   samples y = 6,
%%   domain=.5:1,y domain=0:1,
%%   z buffer=sort]
%%   (x,{(1-y)*(-x^2+1)}, {y*sqrt(3)*(1-x^2)});

%%   \addplot3[surf,shader=faceted,opacity=.7,%front
%%   samples=4,
%%   samples y = 6,
%%   domain=.5:1,y domain=0:1,
%%   z buffer=sort]
%%   (x,{(1-y)*(x^2-1)}, {y*sqrt(3)*(1-x^2)});

%%  \end{axis}
%% \end{tikzpicture}
%% %% \caption{A solid with equilateral triangle cross-sections bounded by
%% %%   the region between $f(x)=x^2-1$ and $g(x)=-x^2+1$.}
%% %% \label{fig:triangular cross-sections}
%% \end{image}

We want to find the volume of the triangular slab at $x$. We know its width is $\d x$.  What is the area,  $A(x)$, of the triangular face of this slab?

\begin{question}
	$A(x) = \answer{\sqrt{3}(1-x^2)^2}$
	
	\begin{hint}
		The base of the slab has length $2(1-x^2)$ since $(1-x^2)- (x^2-1) = 2(1-x^2)$. By geometry, the height is $\sqrt{3}(1-x^2)$.  So the area of this triangle is $\frac{1}{2} \textrm{base} \cdot \textrm{height} = \sqrt{3}(1-x^2)^2$.
	\end{hint}	
\end{question}


We want to ``sum'' (integrate) all of the infinitesmal volumes $\d V = A(x) \d x$ from $x=-1$ to $x=1$.
Thus the total volume is

\begin{question}
\[
\textrm{Volume} = \int_{-1}^1 A(x) \d x=\answer{\frac{16\sqrt{3}}{15}}
\]

\begin{hint}
	\begin{align*}
	\int_{-1}^1 A(x) \d x  &= \sqrt{3} \int_{-1}^1 (1-x^2)^2 \d x\\
		&=\sqrt{3} \int_{-1}^1 1-2x^2+x^4 \d x \\
		&=\sqrt{3} \eval{x-\frac{2}{3}x^3+\frac{1}{5}}_{-1}^1\\
		&=2\sqrt{3}(1-\frac{2}{3}+\frac{1}{5})\\
		&=\frac{16\sqrt{3}}{15} 
	\end{align*}
\end{hint}

\end{question}


\end{explanation}
\end{example}



One easy way to get ``nice'' cross-sections is by rotating a plane
figure around a line. Below we see a function $f$ bounded by two
vertical lines:
\begin{image}
\begin{tikzpicture}
  \begin{axis}[
      xmin=0, xmax=5,domain=0:5, clip=false, axis lines =center,
      xlabel=$x$, ylabel=$z$, every axis y label/.style={at=(current
        axis.above origin),anchor=south}, every axis x
      label/.style={at=(current axis.right of origin),anchor=west},
      axis on top, ]
    \addplot [penColor,very thick,smooth]{16-19*x+8*x^2-x^3};
    \addplot [textColor,dashed] plot coordinates {(1,0) (1,4)};
    \addplot [textColor,dashed] plot coordinates {(4,0) (4,4)};
  \end{axis}
\end{tikzpicture}
%% \caption{A plot of $f(x)$.}
%% \label{figure:regionCurve}
\end{image}


Rotating $f$ around the $x$-axis will generate a figure whose volume
we can compute.


%% \begin{image}
%% \begin{tikzpicture}
%%  \begin{axis}[view={30}{30},colormap/\surfaceColor,
%%    xlabel=$x$, ylabel=$z$, zlabel=$y$,
%%    ]

%%   \addplot3[surf,shader=faceted,opacity=.3,
%%   samples=10,
%%   samples y =20,
%%   domain=1:2.7,y domain=0:2*pi,
%%   z buffer=sort]
%%   (x,{(16-19*x+8*x^2-x^3) * cos(deg(y))}, {(16-19*x+8*x^2-x^3) * sin(deg(y))});

%%   \addplot3[surf,colormap/\sliceColor,shader=flat,%disk
%%   samples=5,
%%   samples y = 20,
%%   domain=0:1,y domain=0:2*pi,
%%   z buffer=sort
%% ]
%%   (2.7,{3.337*x*cos(deg(y))}, {3.337*x*sin(deg(y))});


%%   \addplot3[surf,shader=faceted,opacity=.3,
%%   samples=10,
%%   samples y=20,
%%   domain=2.7:4,y domain=0:2*pi,
%%   z buffer=sort]
%%   (x,{(16-19*x+8*x^2-x^3) * cos(deg(y))}, {(16-19*x+8*x^2-x^3) * sin(deg(y))});


%%   \addplot3[surf,shader=faceted,opacity=.7,
%%   samples=5,
%%   samples y = 20,
%%   domain=0:1,y domain=0:2*pi,
%%   z buffer=sort
%% ]
%%   (4,{4*x*cos(deg(y))}, {4*x*sin(deg(y))});

%%  \end{axis}
%% \end{tikzpicture}
%% \end{image}

The volume of each disk will have the form $\pi \left( f(x)\right)^2 \d x$, and so we only have to accumulate these infinitesmal volumes to obtain the total volume.

\begin{example}
Find the volume of a right circular cone with base radius $10$ and
height $20$. Here, a right circular cone is one with a circular base and
with the tip of the cone directly over the center of the base.

%% \begin{image}
%% %cone
%% \begin{tikzpicture}
%%  \begin{axis}[
%%      view={30}{30},colormap/\surfaceColor,
%%      xlabel=$x$, ylabel=$z$, zlabel=$y$,
%%    ]

%%   \addplot3[surf,shader=faceted,opacity=.3,
%%   samples=10,
%%   samples y =20,
%%   domain=0:12,y domain=0:2*pi,
%%   z buffer=sort]
%%   (x,{(x/2) * cos(deg(y))}, {(x/2) * sin(deg(y))});

%%   \addplot3[surf,colormap/\sliceColor,shader=flat,%disk
%%   samples=5,
%%   samples y = 20,
%%   domain=0:1,y domain=0:2*pi,
%%   z buffer=sort
%% ]
%%   (12,{6*x*cos(deg(y))}, {6*x*sin(deg(y))});


%%   \addplot3[surf,shader=faceted,opacity=.3,
%%   samples=10,
%%   samples y=20,
%%   domain=12:20,y domain=0:2*pi,
%%   z buffer=sort]
%%   (x,{(x/2) * cos(deg(y))}, {(x/2) * sin(deg(y))});


%%   \addplot3[surf,shader=faceted,opacity=.7,
%%   samples=5,
%%   samples y = 20,
%%   domain=0:1,y domain=0:2*pi,
%%   z buffer=sort
%% ]
%%   (20,{10*x*cos(deg(y))}, {10*x*sin(deg(y))});

%%  \end{axis}
%% \end{tikzpicture}
%% %% \caption{A right circular cone with base radius 10 and height 20.}
%% %% \label{fig:line to cone}
%% \end{image}

We can view this cone as produced by the rotation of the line
$y=x/2$ rotated about the $x$-axis.

At a particular point on the $x$-axis, the radius of the resulting
cone is the $y$-coordinate of the corresponding point on the line
$y=x/2$. The area of the cross section is given by

\begin{question}
\[
A(x) = \answer{\pi \left(\frac{x}{2}\right)^2}
\]

\begin{hint}
	The radius is $\frac{x}{2}$, so the area is $\pi \frac{x^2}{4}$
\end{hint}
\end{question}

The infinitesmal volume of each disc is then $A(x) \d x$, so the total volume is the integral of these infinitesmal volumes from $x = 0$ to $x = 20$.

\begin{question}
\[
\textrm{Volume} = 
  \int_0^{20} A(x) \d x=\answer{\frac{2000\pi}{3}}.
\]

\begin{hint}
	\begin{align*}
		\int_0^{20} A(x) \d x &= \int_0^{20} \pi \frac{x^2}{4} \d x\\ 
			&= \pi
	\end{align*}
\end{hint}
\end{question}

Note that we can instead do the calculation with a generic height and
radius: 

\[
  \int_0^{h} \pi\frac{r^2}{h^2}x^2\d x
  = \frac{\pi r^2 h}{3},
\]
giving us the usual formula for the volume of a cone.
\end{example}


\end{document}
