\documentclass{ximera}

\input{../preamble.tex}

\outcome{Compute volumes using the shell method.}
\outcome{Know when to use the shell method.}
\outcome{Set up integrals for the computing volume using the shell method.}

\title[Dig-In:]{Accumulated shells}

\begin{document}
\begin{abstract}
Some volumes of revolution are more easily computed with cylindrical shells.
\end{abstract}
\maketitle


Consider the region bounded by $f(x)=x+1$ and $g(x)=(x-1)^2$:
\begin{image}
\begin{tikzpicture}
  \begin{axis}[
      xmin=0, xmax=3,domain=0:3,
      clip=false,
      axis lines =center, xlabel=$x$, ylabel=$y$,
      every axis y label/.style={at=(current axis.above origin),anchor=south},
      every axis x label/.style={at=(current axis.right of origin),anchor=west},
      axis on top,
    ]
    \addplot [draw=none,fill=fillp] {x+1}\closedcycle;
    \addplot [draw=none,fill=white] {(x-1)^2}\closedcycle;
    \addplot [penColor,very thick] {x+1};
    \addplot [penColor2,very thick] {(x-1)^2};

    \node at (axis cs:2,3.3) [penColor] {$f$};
    \node at (axis cs:2.5,1.7) [penColor2] {$g$};
  \end{axis}
\end{tikzpicture}
\end{image}
If this region is rotated around the $y$-axis, it is possible, but
inconvenient, to compute the volume of the resulting solid by the
methods we have used so far. The issue is that there are two
``kinds'' of cylinderical cross-sections: 
\begin{image}
\begin{tikzpicture}
  \begin{axis}[
      xmin=-3, xmax=3,
      clip=false,
      width=4in,
      height=2in,
      axis lines =center, xlabel=$x$, ylabel=$y$,
      every axis y label/.style={at=(current axis.above origin),anchor=south},
      every axis x label/.style={at=(current axis.right of origin),anchor=west},
      axis on top,
    ]
    \addplot [penColor,very thick,domain=0:1] {x+1};
    \addplot [penColor,very thick,domain=-1:0] {-x+1};
    \addplot [penColor2,very thick,domain=-3:-.25] {(-x-1)^2};
    \addplot [penColor2,very thick,domain=.25:3] {(x-1)^2};
    
    \draw[penColor,very thick,fill=fillp] (axis cs:0,1.9) ellipse (240 and 20);
    \draw[penColor,very thick,fill=fillp] (axis cs:0,2) ellipse (240 and 20);
    \draw[penColor,very thick,fill=white] (axis cs:0,2) ellipse (100 and 10);

    \draw[penColor,very thick,fill=fillp] (axis cs:0,.46) ellipse (175 and 20);
    \draw[penColor,very thick,fill=fillp] (axis cs:0,.56) ellipse (175 and 20);
    \draw[penColor,very thick,fill=white] (axis cs:0,.56) ellipse (25 and 5);

    
    \addplot [penColor,very thick,domain=1:3] {x+1};
    \addplot [penColor2,very thick,domain=0:.25] {(x-1)^2};
    \addplot [penColor,very thick,domain=-3:-1] {-x+1};
    \addplot [penColor2,very thick,domain=-.25:0] {(-x-1)^2};

    \node at (axis cs:2,3.3) [penColor] {$f$};
    \node at (axis cs:2.5,1.7) [penColor2] {$g$};
    \addplot [very thick, penColor4] plot coordinates {(.25,.56) (1.75,.56)};
    \addplot [very thick, penColor4] plot coordinates {(1,2) (2.41,2)};
    
    \end{axis}
\end{tikzpicture}
%% \caption{A plot of $f(x) = x+1$ and $g(x) = (x-1)^2$ with the two
%%   types of ``washers'' indicated.}
%% \label{figure:washerHARD}
\end{image}
As we see above, some of the cylinderical cross sections are defined
by the line that goes from $g$ to $f$, and others are defined by the
line that touches $g$ at both ends.  To compute the volume using
accumulated cross-sections, we need to break the problem into two
intergals:
\begin{itemize}
  \item an integral that computes the volume of the region bounded by
    $g$ and the line $y=1$, rotated about the $y$-axis, and
  \item an integral that computes the volume of the region bounded by
    $f$, $g$ and the line $y=1$, rotated about the $y$-axis.
\end{itemize}
Since we are rotating around the $y$-axis, we should look at $f^{-1}$
and $g^{-1}$. Write with me:
\[
f^{-1}(y) = \answer[given]{y-1}
\]
on the other hand $g$ is \textbf{not} \index{one-to-one}one-to-one, so
we cannot invert it on the entire domain. Nevertheless, if we restrict
the domain of $g$ we may write two separate functions
\[
g^{-1}_1(y) = \answer[given]{1-\sqrt{y}}\qquad\text{when $0<x<1$}
\]
and
\[
g^{-1}_2(y) = \answer[given]{1+\sqrt{y}}\qquad\text{when $1<x<3$}.
\]
With this in mind, we can compute our volume with:
\[
\int_0^1 \pi(g^{-1}_2(y))^2-\pi(g_1^{-1}(y))^2\d y+ \int_1^4
\pi(g^{-1}_2(y))^2-\pi(f^{-1}(y))^2\d y
\]
Substituting in, we find
  \begin{align*}
  =\int_0^1 &\pi(1+\sqrt{y})^2-\answer[given]{\pi(1-\sqrt{y})^2}\d y+
  \int_1^4  \pi(1+\sqrt{y})^2-\answer[given]{\pi(y-1)^2}\d y\\
  &=\frac{8}{3}\pi + \frac{65}{6}\pi\\
  &=\answer[given]{\frac{27}{2}\pi}.
\end{align*}
While we have successfully solved this problem, it wasn't easy. Let's
see another, perhaps easier method.


If instead we consider a vertical rectangle of height $f(x)-g(x)$
(just like we did when we computed areas of regions between curves!)
and width $\d x$, and we additionally rotate this rectangle around the
$y$-axis, we get a thin shell:
\begin{image}
\begin{tikzpicture}
  \begin{axis}[
      xmin=0, xmax=3,domain=0:3,
      clip=false,
      axis lines =center, xlabel=$x$, ylabel=$y$,
      every axis y label/.style={at=(current axis.above origin),anchor=south},
      every axis x label/.style={at=(current axis.right of origin),anchor=west},
      axis on top,
    ] 
   

    \addplot [draw=none,fill=fillp!50!white] plot coordinates {(1.5,2.5) (1.5,.25) (-1.5,.25) (-1.5,2.5)};
    
    \draw[penColor,very thick,fill=fillp] (axis cs:0,2.5) ellipse (150 and 20);
    \draw[penColor,very thick,fill=white] (axis cs:0,2.5) ellipse (120 and 10);


   \draw[penColor, dashed, fill=fillp!50!white] (axis cs:0,0.25) ellipse (150 and 20);
   \draw[penColor, dashed,fill=white] (axis cs:0,0.25) ellipse (120 and 10);


     \addplot [penColor,very thick,domain=-3:0] {-x+1};
   \addplot [penColor2,very thick,domain=-3:0] {(-x-1)^2};
   
   \addplot [penColor,very thick] {x+1};
   \addplot [penColor2,very thick] {(x-1)^2};

    \node at (axis cs:2,3.3) [penColor] {$f$};
    \node at (axis cs:2.5,1.7) [penColor2] {$g$};
    \addplot [very thick, penColor4] plot coordinates {(1.5,2.5) (1.5,.25)};
    \addplot [very thick, penColor4] plot coordinates {(-1.5,2.5) (-1.5,.25)};
    \addplot [very thick, penColor4, dashed] plot coordinates {(1.2,2.5) (1.2,.25)};
    \addplot [very thick, penColor4, dashed] plot coordinates {(-1.2,2.5) (-1.2,.25)};
  \end{axis}
\end{tikzpicture}
%% \caption{A plot of $f(x) = x+1$ and $g(x) = (x-1)^2$ with the
%%   ``shell'' indicated.}
%% \label{figure:shellIndicated}
\end{image}

If we add up the volume of such thin shells we will get an
approximation to the true volume.
\begin{image}
\begin{tikzpicture}
  \begin{axis}[
      xmin=0, xmax=3,domain=0:3,
      clip=false,
      axis lines =center, xlabel=$x$, ylabel=$y$,
      every axis y label/.style={at=(current axis.above origin),anchor=south},
      every axis x label/.style={at=(current axis.right of origin),anchor=west},
      axis on top,
    ] 
    \addplot [penColor2,very thick] {x+1};
    \addplot [penColor,very thick] {(x-1)^2};

    \node at (axis cs:.5,1.7) [penColor2] {$f(x)$};
    \node at (axis cs:2.5,1.7) [penColor] {$g(x)$};
    \addplot [very thick, penColor4] plot coordinates {(1.5,2.5) (1.5,.25)};
  \end{axis}
\end{tikzpicture}
%% \caption{A plot of $f(x) = x+1$ and $g(x) = (x-1)^2$ with the
%%   ``shell'' indicated.}
%% \label{figure:shellIndicated}
\end{image}


If instead we consider a typical vertical rectangle, but still rotate
around the $y$-axis, we get a thin ``shell'' instead of a thin
``washer''. 



If we add up the volume of such thin shells we will get an
approximation to the true volume.


What is the volume of such a shell?  Consider the shell at $x$.
Imagine that we cut the shell vertically in one place and ``unroll''
it into a thin, flat sheet. This sheet will be $f(x)-g(x)$ tall, and
$2\pi x$ wide since this is the circumference of the shell before it
was unrolled.  We may now write the integral


\begin{image}
  \begin{tikzpicture}
    \begin{axis}[
          xmin =0,xmax=4,ymax=5,ymin=-5,
          axis lines=none, xlabel=$x$, ylabel=$y$,
          every axis y label/.style={at=(current axis.above origin),anchor=south},
          every axis x label/.style={at=(current axis.right of origin),anchor=west},
          axis on top,
          width=5in,
          xtick={0,6}, xticklabels={$0$, $20$},
          ytick={0,3},yticklabels={$0$,$20$},
            clip=false,
      ]

            
      %\addplot [draw=penColor, thick] plot coordinates {(-3,-3) (0,0)};
      %\addplot [draw=penColor, thick] plot coordinates {(6,0) (0,0)};
      %\addplot [draw=penColor, thick] plot coordinates {(1.5,3) (3,-3)};
      %\addplot [draw=penColor, thick] plot coordinates {(1.5,3) (0,0)};

      %% slab
      \addplot [draw=penColor, fill=fillp,very thick] plot coordinates {(3,2) (1,2) (0,1) (2, 1) (3,2)};
      \addplot [draw=penColor, fill=fillp,very thick] plot coordinates {(0,.8) (0,1) (2,1) (2, .8) (0,.8)};
      \addplot [draw=penColor, fill=fillp,very thick] plot coordinates {(2,1) (2, .8) (3,1.8) (3,2) (2,1)};

      %\addplot [draw=penColor, fill=fillp,very thick] plot coordinates {(3,1.8) (1,1.8) (0,.8) (2, .8) (3,1.8)};
      %\addplot [draw=penColor, fill=fillp,very thick] plot coordinates {(3,2) (1,2) (0,1) (2, 1) (3,2)};



      \draw[decoration={brace,mirror,raise=.1cm},decorate,thin] (axis cs:0,.8)--(axis cs:2,.8);
      \draw[decoration={brace,mirror,raise=.1cm},decorate,thin] (axis cs:2,.8)--(axis cs:3,1.8);
      \draw[decoration={brace,raise=.1cm},decorate,thin] (axis cs:3,2.05)--(axis cs:3,1.75);
      
     % \addplot [->] plot coordinates {(0,0) (-4,-4)};
      %\node[anchor=north east] at (axis cs:-4,-4) {$z$};

      \node at (axis cs:3.15,1.9) {$\d x$};
      \node at (axis cs:2.8,0.8) {$f(x)-g(x)$};
      \node at (axis cs:1,.4) {$2x\pi$};       
    \end{axis}
  \end{tikzpicture}
\end{image}



$$
  \int_0^3 2\pi x(f(x)-g(x))\d x=
  \int_0^3 2\pi x(x+1-(x-1)^2)\d x={27\over2}\pi.
$$
Not only does this accomplish the task with only one integral, the
integral is somewhat easier than those in the previous
calculation. Things are not always so neat, but it is often the case
that one of the two methods will be simpler than the other, so it is
worth considering both before starting to do calculations.


\begin{example} 
Suppose the area bounded by $y=\sqrt{x}$, the line $y = 2x-1$, and the
$x$-axis is rotated around the $x$-axis. Find the volume of this
solid.

\begin{explanation}

\begin{image}
\begin{tikzpicture}
  \begin{axis}[
      xmin=0, xmax=1.2,ymax = 1.7, ymin = -.5,domain=0:1.2,
      clip=true,
      axis lines =center, xlabel=$x$, ylabel=$y$,
      every axis y label/.style={at=(current axis.above origin),anchor=south},
      every axis x label/.style={at=(current axis.right of origin),anchor=west},
      axis on top,
    ] 
  \addplot [draw=none,fill=fillp, domain=0:1] {sqrt(x)}\closedcycle;
    \addplot [penColor,very thick, samples = 100,smooth] {sqrt(x)};
    \addplot [penColor2,very thick,domain=0.5:1.5] {2*x-1};
     \addplot [draw=none,fill=white,very thick,domain=0.5:1] {2*x-1.02}\closedcycle;
  \end{axis}
\end{tikzpicture}
\end{image}

While we could use vertical rectangles, we would have to break the region of integration up into two parts.

This indicates that using horizontal rectangles may be more profitable.

Solving for $x$, we have $x = y^2$ and $x= \frac{y+1}{2}$.

We obtain a cyclindrical shell with width $\d y$, length $\answer{\frac{y+1}{2} - y^2}$, and whose circumference is $\answer{2\pi y}$


\begin{hint}
	The radius is $y$, so the circumference is $2\pi y$.  The length is $\frac{y+1}{2} - y^2$.
\end{hint}

Thus the volume is

\[
\textrm{Volume} = \int_0^1 \answer{2\pi y (\frac{y+1}{2} - y^2)} \d y
\]
\end{explanation}
\end{example}

\begin{example}
	What if we had wanted to rotate the region from the last example about the line $y = -1$ instead of the $x$-axis?

\begin{explanation}
We could still use the same vertical rectangles.  Our cyclindrical shells would have the same width and height, but the circumference would change to $\answer{2\pi(y+1)}$.

\begin{hint}
	The radius is now $y+1$, so the circumference is $2\pi(y+1)$
\end{hint}

Thus the volume is 

\[
\textrm{Volume} = \int_0^1 \answer{2\pi (y+1) (\frac{y+1}{2} - y^2)}\d y
\]
\end{explanation}


\end{example}


\end{document}
