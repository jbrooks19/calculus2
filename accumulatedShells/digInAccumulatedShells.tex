\documentclass{ximera}

\input{../preamble.tex}

\title[Dig-In:]{Accumulated shells}

\begin{document}
\begin{abstract}
SOMETHING
\end{abstract}
\maketitle


Consider the region bounded by $f(x)=x+1$ and $g(x)=(x-1)^2$:
\begin{image}
\begin{tikzpicture}
  \begin{axis}[
      xmin=0, xmax=3,domain=0:3,
      clip=false,
      axis lines =center, xlabel=$x$, ylabel=$y$,
      every axis y label/.style={at=(current axis.above origin),anchor=south},
      every axis x label/.style={at=(current axis.right of origin),anchor=west},
      axis on top,
    ]
    \addplot [draw=none,fill=fillp] {x+1}\closedcycle;
    \addplot [draw=none,fill=white] {(x-1)^2}\closedcycle;
    \addplot [penColor,very thick] {x+1};
    \addplot [penColor2,very thick] {(x-1)^2};

    \node at (axis cs:.5,1.7) [penColor] {$f$};
    \node at (axis cs:2.5,1.7) [penColor2] {$g$};
  \end{axis}
\end{tikzpicture}
\end{image}
If this region is rotated around the $y$-axis, it is possible, but
inconvenient, to compute the volume of the resulting solid by the
methods we have used so far. The problem is that there are two
``kinds'' of cylinderical cross-sections: 
\begin{image}
\begin{tikzpicture}
  \begin{axis}[
      xmin=-3, xmax=3,
      clip=false,
      axis lines =center, xlabel=$x$, ylabel=$y$,
      every axis y label/.style={at=(current axis.above origin),anchor=south},
      every axis x label/.style={at=(current axis.right of origin),anchor=west},
      axis on top,
    ]
    \draw[penColor,very thick,fill=fillp] (axis cs:0,2) ellipse (240 and 20);
    \draw[penColor,very thick,fill=white] (axis cs:0,2) ellipse (100 and 10);

    \draw[penColor,very thick,fill=fillp] (axis cs:0,.56) ellipse (175 and 20);
    \draw[penColor,very thick,fill=white] (axis cs:0,.56) ellipse (25 and 5);

    
    \addplot [penColor,very thick,domain=0:3] {x+1};
    \addplot [penColor2,very thick,domain=0:3] {(x-1)^2};
    \addplot [penColor,very thick,domain=-3:0] {-x+1};
    \addplot [penColor2,very thick,domain=-3:0] {(-x-1)^2};

    %\node at (axis cs:.5,1.7) [penColor] {$f$};
    %\node at (axis cs:2.5,1.7) [penColor2] {$g$};
    \addplot [very thick, penColor4] plot coordinates {(.25,.56) (1.75,.56)};
    \addplot [very thick, penColor4] plot coordinates {(1,2) (2.41,2)};
    
    \end{axis}
\end{tikzpicture}
%% \caption{A plot of $f(x) = x+1$ and $g(x) = (x-1)^2$ with the two
%%   types of ``washers'' indicated.}
%% \label{figure:washerHARD}
\end{image}
As we see above, some of the cylinderical cross sections are defined
by the line that goes from the line to the parabola and others are
defined by the line that touches the parabola on both ends.  To
compute the volume using accumulated cross-sections, we need to break
the problem into two parts and compute two integrals:
\begin{align*}
  \pi\int_0^1 &(1+\sqrt{y})^2-(1-\sqrt{y})^2\d y+
  \pi\int_1^4  (1+\sqrt{y})^2-(y-1)^2\d y\\
  &={8\over3}\pi + {65\over6}\pi\\
  &={27\over2}\pi.
\end{align*}

\begin{image}
\begin{tikzpicture}
  \begin{axis}[
      xmin=0, xmax=3,domain=0:3,
      clip=false,
      axis lines =center, xlabel=$x$, ylabel=$y$,
      every axis y label/.style={at=(current axis.above origin),anchor=south},
      every axis x label/.style={at=(current axis.right of origin),anchor=west},
      axis on top,
    ] 
    \addplot [penColor2,very thick] {x+1};
    \addplot [penColor,very thick] {(x-1)^2};

    \node at (axis cs:.5,1.7) [penColor2] {$f(x)$};
    \node at (axis cs:2.5,1.7) [penColor] {$g(x)$};
    \addplot [very thick, penColor4] plot coordinates {(1.5,2.5) (1.5,.25)};
  \end{axis}
\end{tikzpicture}
%% \caption{A plot of $f(x) = x+1$ and $g(x) = (x-1)^2$ with the
%%   ``shell'' indicated.}
%% \label{figure:shellIndicated}
\end{image}


If instead we consider a typical vertical rectangle, but still rotate
around the $y$-axis, we get a thin ``shell'' instead of a thin
``washer,'' see Figure~\ref{figure:shellIndicated}. If we add up the
volume of such thin shells we will get an approximation to the true
volume.

%% \begin{image}
%% \begin{tikzpicture}
%%  \begin{axis}[
%%      view={30}{30},colormap/\surfaceColor,
%%      xlabel=$x$, ylabel=$z$, zlabel=$y$,
%%    ]

%%   \addplot3[surf,colormap/\surfaceColorTwo,shader=faceted,opacity=.7, %inside
%%   samples=10,
%%   samples y =20,
%%   domain=0:3,y domain=0:2*pi,
%%   z buffer=sort]
%%   (x* cos(deg(y)), {(x) * sin(deg(y))},{x+1});

%%   \addplot3[surf,colormap/\sliceColor,shader=interp,%shell
%%   samples=5,
%%   samples y = 20,
%%   domain=.25:2.5,y domain=0:2*pi,
%%   z buffer=sort
%% ]
%%   ({1.5*cos(deg(y))}, {1.5*sin(deg(y))},x);


%%   \addplot3[surf,shader=faceted,opacity=.3, %outside
%%   samples=10,
%%   samples y =20,
%%   domain=0:3,y domain=0:2*pi,
%%   z buffer=sort]
%%   (x*cos(deg(y)), {(x) * sin(deg(y))},{(x-1)^2});


%%  \end{axis}
%% \end{tikzpicture}
%% \end{image}



What is the volume of such a shell?  Consider the shell at
$x$.  Imagine that we cut the shell vertically in one place and
``unroll'' it into a thin, flat sheet. This sheet will be $f(x)-g(x)$
tall, and $2\pi x$ wide namely, the circumference of the shell before
it was unrolled.  We may now write the integral
$$
  \int_0^3 2\pi x(f(x)-g(x))\d x=
  \int_0^3 2\pi x(x+1-(x-1)^2)\d x={27\over2}\pi.
$$
Not only does this accomplish the task with only one integral, the
integral is somewhat easier than those in the previous
calculation. Things are not always so neat, but it is often the case
that one of the two methods will be simpler than the other, so it is
worth considering both before starting to do calculations.


\section{Accumulation of shells}


\begin{example} 
Suppose the area under $y=-x^2+1$ between $x=0$ and $x=1$ is rotated
around the $x$-axis.
\begin{explanation}
We'll just set up integrals for each method.

Disk method: $\int_0^1 \pi(1-x^2)^2\d x={8\over15}\pi$.


Shell method: $\int_0^1 2\pi y \sqrt{1-y}\d y={8\over15}\pi$.
\end{explanation}
\end{example}

\section{Revolving around other lines}


\end{document}
