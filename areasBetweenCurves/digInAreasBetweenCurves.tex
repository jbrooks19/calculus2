\documentclass{ximera}

\title[Dig-In:]{Area between curves}

\begin{document}
\begin{abstract}
  Here
\end{abstract}
\maketitle


We have seen how integration can be used to find signed area between a
curve and the $x$-axis. With very little change we can find some areas
between curves. Let's see an example:

\begin{example} Find the area below $f(x)= -x^2+4x+3$ and above
$g(x)=-x^3+7x^2-10x+5$ over the interval $1\le x\le2$. 
\end{example}

\begin{marginfigure}
\begin{tikzpicture}
	\begin{axis}[
            domain=0:3, ymax=14,xmax=3,ymin=0, xmin=0,
            axis lines =left, xlabel=$x$, ylabel=$y$,
            every axis y label/.style={at=(current axis.above origin),anchor=south},
            every axis x label/.style={at=(current axis.right of origin),anchor=west},
            axis on top,
          ]
          \addplot [draw=none,fill=fillp,domain=1:2] {-x^2+4*x+3} \closedcycle;
          \addplot [draw=none,fill=background,domain=1:2] {-x^3 + 7*x^2-10*x+5} \closedcycle;
          \addplot [draw=penColor,very thick] {-x^2+4*x+3};
          \addplot [draw=penColor2,very thick] {-x^3 + 7*x^2-10*x+5};
          \node at (axis cs:1,6.7) [penColor] {$f(x)$};
          \node at (axis cs:2,4) [penColor2] {$g(x)$};
        \end{axis}
\end{tikzpicture}
\caption{The area below $f(x)= -x^2+4x+3$ and above
$g(x)=-x^3+7x^2-10x+5$ over the interval $1\le x\le2$. }
\label{fig:area between curves}
\end{marginfigure}

\begin{solution}
In Figure~\ref{fig:area between curves} we show the two curves
together, with the desired area shaded.

It is clear from the figure that the area we want is the area under
$f(x)$ minus the area under $g(x)$, which is to say
\[
\int_1^2 f(x)\d x-\int_1^2 g(x)\d x = \int_1^2 \left(f(x)-g(x)\right)\d x.
\]
It doesn't matter whether we compute the two integrals on the left and
then subtract or compute the single integral on the right. In this
case, the latter is perhaps a bit easier:
\begin{align*}
  \int_1^2 f(x)-g(x)\d x&=\int_1^2 -x^2+4x+3-(-x^3+7x^2-10x+5)\d x \\
  &=\int_1^2 x^3-8x^2+14x-2\d x \\
  &=\left.{x^4\over4}-{8x^3\over3}+7x^2-2x\right|_1^2 \\
  &={16\over4}-{64\over3}+28-4-({1\over4}-{8\over3}+7-2) \\
  &=23-{56\over3}-{1\over4}={49\over12}.
\end{align*}
\end{solution}

In our first example, one curve was higher than the other over the
entire interval. This does not always happen.


\begin{example} Find the area between $f(x)= -x^2+4x$ and
$g(x)=x^2-6x+5$ over the interval $0\le x\le 1$.
\end{example}

\begin{marginfigure}
\begin{tikzpicture}
	\begin{axis}[
            domain=-1:2, ymax=6,xmax=1.5,ymin=0, xmin=-.5,
            axis lines =center, xlabel=$x$, ylabel=$y$,
            every axis y label/.style={at=(current axis.above origin),anchor=south},
            every axis x label/.style={at=(current axis.right of origin),anchor=west},
            axis on top,
          ]
          \addplot [draw=none,fill=fillp,domain=0:0.56] {x^2-6*x+5} \closedcycle;
          \addplot [draw=none,fill=background,domain=0:0.56] {-x^2+4*x} \closedcycle;
          \addplot [draw=none,fill=fillp,domain=0.56:1] {-x^2+4*x} \closedcycle;
          \addplot [draw=none,fill=background,domain=0.56:1] {x^2-6*x+5} \closedcycle;
          \addplot [draw=penColor,very thick] {-x^2+4*x};
          \addplot [draw=penColor2,very thick] {x^2 - 6*x+5};
          \node at (axis cs:1.25,3.1) [penColor] {$f(x)$};
          \node at (axis cs:.25,4.3) [penColor2] {$g(x)$};
        \end{axis}
\end{tikzpicture}
\caption{The area between $f(x)= -x^2+4x$ and
$g(x)=x^2-6x+5$ over the interval $0\le x\le 1$.}
\label{fig:curves cross}
\end{marginfigure}


\begin{solution}
The curves are shown in Figure~\ref{fig:curves cross}. Generally we
should interpret ``area'' in the usual sense, as a necessarily
positive quantity. Since the two curves cross, we need to compute two
areas and add them. First we find the intersection point of the
curves:
\begin{align*}
  -x^2+4x&=x^2-6x+5 \\
  0&=2x^2-10x+5 \\
  x&={10\pm\sqrt{100-40}\over4}={5\pm\sqrt{15}\over2}.
\end{align*}
The intersection point we want is $x=a=(5-\sqrt{15})/2$. Then
the total area is 
\begin{align*}
  \int_0^a x^2-6x+5-(-x^2+4x)\d x&+\int_a^1 -x^2+4x-(x^2-6x+5)\d x \\
  &=\int_0^a 2x^2-10x+5\d x+\int_a^1 -2x^2+10x-5\d x \\
  &=\left.{2x^3\over3}-5x^2+5x\right|_0^a + 
    \left.-{2x^3\over3}+5x^2-5x\right|_a^1 \\
  &=-{52\over3}+5\sqrt{15},
\end{align*}
after a bit of simplification.
\end{solution}


In both of our examples above, we gave you the limits of integration
by bounding the $x$-values between $0$ and $1$. However, some problems
are not so simple.

\begin{example} Find the area between $f(x)= -x^2+4x$ and
$g(x)=x^2-6x+5$.
\end{example}

\begin{marginfigure}
\begin{tikzpicture}
	\begin{axis}[
            domain=0:5, ymax=5,xmax=5,ymin=-5, xmin=0,
            axis lines =center, xlabel=$x$, ylabel=$y$,
            every axis y label/.style={at=(current axis.above origin),anchor=south},
            every axis x label/.style={at=(current axis.right of origin),anchor=west},
            axis on top,
          ]
          \addplot [draw=none,fill=fillp,domain=.56:4] {-x^2+4*x} \closedcycle;
          \addplot [draw=none,fill=fillp,domain=.56:4.44] {x^2-6*x+5} \closedcycle;
          \addplot [draw=none,fill=background,domain=4:5] {-x^2+4*x} \closedcycle;
          \addplot [draw=none,fill=background,domain=0:1] {x^2-6*x+5} \closedcycle;
          %\addplot [draw=none,fill=fillp,domain=.56:4] {-x^2+4*x} \closedcycle;       
          \addplot [draw=penColor,very thick,smooth] {-x^2+4*x};
          \addplot [draw=penColor2,very thick,smooth] {x^2-6*x+5};
          
          \node at (axis cs:2,4.4) [penColor] {$f(x)$};
          \node at (axis cs:1,-1) [penColor2] {$g(x)$};
        \end{axis}
\end{tikzpicture}
\caption{The area between $f(x)= -x^2+4x$ and $g(x)=x^2-6x+5$.}
\label{fig:area bounded by curves}
\end{marginfigure}

\begin{solution}
The curves are shown in Figure~\ref{fig:area bounded by curves}. Here
we are not given a specific interval, so it must be the case that
there is a ``natural'' region involved. Since the curves are both
parabolas, the only reasonable interpretation is the region between
the two intersection points, which we found in the previous example:
$${5\pm\sqrt{15}\over2}.$$
If we let $a=(5-\sqrt{15})/2$ and $b=(5+\sqrt{15})/2$,
the total area is 
\begin{align*}
  \int_a^b -x^2+4x-(x^2-6x+5)\d x
  &=\int_a^b -2x^2+10x-5\d x \\
  &=\left.-{2x^3\over3}+5x^2-5x\right|_a^b \\
  &=5\sqrt{15},
\end{align*}
after a bit of simplification.
\end{solution}


\end{document}
