\documentclass{ximera}

\input{../preamble.tex}

\outcome{Compute limits of vectored valued functions.}
\outcome{Determine continuity of vector valued functions.}
\outcome{Compute the derivative of a vector valued function.}
\outcome{Compute the tangent vector of a vector valued function.}
\outcome{Find the unit tangent vector of a vector valued function.}
\outcome{Compute integrals of vector valued functions.}


\title[Dig-In:]{Beginnings of vector calculus}

\begin{document}
\begin{abstract}
  With one input, we work component-wise.
\end{abstract}
\maketitle

\section{Component-wise}


\section{Limits of vector valued functions}

\begin{definition}
  Let $\vec{f}:\R \to \R^3$ be a vector valued function.  Then we say
  that the limit of $f$ as $t$ approaches $a$ is $\vec{L}$ if
  \[
  \lim_{t \to a} \left| f(t) - \vec{L}\right| = 0
  \]
  In this case we write
  \[
  \lim_{t \to a} f(t) = \vec{L}.
  \]
\end{definition}

We can evaluate limits by just taking the limit of each component
separately.

\begin{question}
  Let $f(t) = \vector{\sin(t),\cos(t),\frac{\sin(t)}{t}}$.  
  \[
  \lim_{t \to 0} f(t) = \vector{\answer{0},\answer{1},\answer{1}}
  \]
  \begin{hint}
    Taking the limit of each component separately, we have $\vector{0,1,1}$.
  \end{hint}
\end{question}


This also lets us define the concept of continuity of vector valued functions: 

\begin{definition}
  A vector valued function $\vec{f}$ is \dfn{continuous} at $t= a$ if
  and only if
  \[
  \lim_{t \to a} \vec{f}(t)  = \vec{f}(a)
  \]
\end{definition}
Because of the component-wise nature of limits, we can see that a
function $\vec{f}$ is continuous if and only if each of its component
functions $x(t)$, $y(t)$, $z(t)$ is also continuous at $t=a$.


Find $a,b,c$ that make a function continuous. 




\end{document}
