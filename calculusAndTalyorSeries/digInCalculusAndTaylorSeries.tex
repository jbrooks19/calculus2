\documentclass{ximera}

\input{../preamble.tex}

\outcome{Use Taylor series to read-off derivatives of a function.}
\outcome{Use Taylor series to solve differential equations.}
\outcome{Use Taylor series to compute integrals}

\title[Dig-In:]{Calculus and Taylor series}

\begin{document}
\begin{abstract}
  Power series interact nicely with other calculus concepts
\end{abstract}
\maketitle

\section{Differentiating power series}

The following theorem is beyond what we will cover in the course (although, if you are industrious, it is within your grasp.  See \href{https://gowers.wordpress.com/2014/02/22/differentiating-power-series/}{this blog post} if you are interested) 

\begin{theorem}
	If $f(x) = \sum_0^\infty c_n(x-a)^n$ is a power series with radius of convergence $R$, then $f'(x)$ is equal to  $\sum_1^\infty nc_n(x-a)^{n-1}$, and the radius of convergence of the new series is still $R$.  The new series is said to be obtained from the old series by ``differentiating the series term by term''.  The new series may lose (but not gain) convergence at the endpoints $a+R$ and $a-R$.
\end{theorem}

\begin{example}
	Say we want to find the power series for $\frac{3}{(1-x)^2}$.  We already know that the power series for $\frac{1}{1-x}$ is
	
	\[
	\frac{1}{1-x} = 1+x+x^2+x^3+\dots
	\]
	
	So differentiating both sides we get
	
	\[
	\answer{\frac{1}{(1-x)^2}} = 1+\answer{2}x+\answer{3}x^2+\dots
	\]
	
	And thus
	
	\[
	\frac{3}{(1-x)^2} = 3+\answer{6}x+\answer{9}x^2+\dots
	\]
	
	Or, in more compact notation
	
	\[
	\frac{3}{(1-x)^2} = \sum_1^\infty \answer{3n}x^{n-1}
	\]
	
	We already know that the radius of convergence of $\frac{1}{1-x}$ is $1$, so we get that this new series also has radius of convergence $1$ by the theorem.
\end{example}

\begin{question}
	If $\sum_0^\infty c_n$ converges, must $\sum_1^\infty nc_n(\frac{1}{2})^n$ also converge?
	
	\begin{multipleChoice}
		\choice[correct]{Yes}
		\choice{No}
	\end{multipleChoice}
	
	\begin{hint}
		We can think of $\sum_0^\infty c_n$ as $\sum_0^\infty c_n x^n$ evaluated at $x=1$.  Since this converges, the radius of convergence of the series is at least $1$.  Thus the radius of convergence of its derivative $\sum_1^\infty nc_n x^n$ is also at least $1$.  Thus $\sum_1^\infty nc_n(\frac{1}{2})^n$ converges.
	\end{hint}
\end{question}


\section{Solving differential equations using power series}

If we have a differential equation we can frequently use the ``method of undetermined coefficients'' to solve it using power series.

\begin{example}
Say we want to solve the differential equation

\[
y' = 2y-3
\]

subject to the initial condition that $y(0)=1$

Assume that $y$ can be represented by a power series

\[
y = c_0+c_1x+c_2x^2+c_3x^3+\dots
\]

Then, differentiating both sides we have

\[
y' = c_1+2c_2x+3c_3x^2+\dots
\]

Using the differential equation, we know that

\[
2y-3 = c_1+2c_2x+3c_3x^2+\dots
\]

so

\[
2c_0-3+2c_1x+2c_2x^2+2c_3x^3+\dots = c_1+2c_2x+3c_3x^2+\dots
\]



This is equivalent to infinitely many simultaneous equations:

\begin{align*}
	2c_0-3 &= c_1\\
	2c_1 &= 2c_2\\
	2c_2 &= 3c_3\\
	2c_3 &=4c_4\\
	\vdots
\end{align*}

We know that $c_0 = 1$ since this is the initial condition. 
This lets us find

\begin{align*}
c_0 = 1\\
c_1 = \answer{-1}\\
c_2=\answer{-1}\\
c_3=\answer{-\frac{2}{3}}\\
c_4=\answer{-\frac{1}{3}}\\
\vdots
\end{align*}

In fact, since $2c_{n-1} = nc_n$, we can find $c_n = \frac{2}{n}c_{n-1}$, so $c_n = -\frac{2^{n-1}}{n!}$ for $n \geq 1$.

Thus the solution is 

\[
y = 1+\sum_1^\infty -\frac{2^{n-1}}{n!} x^n
\]

This looks a lot like the series for $e^{2x}$, so we can try to manipulate this into a closed form solution.

\[
e^{2x} = \sum_0^\infty \frac{2^n}{n!} x^n
\]

so 

\begin{align*}
-\frac{1}{2} e^{2x} &= \sum_0^\infty -\frac{2^{n-1}}{n!} x^n\\
-\frac{1}{2} e^{2x} &= -\frac{1}{2}+ \sum_1^\infty -\frac{2^{n-1}}{n!} x^n\\
-\frac{1}{2} e^{2x} +\frac{3}{2}&= 1+ \sum_1^\infty -\frac{2^{n-1}}{n!} x^n\\
y &= -\frac{1}{2} e^{2x} +\frac{3}{2}\\
\end{align*}

So we have found a closed form solution

\[
y = -\frac{1}{2} e^{2x} +\frac{3}{2}
\]

Make sure to check that this is really a solution to the differential equation!

\end{example}

\section{Integration}

Just as we can differentiate term by term, we can also integrate term by term

\begin{theorem}
	If $f(x) = \sum_0^\infty c_n(x-a)^n$ is a power series with radius of convergence $R$, then the antiderivatives of $f(x)$ are given by $C + \sum_0^\infty \frac{c_n}{n+1}(x-a)^{n+1}$ where $C$ is an arbitrary constant, and the radius of convergence of the new series is still $R$.  The new series is said to be obtained from the old series by ``integrating the series term by term''.  The new series may gain (but not lose) convergence at the endpoints $a+R$ and $a-R$.
\end{theorem}

\begin{example}
The antiderivative of $ f(x) =\frac{\sin(x)}{x}$ cannot be expressed as an elementary function.  However, its integral

\[
Si(x) = \int_0^x f(t) \d t
\]

is important in signal analysis.

We can find a series expression for $Si(x)$ using term by term integration:

\[
\sin(x) = x-\frac{x^3}{3!}+\frac{x^5}{5!}-\dots = \sum_0^\infty \frac{1}{(2n+1)!} x^{2n+1}
\]

So 

\[
\frac{\sin(x)}{x} =1-\frac{x^2}{3!}+\frac{x^5}{5!}-\dots = \sum_0^\infty \frac{1}{(2n+1)!} x^{2n}
\]

Thus

\[
\int \frac{\sin(x)}{x} \d x = C+x-\frac{x^3}{3 \cdot 3!}+\frac{x^5}{5 \cdot 5!}-\dots = \sum_0^\infty \frac{1}{(2n+1) \cdot (2n+1)!} x^{2n+1}
\]

$Si(x)$ is the antiderivative of $f(x)$ with $Si(0) = 0$, so  $C=0$, and 

\[
Si(x) = \sum_0^\infty \frac{1}{(2n+1) \cdot (2n+1)!} x^{2n+1}
\]

Say we wanted to approximate $Si(2)$ with an accuracy of $\frac{1}{10}$.  How many terms would we need?

By the alternating series estimation test, we would need

\[\frac{1}{(2n+1) \cdot (2n+1)!} 2^{2n+1} \leq  \frac{1}{10}\]

We can experiment a bit to find that the least integer which makes this true is 

\[
 n= \answer{2}
\]

Thus $Si(2) \approx 2-\frac{8}{3 \cdot 3!} = 1.\bar{5}$ should be accurate to within $\frac{1}{10}$.  Wolfram alpha reports that $Si(2) \approx 1.60541 $, we this appears to be accurate!
\end{example}




\end{document}
