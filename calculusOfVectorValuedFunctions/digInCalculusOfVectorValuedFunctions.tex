\documentclass{ximera}

\input{../preamble.tex}

\outcome{Compute limits of vectored valued functions.}
\outcome{Determine continuity of vector valued functions.}
\outcome{Compute the derivative of a vector valued function.}
\outcome{Compute the tangent vector of a vector valued function.}
\outcome{Find the unit tangent vector of a vector valued function.}
\outcome{Compute integrals of vector valued functions.}


\title[Dig-In:]{Beginnings of vector calculus}

\begin{document}
\begin{abstract}
  With one input, we work component-wise.
\end{abstract}
\maketitle


Since we are currently thinking about vector valued functions that
only have a single input, we can work \dfn{component-wise}. This
means that when working with vector valued functions, we simply treat
each component as a regular, single variable function. Let's see what
we mean.

\section{Limits of vector valued functions}

\begin{definition}
  Let $\vec{f}:\R \to \R^3$ be a vector valued function
  \[
  \vec{f}(t) = \vector{x(t),y(t),z(t)}
  \]
  then the vector-valued \dfn{limit} as $t$ goes to $a$ is given by 
  \[
  \lim_{t \to a} \vec{f}(t) = \vector{\lim_{t \to a}x(t),\lim_{t \to a}y(t),\lim_{t \to a}z(t)}.
  \]
  This limit exists if and only if each of the limits of the
  components exist.
\end{definition}

We evaluate limits by just taking the limit of each component
separately.

\begin{question}
  Let $\vec{f}(t) = \vector{\sin(t),\cos(t),\frac{\sin(t)}{t}}$.
  Compute:
  \[
  \lim_{t \to 0} \vec{f}(t)
  \begin{prompt}
    =\vector{\answer{0},\answer{1},\answer{1}}
  \end{prompt}
  \]
  \begin{hint}
    Take the limit of each component separately.
  \end{hint}
\end{question}

Now that we have the notion of limits, we may also define the concept
of continuity of vector valued functions:

\begin{definition}
  A vector valued function $\vec{f}$ is \dfn{continuous} at $t= a$ if
  and only if
  \[
  \lim_{t \to a} \vec{f}(t)  = \vec{f}(a)
  \]
\end{definition}
Because of the component-wise nature of limits, we can see that a
function $\vec{f}$ is continuous if and only if each of its component
functions $x(t)$, $y(t)$, $z(t)$ is also continuous at $t=a$.

\begin{question}

Find $a,b,c$ that make a function continuous. 

\end{question}


\section{Derivatives}


\section{Integrals}


\end{document}
