\documentclass{ximera}

\input{../preamble.tex}

\outcome{Use the comparison test to determine if a series diverges or converges}

\title[Dig-In:]{The Comparison Test}

\begin{document}
\begin{abstract}
We compare infinite series to each other using inequalities
\end{abstract}
\maketitle

\begin{question}
	If  $0 \leq a_k \leq b_k$ for all $k$, which of the following do you think are true?
	
	\begin{multipleChoice}
		\choice[correct]{If $\sum b_k$ converges then $\sum a_k$ converges}
		\choice{If $\sum a_k$ converges then $\sum b_k$ converges}
		\choice{If $\sum b_k$ diverges then $\sum a_k$ diverges}
		\choice[correct]{If $\sum a_k$ diverges then $\sum b_k$ diverges}
	\end{multipleChoice}
\end{question}

If you got this question right, you already understand the Comparison Test:

\begin{theorem}[The Comparison Test]
	Let $\sum a_k$ and $\sum b_k$ be series with positive terms.
	
	\begin{itemize}
		\item If $a_k \leq b_k$ and $\sum a_k$ is divergent, then $\sum b_k$ is divergent
		\item If $a_k \leq b_k$ and $\sum b_k$ is convergent, then $\sum a_k$ is convergent.
	\end{itemize}
\end{theorem}

The proof of this theorem is beyond this course, but it should make intuitive sense.  Making the terms of a convergent series smaller should result in another convergent series.  Making the terms of  divergent series larger should result in another divergent series.

While this theorem is intuitive, its use involves considerable creativity.  You have to:

\begin{itemize}
	\item Find a simpler series which ``behaves like'' your series 
	\item Use this simpler series to predict whether the original series converges or diverges
	\item If you predict your series is convergent, you have to hunt for a simpler series which is also convergent, but all of whom's  terms are larger
	\item If you predict your series is divergent, you have to hunt for a simpler series which is also divergent, but all whom's terms are smaller
\end{itemize}

All of these steps (aside from the second) require real insight and creativity.  This is not a ``mechanical'' test.

\begin{question}
Is $\sum_1^\infty \frac{k}{1+k^3}$ convergent or divergent?  Justify your answer using the Comparison Test.

\begin{multipleChoice}	
	\choice[correct]{convergent}
	\choice{divergent}
\end{multipleChoice}

\begin{hint}
	Intuitively, we should feel that this should be similar to the series $\sum \frac{k}{k^3} = \sum \frac{1}{k^2}$ since the $1$ should not matter that much.
	
	We know $\sum \frac{1}{k^2}$ is convergent by the $p$-series test (which ultimately comes from the integral test).
	
	Since we want to prove convergence, we need to find another series whose terms are always \textbf{larger} than our original series, but which converges. 
\end{hint}

\begin{hint}
	 $\frac{k}{1+k^3} < \frac{k}{0+k^3}$ since we have made the denominator smaller, which makes the fraction larger.
	 
	 Thus $\frac{k}{1+k^3} < \frac{1}{k^2}$.
	 
	 Since $\sum \frac{1}{k^2}$ converges, then our original series also converges by the comparison test.
\end{hint}

\end{question}

\begin{question}
Is $\sum_1^\infty \frac{k^2}{1+k^3}$ convergent or divergent?  Justify your answer using the Comparison Test.

\begin{multipleChoice}	
	\choice{convergent}
	\choice[correct]{divergent}
\end{multipleChoice}

\begin{hint}
	Intuitively, we should feel that this should be similar to the series $\sum \frac{k^2}{k^3} = \sum \frac{1}{k}$ since the $1$ should not matter that much.
	
	We know $\sum \frac{1}{k}$ is divergent by the $p$-series test (which ultimately comes from the integral test).
	
	Since we want to prove divergence, we need to find another series whose terms are always \textbf{smaller} than our original series, but which also diverges.  This requires some creativity. 
\end{hint}

\begin{hint}
	 $\frac{k^2}{1+k^3} \geq \frac{k^2}{k^3+k^3}$ since we have made the denominator larger, which makes the fraction smaller. (Note that $k^3 \geq 1$ whenever $k \geq 1$)
	 
	 Thus $\frac{k}{1+k^3} \geq \frac{1}{2k}$.
	 
	 Since $\sum \frac{1}{2k}$ diverges, then our original series also diverges by the comparison test.
\end{hint}

\end{question}


\end{document}


