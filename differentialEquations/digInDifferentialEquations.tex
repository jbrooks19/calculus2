\documentclass{ximera}

\input{../preamble.tex}


\outcome{Identify a differential equation.}
\outcome{Verify a solution to a differential equation.}
\outcome{Compute a general solution to a differential equation via integration.}
\outcome{Solve initial value problems.}
\outcome{Determine the order of a differential equation.}


\title[Dig-In:]{Differential equations}

\begin{document}
\begin{abstract}
  Differential equations show you relationships between rates of
  functions.
\end{abstract}
\maketitle

A \textit{differential equation}\index{differential equation} is
simply an equation with a derivative in it. Here is an example:
\[
a\cdot f''(x) + b\cdot f'(x) + c\cdot f(x) = g(x). 
\]
\begin{question}
  What is a differential equation?
  \begin{multipleChoice}
    \choice{An equation that you take the derivative of.}
    \choice[correct]{An equation that relates the rate of a function to other values.}
    \choice{It is a formula for the slope of a tangent line at a given point.}  
  \end{multipleChoice}
\end{question}

When a mathematician solves a differential equation, they are finding
\textit{functions} satisfying the equation.
\begin{question}
  Which of the following functions solve the differential equation
  \[
  f^{(4)}(x) = f(x)?
  \]
  \begin{hint}
    Remember, $f^{(4)}$ is the fourth derivative of $f$.
  \end{hint}
  \begin{selectAll}
    \choice[correct]{$f(x) = \sin(x)$}
    \choice{$f(x) = x^2$}
    \choice[correct]{$f(x) = e^x$}
    \choice[correct]{$f(x) = e^{-x}$}
    \choice{$f(x) = \tan(x)$}
  \end{selectAll}
\end{question}


It turns out that the complete solution to this differential equation
is $c_1\sin(x)+c_2\cos(x)+c_3e^x+c_4e^{-x}$.  In other words, every
solution to this differential equation can be written in this form.
You should check that these are all solutions (for example $f(x) =
\sin(x)+3\cos(x)-7e^x+\pi e^{-x}$ is a solution).  Proving that these
are \textbf{all} of the solutions is beyond the scope of this course.

Differential equations are one of the most practical objects of
mathematical study.  They appear constantly in every field of science
and engineering.  They are a powerful way to model many diverse
situations.

\begin{question}
  Imagine that a glass of water has initial temperature $5^\circ
  \unit{C}$, and that the ambient temperature is $22^\circ \unit{C}$.
  The water will warm up over time.  Assume that the rate of change in
  the temperature of the water is directly proportional to the
  difference between the current water temperature and the ambient
  temperature.  Which of the following differential equations must be
  satisfied by the function $H(t)$ which measures the temperature of
  the water with respect to time?
	
	\begin{multipleChoice}
		\choice{$y' = 5+\frac{y}{22}$}
		\choice[correct]{$y' = k(22-y)$ for some $k>0$}
		\choice{$y' = k(y-22)$ for some $k>0$}
		\choice{$y' = k(5-y)$ for some $k>0$}
		\choice{$y' = k(y-5)$ for some $k>0$}
	\end{multipleChoice}
	
	\begin{hint}
	  This is just a straight translation job.  ``The rate of change in the temperature of the water" is $y'$.  ``Directly proportional to" means that it is equal to some constant (say $k$) times whatever it is proportional to.  ``The difference between the current water temperature and the ambient temperature" is either $22-y$ or $y-22$, since $y$ is the temperature of the water and $22$ is the ambient temperature.   Think about which we should choose before looking at the next hint.  Will it be $y'=k(22-y)$ or $y'=k(y-22)$ where $k>0$?
	\end{hint}
	
	\begin{hint}
		Since the temperature of the water is increasing over time, we want $y'>0$.  Since the temperature will be increasing (and it is reasonable to assume it never surpasses the ambient temperature!)  $22-y$ is positive.  So we can conclude that $y' = k(22-y)$ for some $k>0$.  
	\end{hint}

	\begin{feedback}
		The differential equation does not involve the number $5$.  If we wanted to incorporate that piece of data into our model we could ask ``Which solution(s) to this differential equation satisfy $H(0) = 5$"?  This is known as an \textbf{initial value problem}. 
	\end{feedback}
\end{question} 

\begin{question}
	One can approximate the force of gravity as constant near the Earth.  So the acceleration of a falling object is a constant $g>0$.  If $h(t)$ is the height of an object at time $t$, which differential equation must $h$ satisfy?
	
	\begin{multipleChoice}
		\choice{y' = g}
		\choice{y=-g}
		\choice[correct]{y''=-g}
		\choice{y'' = g}
		\choice{y'' = -gy}
	\end{multipleChoice}
	
	\begin{hint}
		The acceleration of an object is the second derivative of its position, so the differential equation should say the second derivative ($y''$) is constant.  Should it be a positive of negative constant?
	\end{hint}
	\begin{hint}
		A falling object will fall quicker and quicker, so the second derivative of its height should be negative.  Thus $y''=g$ is the correct answer.
	\end{hint}	
\end{question}

We have already seen, and solved, a particular kind of differential equation in this course.  Namely a solution $F$ to the differential equation $y' = f(x)$ is just an antiderivative of $f$!  We know the ``general solution'' of this differential equation is just $F(x)+C$, as long as the domain of $f$ is an interval.  We can use this idea to solve differential equations of the form $y^{(n)} = f(x)$, by just repeatedly integrating.

\begin{example}
Let's find the general solution to the differential equation $y'' =  x$, and then find the particular solution which passes through the points $(0,1)$ and $(1,2)$.

Since $y'' = x$, we know that $y' = \int x \d x$. 

 Thus $y' = \frac{1}{2} x^2 + C_1$ for some constant $C_1$.  
 
Now this further implies that $y = \int \frac{1}{2} x^2 + C_1 \d x$, so 
we must have that $y = \frac{1}{6}x^3+C_1x+C_2$, for some constant $C_2$.

This is the general solution of the differential equation.  We have shown that every solution is of the form $y = \frac{1}{6}x^3+C_1x+C_2$.

Now to find the particular solution we are interested in, we can just solve a system of equations.

We find that the only solution to this differential equation passing through $(0,1)$ and $(1,2)$ is 

\[
y = \answer{\frac{1}{6}x^3+\frac{5}{6}x+1}
\]

\begin{hint}
	Substituting $(0,1)$ into the equation yields $C_2 = 1$, and then substituting $(1,2)$ into the equation one can solve for $C_1$ to find $C_1 = \frac{5}{6}$
\end{hint}


\end{example}



\end{document}
