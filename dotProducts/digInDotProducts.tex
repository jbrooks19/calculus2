\documentclass{ximera}

\input{../preamble.tex}


\outcome{Compute dot products.}
\outcome{Use dot products to compute the angle between vectors.}
\outcome{Find orthogonal projections.}
\outcome{Use the dot product in applied settings.}

\title[Dig-In:]{Dot Product}

\begin{document}
\begin{abstract}
  The dot product is one way to multiply two vectors
\end{abstract}
\maketitle

We have already seen how to add vectors, and how to multiply vectors by scalars.

\begin{warning}
We have not yet defined how to multiply a vector by a vector.  You might think it is reasonable to define 


\[
\begin{bmatrix}x_1\\x_2\\\vdots\\x_n\end{bmatrix} \cdot  \begin{bmatrix}y_1\\y_2\\\vdots\\y_n\end{bmatrix}=\begin{bmatrix}x_1y_1\\x_2y_2\\\vdots\\x_ny_n\end{bmatrix}
\] 
\begin{center}
	\Large{Wrong Definition of Product}
\end{center}

but actually this operation is not especially useful, and will \textbf{never be utilized in this course}.  
\end{warning}

In this section we will define a way to ``multiply'' two vectors called the ``dot product''.

\section{The Algebra of the Dot Product}
Instead of defining the dot product by a formula, we will define it by some nice properties it satisfies.  This is a little bit abstract, but is a common practice in mathematics.  We will then see that there is only one formula which gives us all of these properties.

\begin{definition}
	The \dfn{dot product} is an operation which takes two vectors in $\mathbb{R^n}$ and produces a real number (not another vector!).   We write $\vec{v} \cdot \vec{w}$ for the dot product between $v$ and $w$. It is the only operation which obeys the following rules:
	
	If $\vec{u},\vec{v},\vec{w} \in \R^n$ and $a \in \R$
	
\begin{enumerate}
\item $\vec{v} \cdot \vec{w} = \vec{w} \cdot \vec{v}$ (The dot product is commutative)
			
\item $(\vec{u}+\vec{v})\cdot \vec{w} = \vec{u}\cdot \vec{w} + \vec{v}\cdot \vec{w}$ 	and $(a\vec{v})\cdot \vec{w} = a(\vec{v} \cdot \vec{w})$ (The dot product is linear in the first argument)
			
\item $\vec{u} \cdot (\vec{v}+\vec{w}) = \vec{u}\cdot \vec{v}+ \vec{u}\cdot \vec{w}$ and  $\vec{v} \cdot (a\vec{w}) = a(\vec{v} \cdot  \vec{w})$ (The dot product is linear in the second argument)
			
\item $\vec{v} \cdot \vec{v} = |v|^2$ (The dot product of a vector with itself is the square of its length)
			
\item If  $\vec{v}$ is perpendicular to $\vec{w}$ then $\vec{v} \cdot \vec{w} = 0$.  
\end{enumerate}

\end{definition}

\begin{question}
	Which of the following expressions make sense?  Remember that the dot product of two vectors is a scalar, not another vector.
	
	\begin{selectAll}
		\choice[correct]{$\mathbf{v} \cdot \mathbf{u} \mathbf{u}$}
		\choice[correct]{$5(\mathbf{u} +\mathbf{v}) \cdot {\mathbf{u}}$}
		\choice{$\mathbf{v} \cdot \mathbf{u} \cdot \mathbf{w}$}
		\choice[correct]{$\begin{bmatrix}  2 \\ 3\end{bmatrix} \cdot \begin{bmatrix}  4 \\ 2\end{bmatrix} + 7$}
		\choice[correct]{$\mathbf{v} / ( \mathbf{u} \cdot \mathbf{u})$}
		\choice{$\begin{bmatrix}  1 \\ 3\end{bmatrix} \cdot \begin{bmatrix}  -1 \\ 2 \\ 7\end{bmatrix}$}r
	\end{selectAll}
\end{question}

\begin{question}
	\[
	\begin{bmatrix} 1 \\ 1\end{bmatrix} \cdot \begin{bmatrix} -1 \\ 1\end{bmatrix} = \answer{0}
	\]
		\begin{hint}
			By drawing these vectors, you can see that they are perpendicular, so according to property $5$, we must have that $\begin{bmatrix} 1 \\ 1\end{bmatrix} \cdot \begin{bmatrix} -1 \\ 1\end{bmatrix} = 0$
		\end{hint}
\end{question}

\begin{question}
\[
\begin{bmatrix} 2 \\ 3 \\ 1\end{bmatrix} \cdot \begin{bmatrix} 2 \\ 3 \\1\end{bmatrix}= \answer{14} 
\]	

\begin{hint}
	By property $4$, we have that the dot product of any vector with itself is the square of its length.  The length of $\begin{bmatrix} 2 \\ 3 \\ 1\end{bmatrix}$ is $\sqrt{2^2+3^2+1^2}$, so the answer must be $2^2+3^2+1^2  =14$
\end{hint}
\end{question}

\begin{question}
\[
\begin{bmatrix} 2 \\ 3 \end{bmatrix} \cdot \begin{bmatrix}4 \\ -5\end{bmatrix} = \answer{-7} 
\]

This is a hard problem, and will require significant creativity to answer.

\begin{hint}
	The only kinds of vectors we actually know how to compute the cross products of are vectors which are perpendicular or vectors which are parallel.  These vectors are neither.  However, we can break these vectors down into components which are all mutually either perpendicular or parallel!  In particular, the unit vectors $\mathbf{i}$ and $\mathbf{j}$ are perpendicular. 
\end{hint}

\begin{hint}
	We can rewrite the product as
	
	\[
	(2\mathbf{i}+3\mathbf{j}) \cdot (5\mathbf{i} +-5\mathbf{j})
	\]
\end{hint}

\begin{hint}
	\begin{align*}
		(2\mathbf{i}+3\mathbf{j}) \cdot (4\mathbf{i} +-5\mathbf{j}) &= 4(2\mathbf{i}+3\mathbf{j}) \cdot \mathbf{i}+ -5(2\mathbf{i}+3\mathbf{j}) \mathbf{j}\\
			&=8\mathbf{i} \cdot \mathbf{i} +12 \mathbf{i} \cdot \mathbf{j}-10\mathbf{i} \cdot \mathbf{j}-15\mathbf{j} \cdot \mathbf{j}\\
			&=8(1)+12(0)-10(0)-15(1)\\
			&=-7
	\end{align*}
\end{hint}
\end{question}

The last question should convince you of the following theorem

\begin{theorem}
	The dot product is given by the following formula:
	\[
	\begin{bmatrix}x_1\\x_2\\\vdots\\x_n\end{bmatrix} \cdot  \begin{bmatrix}y_1\\y_2\\\vdots\\y_n\end{bmatrix} =x_1y_1+x_2y_2+x_3y_3+\dots + x_ny_n
	\]
\end{theorem}

\begin{proof}
	Let 
	\[ 
	\mathbf{x} = \begin{bmatrix}x_1\\x_2\\\vdots\\x_n\end{bmatrix} = \sum_1^n x_i \mathbf{e_i}
	\]
	
	 and
	 
	\[ 
	\mathbf{y} = \begin{bmatrix}y_1\\y_2\\\vdots\\y_n\end{bmatrix} = \sum_1^n y_j \mathbf{e_j}
	\]	 
	 Then
	 
	\begin{align*}
		\mathbf{x} \cdot \mathbf{y} &= (\sum_1^n x_i \mathbf{e_i}) \cdot (\sum_1^n y_j \mathbf{e_j})\\
			&=\sum_{i,j =1}^n x_iy_j \mathbf{e_i} \cdot \mathbf{e_j} \textrm{ by the linearity properties of the dot product}\\
			&=\sum_{1}^n x_iy_1 \textrm{ since $\mathbf{e_i} \cdot \mathbf{e_j} = 1$ if $i=j$ and $0$ otherwise}\\
			&=x_1y_1+x_2y_2+x_3y_3+\dots + x_ny_n
	\end{align*}
\end{proof}

\begin{question}
	\[
	\begin{bmatrix} 5 \\ 4 \\ 1 \end{bmatrix} \cdot \begin{bmatrix}-2 \\ 0 \\  13\end{bmatrix} = \answer{3} 
	\]
	\begin{hint}
		By the theorem above, 
		
		\[
		\begin{bmatrix} 5 \\ 4 \\ 1 \end{bmatrix} \cdot \begin{bmatrix}-2 \\ 0 \\  13\end{bmatrix} = 5(-2)+4(0)+1(13) = 3
		\]
	\end{hint}
\end{question}

\section{The geometry of the Dot Product}

Recall the law of cosines:

\begin{theorem}[Law of Cosines]
	Given a triangle with sides $a$, $b$, and $c$, and with $\theta$ being the angle between $a$ and $b$, we have
	
	\[
	c^2 = a^2+b^2-2ab\cos(\theta)
	\]
\end{theorem}

We can rephrase this theorem in the language of vectors.  The vectors $\textbf{u}$, $\textbf{v}$ and $\textbf{v} - \textbf{u}$ form a triangle, so if $\theta$ is the angle between $\textbf{u}$ and $\textbf{v}$ we must have

\[
|\textbf{v} - \textbf{u}|^2=|\textbf{v}|^2+|\textbf{u}|^2-2|\textbf{u}||\textbf{v}|\cos(\theta)
\]



\begin{theorem}[Geometric Interpretation of the Dot Product]
	For any two vectors $\textbf{u}$ and $\textbf{v}$,  $\textbf{u} \cdot \textbf{v} = |\textbf{u}||\textbf{v}|\cos(\theta)$, where $\theta$ is the angle between $\textbf{u}$ and $\textbf{v}$.
\end{theorem}

\begin{proof}
	You should really try to do this for yourself!
	
	The biggest idea is to start with the vector version of the law of cosines above, rewrite $|\textbf{v} - \textbf{u}|^2$ as $(\textbf{v} - \textbf{u})\cdot(\textbf{v} - \textbf{u})$, and then just do algebra and see what cancels.
	
	See the hint for the full proof.
	
	\begin{hint}
		\begin{align*}
		|\textbf{v} - \textbf{u}|^2&=|\textbf{v}|^2+|\textbf{u}|^2-2|\textbf{u}||\textbf{v}|\cos(\theta)\\
		(\textbf{v} - \textbf{u})\cdot(\textbf{v} - \textbf{u}) &=|\textbf{v}|^2+|\textbf{u}|^2-2|\textbf{u}||\textbf{v}|\cos(\theta)\\
		\textbf{v}\cdot\textbf{v} -2\textbf{v}\cdot\textbf{u}+\textbf{u}\cdot\textbf{u}&=|\textbf{v}|^2+|\textbf{u}|^2-2|\textbf{u}||\textbf{v}|\cos(\theta)\\
		|\textbf{v}|^2+|\textbf{u}|^2 -2\textbf{v}\cdot\textbf{u} &=|\textbf{v}|^2+|\textbf{u}|^2-2|\textbf{u}||\textbf{v}|\cos(\theta)\\
		\textbf{u} \cdot \textbf{v} &= |\textbf{u}||\textbf{v}|\cos(\theta)
		\end{align*}
	
	\end{hint}
\end{proof}

Note that this agrees with what we know about the dot product:  if $\textbf{u}$ and $\textbf{v}$ are perpendicular, then $\cos(\theta) = 0$, so $\textbf{u} \\cdot \textbf{v} = 0$.  Also $\textbf{u} \cdot \textbf{u} = |\textbf{u}|^2$ since $\cos(\theta) = 1$ in this case.

\begin{question}
	Find the angle between the vector $\textbf{u} = 2\textbf{i}+3\textbf{j}+6\textbf{k}$ and $\textbf{v} = 1\textbf{i}+2\textbf{j}+2\textbf{k}$.
	
	\[
	\theta = \answer{ \arccos(\frac{20}{21})}
	\]
	
	Also, think about how hard this question would have been before you read this section!
	
	\begin{hint}
		\begin{align*}
		\textbf{u} \cdot \textbf{v} &= |\textbf{u}||\textbf{v}|\cos(\theta)\\
		2(1)+3(2)+6(2)&= \sqrt{2^2+3^2+6^2}\sqrt{1^2+2^2+2^2}\cos(\theta)\\
		20 &= \sqrt{49}\sqrt{9}\cos(theta)\\
		20 &=7(3)\cos(\theta)\\
		\cos(\theta) &= \frac{20}{21}\\
		\theta &= \arccos(\frac{20}{21})
		\end{align*}
	\end{hint}
	
\end{question}

One of the major uses of the dot product is to let us \textbf{project} one vector in the direction of another.

%BADBAD PICTURE



\end{document}
