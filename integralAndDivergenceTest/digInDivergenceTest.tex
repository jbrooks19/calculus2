\documentclass{ximera}

\input{../preamble.tex}

\outcome{Use the divergence test to determine that a series diverges.}
\outcome{Recognize known convergent or divergent series.}

\title[Dig-In:]{The divergence test}

\begin{document}
\begin{abstract}
If an infinite sum converges, then its terms must tend to zero.
\end{abstract}
\maketitle


As one contemplates the behavior of series, a few facts become clear:
\begin{enumerate}
\item In order to add an infinite list of nonzero numbers and get a
  finite result, ``most'' of those numbers must be ``very near'' $0$.
\item If a series diverges, it means that the sum of an infinite list
  of numbers is not finite (it may approach $\pm \infty$ or it may
  oscillate), and:
  \begin{enumerate}
  \item The series will still diverge if the first term is removed.
  \item The series will still diverge if the first $10$ terms are
    removed.
  \item The series will still diverge if the first $1,000,000$ terms
    are removed.
  \item The series will still diverge if any finite number of terms
    from anywhere in the series are removed.
  \end{enumerate}
\end{enumerate}

These concepts are very important and lie at the heart of the next two
theorems.

\begin{theorem}[Divergence test]
  Consider the series
  \[
  \sum_{n=1}^\infty a_n.
  \]
\begin{enumerate}
\item If $\sum_{n=1}^\infty a_n$ converges, then $\lim_{n\to\infty}a_n
  =0$.
\item If $\lim_{n\to\infty}a_n \neq 0$, then $\sum_{n=1}^\infty a_n$
  diverges.
\end{enumerate}


Note that the two statements above are really the same. In order to
converge, the limit of the terms of the sequence must approach $0$; if
they do not, the series will not converge.


\begin{warning}
  This theorem \emph{does not state} that if $\lim_{n\to\infty} a_n =
  0$ then $\sum_{n=1}^\infty a_n$ converges.
  
\end{warning}

The standard example of
  this is the Harmonic Series, as given in Key Idea
  \ref{idea:famous_series}. The Harmonic Sequence, $\{1/n\}$,
  converges to 0; the Harmonic Series, $\sum_{n=1}^\infty 1/n$,
  diverges.


\theorem{thm:series_behavior}{Infinite Nature of Series}
{The convergence or divergence remains unchanged by the addition or subtraction of any finite number of terms. That is:
	\begin{enumerate}
	\item		A divergent series will remain divergent with the addition or subtraction of any finite number of terms.
	\item		A convergent series will remain convergent with the addition or subtraction of any finite number of terms. (Of course, the \emph{sum} will likely change.)
	\end{enumerate}
}



\begin{question}
If we have a sequence $a_k$, $k=0,1,2, \dots $ we know how to generate
the sequence of partial sum $S_n = \sum_0^n a_k$.

What about going backwards?

For instance, if I know that $S_n = 3-\frac{4}{n+1}$ is the sequence of partial sums for some series $a_k$, can you find a formula for $a_k$?

\[
a_0 = \answer{-1}
\]

\[
a_k = \answer{\frac{4}{k+1} - \frac{4}{k}} \text{when $k>0$}
\]

\[
\sum_0^\infty a_k = \answer{3}
\]


\begin{hint}
$S_0 = a_0$, so $a_0 = 3-\frac{4}{0+1} = -1$.
\end{hint}

\begin{hint}
$S_k = a_0+a_1+ \dots +a_{k-1}+a_k = S_{k-1}+a_k$.  So $a_k = S_k - S_{k-1}$
\end{hint}

\begin{hint}
Thus $a_k = (3-\frac{4}{k+1})  - (3-\frac{4}{k}) = \frac{4}{k+1} - \frac{4}{k}$
\end{hint}


\begin{hint}
 $\sum_0^\infty a_k $ is defined as the limit of its sequence of partial sums.  Thus it is equal to $\lim_{n \to \infty} 3-\frac{4}{n+1} = 3$
\end{hint}

\end{question}

\begin{explanation}
In the last question, we used the key idea that if $a_k$ is a sequence, and $S_n = \sum_0^n a_k $ is its sequence of partial sums, then $a_k = S_k-S_{k-1}$.

Suppose that $\sum_0^\infty a_k$ converges to $L$.  Then we can use this equation to learn something about the sequence $a_k$.

\begin{align*}
	a_k &= S_k-S_{k-1}\\
	\lim_{k \to \infty} a_k &= \lim_{k \to \infty} S_k -  \lim_{k \to \infty} S_{k-1}\\
	\lim_{k \to \infty} a_k &= \answer{L} - \answer{L}\\
	\lim_{k \to \infty} a_k &= \answer{0}
\end{align*}

\begin{hint}
	$\lim_{k \to \infty} S_{k}$ is the definition of $\sum_0^\infty a_k$, so it is equal to $L$.  $S_{k-1}$ is the same sequence, just shifted by one slot, so it has the same limit of $L$.  Thus their difference is $0$.  So $\lim_{k \to \infty} a_k = 0$.
\end{hint}
\end{explanation}

We have just proven the following theorem:

\begin{theorem}[Divergence Test]
	If $\sum_0^\infty a_k$ is a convergent series, then $\lim_{k \to \infty} a_k = 0$.  In words ``The sequence of terms of any convergent series must tend to zero''.
\end{theorem}

\begin{question}
Which of the following statements are true?  Mark all that apply.

\begin{multipleChoice}
	\choice[correct]{If $\sum_0^\infty a_k$ is convergent, then $\lim_{k \to \infty} a_k = 0$ }
	\choice{If $a_k \to 0$ as $k \to \infty$, then $\sum_0^\infty a_k$ is convergent}
	\choice{If $\sum_0^\infty a_k$ is divergent, then $\lim_{k \to \infty} a_k \neq 0$ }
	\choice[correct]{If  $\lim_{k \to \infty} a_k \neq 0$, then $\sum_0^\infty a_k$ is divergent}
\end{multipleChoice}


\begin{warning}
This question is important!  Mistakes in this logic are among the most common mistakes made by calculus students answering questions about sequences and series.
\end{warning}

\end{question}

\begin{question}
	We say that a series ``passes the divergence test'' if its sequence of terms tends to zero.  Which of the following series pass the divergence test?

\begin{multipleChoice}
	\choice[correct]{$\sum_3^\infty \frac{1}{\ln{ n }}$}
	\choice{$\sum_0^\infty \sin(n)$}
	\choice[correct]{$\sum_0^\infty \frac{\sin(n)}{n^2}$}
	\choice[correct]{$\sum_5^\infty \frac{n+7}{n+6} - \frac{\sin(n)}{n}$}
	\choice{$\sum_0^\infty \frac{2n}{n - 5}$}
\end{multipleChoice}

\end{question}



\end{document}


