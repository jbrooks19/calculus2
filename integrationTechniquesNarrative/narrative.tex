\documentclass{ximera}
\input{../preamble}

\title[Dig-In:]{Motivation for Techniques of Integration}

\begin{document}
\begin{abstract}
	What's the big idea with all these techniques?
\end{abstract}
\maketitle

Finding "nice" expressions for the integrals of most functions is impossible.

For example, the \textbf{Error function} is defined by

\[
\mathrm{Erf}(x) = \frac{2}{\sqrt{\pi}} \int_0^x e^{-t^2} \d t
\]

This function is very important in statistics, but \href{cannot be expressed as an elemenary function}{BADBAD WIKIPEDIA} (you cannot write it down in terms of rational functions, trig functions, exponentials, etc without an integral).

In fact, in some sense, it is incredibly rare for an elementary function to have an elementary antiderivative.

To a mathematician, the first thing they will ask when confronted with this fact is "why can't I find such elementary antiderivatives?".  The answer to this question is beyond the scope of this course, and involves something called ``\href{differential galois theory}{BADBAD WIKIPEDIA}''.

The second question a mathematician might ask is ``Is there a criteria which allows me to decide when a function has an elementary antiderivative?''.  This question is also too difficult, but is answered by ``\href{Louville's Criteria}{BADBAD WIKI}''

The third question would be "Well, if you cannot tell me exactly which functions I can integrate, can we at least figure out some large classes of functions which \textbf{do} have elementary antiderivatives?"

The answer to this last question is yes, and we will be able to treat some large classes of functions completely.

For instance, we already know that all polynomial functions have elementary antiderivatives.

Perhaps the simplest class of functions next to polynomials is the class of rational functions.  It turns out that we can *always* find an elementary antiderivative for any rational function, and there is a somewhat involved systematic approach to finding it.

Let's take a look at a particular rational function.

\end{document}
