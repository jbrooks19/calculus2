\documentclass{ximera}

\input{../preamble.tex}

\outcome{Give the definition of a Maclaurin series.}
\outcome{Give the definition of a Taylor series.}
\outcome{Find the Maclaurin/Taylor series of a function.}
\outcome{Use given Maclaurin/Taylor series to find new power series.}
\outcome{Use the binomial series to find Maclaurin series.}
\outcome{Find the interval and radius of convergence of a Maclaurin/Taylor series.}

\title[Dig-In:]{Introduction to Taylor series}

\begin{document}
\begin{abstract}
  We study Taylor and Maclaurin series.
\end{abstract}
\maketitle

We've seen that we can approximate functions with polynomials, given
that enough derivative information is available.  We have also seen
that certain functions can be represented by a power series.  In this
section we combine these concepts: If a function $f(x)$ is infinitely
differentiable, we show how to represent it with a power series
function.

\begin{definition}
  Let $f(x)$ have derivatives of all orders at $x=c$.  
  \begin{itemize}
  \item The \dfn{Taylor series} of $f(x)$, centered at $c$ is
    \[
    \sum_{n=0}^\infty \frac{f^{(n)}(c)}{n!}(x-c)^n.
    \]
  \item Setting $c=0$ gives the \dfn{Maclaurin series} of $f(x)$:
    \[
    \sum_{n=0}^\infty \frac{f^{(n)}(0)}{n!}x^n.
    \]
  \end{itemize}
\end{definition}

\begin{question}
  Quick: Write down the Taylor series for $f(x) = x^3-6x^2+1$ centered
  at $x=0$.


  \begin{question}
    Write down the Taylor series for $f(x) = x^3-6x^2+1$ centered at
    $x=1$.
  \end{question}
\end{question}

The difference between a Taylor polynomial and a Taylor series is the
former is a polynomial, containing only a finite number of terms,
whereas the latter is a series, a summation of an infinite set of
terms. When creating the Taylor polynomial of degree $n$ for a
function $f(x)$ at $x=c$, we needed to evaluate $f$, and the first $n$
derivatives of $f$, at $x=c$. When creating the Taylor series of $f$,
it helps to find a pattern that describes the $n$th derivative of $f$
at $x=c$.


We demonstrate this in the next two examples.

\end{document}
