\documentclass{ximera}

\input{../preamble.tex}

\outcome{Write equations representing lines in space.}
\outcome{Answer questions about lines and curves in space.}

\title[Dig-In:]{Lines and Curves in Space}

\begin{document}
\begin{abstract}
  Vector valued functions are parametrized curves.
\end{abstract}
\maketitle

\section{Vector valued functions}

A function $\vec{f}: \R \to \R^3$ can be thought of as associating to
each time $t$ a vector $\vector{x(t),y(t),z(t)}$.  Placing the tail of
the vector at the origin, its head will sweep out a curve
parameterized by $t$:
\begin{image}
    \begin{tikzpicture}
      \begin{axis}%
        [tick label style={font=\scriptsize},axis on top,
	  axis lines=center,
	  view={155}{10},no markers,
	  ymin=-1.1,ymax=1.1,
	  xmin=-7,xmax=7,
	  zmin=-1.1, zmax=1.1,
	  every axis x label/.style={at={(axis cs:\pgfkeysvalueof{/pgfplots/xmax},0,0)},xshift=-3pt,yshift=-3pt},
	  xlabel={\scriptsize $x$},
	  every axis y label/.style={at={(axis cs:0,\pgfkeysvalueof{/pgfplots/ymax},0)},xshift=5pt,yshift=-2pt},
	  ylabel={\scriptsize $y$},
	  every axis z label/.style={at={(axis cs:0,0,\pgfkeysvalueof{/pgfplots/zmax})},xshift=0pt,yshift=4pt},
	  zlabel={\scriptsize $z$}
	]
        \draw[thick,->,penColor!50!white] (axis cs: 0,0,0)--(axis cs: {-4},{sin(deg(-4))},{cos(deg(-4))});
        \draw[thick,->,penColor!50!white] (axis cs: 0,0,0)--(axis cs: {-3},{sin(deg(-3))},{cos(deg(-3))});
        \draw[thick,->,penColor!50!white] (axis cs: 0,0,0)--(axis cs: {-2},{sin(deg(-2))},{cos(deg(-2))});
        \draw[thick,->,penColor!50!white] (axis cs: 0,0,0)--(axis cs: {-1},{sin(deg(-1))},{cos(deg(-1))});
        \draw[thick,->,penColor!50!white] (axis cs: 0,0,0)--(axis cs: {0},{sin(deg(0))},{cos(deg(0))});
        \draw[thick,->,penColor!50!white] (axis cs: 0,0,0)--(axis cs: {1},{sin(deg(1))},{cos(deg(1))});
        \draw[thick,->,penColor!50!white] (axis cs: 0,0,0)--(axis cs: {2},{sin(deg(2))},{cos(deg(2))});
        \draw[thick,->,penColor!50!white] (axis cs: 0,0,0)--(axis cs: {3},{sin(deg(3))},{cos(deg(3))});
        \draw[thick,->,penColor!50!white] (axis cs: 0,0,0)--(axis cs: {4},{sin(deg(4))},{cos(deg(4))});
        \addplot3+[very thick, penColor, smooth,samples=200,samples y=0,domain=-12:12] ({x},{sin(deg(x))},{cos(deg(x)});
      \end{axis}
    \end{tikzpicture}
    \end{image}
\begin{example}
  Consider the function $\vec{f}(t) = \vector{\cos(t),\sin(t),t}$.  The
  projection of the point $\vec{f}(t)$ into the $(x,y)$-plane moves around the
  unit circle in the positive direction.  The projection onto the $z$
  axis moves at a constant rate in the positive direction.  So we
  expect that $\vec{f}$ parameterizes
  \begin{multipleChoice}
    \choice{a straight line}
    \choice{a circle}
    \choice[correct]{a spiral}
  \end{multipleChoice}
  \begin{feedback}
    Here is the graph of $\vec{f}$:
    \begin{image}
    \begin{tikzpicture}
      \begin{axis}%
        [tick label style={font=\scriptsize},axis on top,
	  axis lines=center,
	  view={155}{10},no markers,
	  ymin=-1.1,ymax=1.1,
	  xmin=-1.1,xmax=1.1,
	  zmin=-10, zmax=10,
	  every axis x label/.style={at={(axis cs:\pgfkeysvalueof{/pgfplots/xmax},0,0)},xshift=-3pt,yshift=-3pt},
	  xlabel={\scriptsize $x$},
	  every axis y label/.style={at={(axis cs:0,\pgfkeysvalueof{/pgfplots/ymax},0)},xshift=5pt,yshift=-2pt},
	  ylabel={\scriptsize $y$},
	  every axis z label/.style={at={(axis cs:0,0,\pgfkeysvalueof{/pgfplots/zmax})},xshift=0pt,yshift=4pt},
	  zlabel={\scriptsize $z$}
	]
        \addplot3+[very thick, penColor, smooth,samples=200,samples y=0,domain=-12:12] ({cos(deg(x))},{sin(deg(x))},{x});
      \end{axis}
    \end{tikzpicture}
    \end{image}
  \end{feedback}
\end{example}

This is an example of a \textit{vector valued function}.
\begin{definition}
  A \dfn{vector valued function} maps real numbers to vectors in $\R^n$.
\end{definition}
Vector valued functions simply map numbers to lists of numbers, that we interpret as vectors:
\begin{image}
  \begin{tikzpicture}
    \node at (0,0) {$\vec{f}(t) = \underbrace{\vector{x(t),y(t),z(t)}}_{\text{vector}}$};
    \draw[ultra thick,->,gray] (.5,1) -- (0,.5);
    \draw[ultra thick,->,gray] (.5,1) -- (.5,.5);
    \draw[ultra thick,->,gray] (.5,1) -- (1,.5);
    \node at (.5,1.2) {\scriptsize numbers};
\end{tikzpicture}
\end{image}



\subsection{How are vector valued functions useful?}

Here is a list of examples of what a
function $f: \R \to \R^3$ could represent, to get your imagination
going:

\begin{itemize}
\item The $3$ dimensional position of a rocket in space as a function of time. 
\item The average size of $3$ different species of bacteria as a function of the amount of chlorine in the water.
\item The performance of $3$ different stocks a function of time.
\item The trunk width, height, and canopy radius of a tree as a function of time.
\item The average temperature, humidity, and air pressure at a given latitude as a function of that latitude.
\end{itemize}

Basically any time you can see that three different quantities all
depend on one other quantity, these sorts of vector valued functions
are going to be useful models.  Of course, if you have more relevant
inputs, or outputs, you may need a function $\R^m \to \R^n$, but we
will restrict our attention to $\R \to \R^3$ for the time being.



\section{Lines in space}

It is easy to create a vector valued function that produces a line
that passes through two points $\vec{p}$ and $\vec{q}$:
\[
\boldsymbol{\l}(t) = \vec{p} + t(\vec{q}-\vec{p}).
\]
\begin{question}
  What is the value of $\boldsymbol{\l}(0)$?
  \begin{multipleChoice}
    \choice{$\boldsymbol{\l}(0)$ is unknowable}
    \choice[correct]{$\boldsymbol{\l}(0)=\vec{p}$}
    \choice{$\boldsymbol{\l}(0)=\vec{q}$}
    \choice{$\boldsymbol{\l}(0)=\vec{q}-\vec{p}$}
  \end{multipleChoice}
  \begin{question}
    What is the value of $\boldsymbol{\l}(1)$?
    \begin{multipleChoice}
      \choice{$\boldsymbol{\l}(1)$ is unknowable}
      \choice{$\boldsymbol{\l}(1)=\vec{p}$}
      \choice[correct]{$\boldsymbol{\l}(1)=\vec{q}$}
      \choice{$\boldsymbol{\l}(1)=\vec{q}-\vec{p}$}
    \end{multipleChoice}
  \end{question}
\end{question}


\begin{question}
   Compare and contrast the curves $\vec{f}(t) =
   \vector{-3+t,5+2t,1+3t}$ and $\vec{g}(t)=\vector{-3+2t,5+4t,1+6t}$.
   \begin{multipleChoice}
     \choice{They parameterize different lines}
     \choice{They parameterize the same line, but $\vec{f}(t)$ moves ``twice as fast'' as $\vec{g}(t)$ }
     \choice[correct]{They parameterize the same line, but $\vec{g}(t)$ moves ``twice as fast'' as $\vec{f}(t)$ }
     \choice{These are the same function!}
   \end{multipleChoice}
   \begin{hint}
     We can rewrite $\vec{f}$ and $\vec{g}$ as $\vec{f}(t) =
     \vector{-3,5,1}+t\vector{1,2,3}$ and
     $\vec{g}(t)=\vector{-3,5,1}+t\vector{2,4,6}$.
   \end{hint}
   \begin{question}
   Compare and contrast the curves $\vec{f}(t) =
   \vector{-3+t,5+2t,1+3t}$ and $\vec{g}(t)=\vector{-3-t,5-2t,1-3t}$.
   \begin{multipleChoice}
     \choice{They parameterize different lines}
     \choice[correct]{They parameterize the same line, but $\vec{f}(t)$ moves in the opposite direction compared with $\vec{g}(t)$}
     \choice{They parameterize the same line, but $\vec{g}(t)$ moves ``twice as fast'' as $\vec{f}(t)$ }
     \choice{These are the same function!}
   \end{multipleChoice}
 \end{question}
\end{question}

\begin{question}
  If $\boldsymbol{\l}(t) = \vector{x(t),y(t),z(t)}$ is a line and
  passes through the points $\boldsymbol{\l}(0) = \vector{1,2,3}$ and
  $\boldsymbol{\l}(1) = \vector{2,2,2}$, then
  \begin{align*}
    x(t) &= \answer{1+t}\\
    y(t) &= \answer{2}\\
    z(t) &= \answer{3-t}
  \end{align*}
  \begin{hint}
    Let $\boldsymbol{\l}(t) = \vec{p}+t\vec{v}$.  Then $f(0) = \vec{p}$.
  \end{hint}
\end{question}

We can use these ideas to parameterize any line in space, but as we
have already seen, sometimes you can find more than one expression
that parameterizes a given line: Some may ``move faster'' than
others, or in the opposite direction, or even at uneven rates!

Often, for a given line, we will already know that the line passes
through the tip of $\vec{p}$, and points in a direction $\vec{v}$.
Then we write
\[
\boldsymbol{\l}(t) = \vec{p}+t\vec{v}.
\]
If we know that a line passes through two points (that we'll notate
with vectors) $\vec{p}$ and $\vec{q}$, then we know that it points in
the direction $\vec{v} = \vec{q} - \vec{p}$, and passes through the
tip of $\vec{p}$.  So we can parameterize it as
\[
\boldsymbol{\l}(t) = \vec{p}+t(\vec{q} - \vec{p}).
\]

\begin{question}
  Using the ideas above, find an expression parameterizing the line
  passing through $\vec{p} = \vector{0,2,4}$ and $\vec{q} =
  \vector{1,1,1}$.
  \[
  \boldsymbol{\l}(t) = \vector{\answer{t},\answer{2-t},\answer{4-3t}}
  \]
  \begin{hint}
    The line passes through $\vec{p}$ and points in the direction
    \[
    \vec{q} - \vec{p} = \vector{1,1,1} - \vector{0,2,4}
    \]
  \end{hint}
\end{question}

\subsection{Lines embedded in surfaces}


Sometimes lines lie on surprising surfaces:

\begin{question}
  Consider the surface determined by all $x$, $y$ and $z$ such that:
  \[
  x^2+y^2=z^2+1
  \]
  This surface looks something like:
  \begin{image}
    \begin{tikzpicture}
      \begin{axis}[
          xmin=-4, xmax=4,
          clip=false,
          width=4in,
          height=2in,
          axis lines =none,
        ]
        \draw[penColor,very thick,fill=fillp] (axis cs:0,2.8) ellipse (290 and 50);
        \draw[penColor,very thick,] (axis cs: -1,0) arc (180:360:100 and 30);
        \draw[penColor,very thick,] (axis cs: -2.9,-2.7) arc (180:360:290 and 50);
       
        \addplot [penColor,very thick,domain=-2.9:-1,smooth,samples=100] {sqrt(x^2-1)};%(sqrt(1+u^2) cos(v), sqrt(1+u^2) sin(v), u)
        \addplot [penColor,very thick,domain=1:2.9,smooth,samples=100] {sqrt(x^2-1)};    % v in 0 2pi % u is whatevs
        \addplot [penColor,very thick,domain=1:2.9,smooth,samples=100] {-sqrt(x^2-1)};
        \addplot [penColor,very thick,domain=-2.9:-1,smooth,samples=100] {-sqrt(x^2-1)};

          \draw[->] (axis cs: -3,0)--(axis cs: 3,0);
          \draw[->] (axis cs: 0,-3.5)--(axis cs: 0,3.5);
          \draw[->] (axis cs: 1.2,2.1)--(axis cs: -1.2,-2.1);
          \node[right] at (axis cs: 3,0) {$x$};
          \node[above] at (axis cs: 0,3.5) {$y$};
          \node[below] at (axis cs: -1.2,-2.1) {$z$};
      \end{axis}
    \end{tikzpicture}
  \end{image}
  Which of the following lines lie on the surface $x^2+y^2=z^2+1$?
  \begin{selectAll}% l(t) = (cos v,sin v, 0) + t(sin v,-cos v,1)% m(t) = (cos v,sin v, 0) + t(-sin v,cos v,1)
    \choice{$\vector{1,0,0} + t\vector{-5,0,5}$}
    \choice[correct]{$\vector{1,0,0} + t\vector{0,-3,3}$}
    \choice[correct]{$\vector{0,1,0} + t\vector{-7,0,-7}$}
    \choice{$\vector{0,1,0} + t\vector{-4,4,0}$}
    \choice[correct]{$\vector{1/2,\sqrt{3}/2,0} + t\vector{-\sqrt{3},1,2}$}
    \choice{$\vector{1/2,\sqrt{3}/2,0} + t\vector{-1,\sqrt{3},2}$}
  \end{selectAll}
  \begin{hint}
    Separate each line into its component functions: $x$, $y$, and
    $z$, and see if the equation defining the surface is valid for all
    $t$.
  \end{hint}
\end{question}



\section{Curves}

Though their formulation may be more complex, a vector valued function
that produces a curve is no different from that which produces a line
(a line is a special type of curve!). Though there are some cases we should discuss.

\subsection{Circles and ellipses}

Given two orthogonal unit vectors, $\vec{u}$ and $\vec{v}$, and any
other vector $\vec{p}$, the vector valued function
\[
\vec{f}(t) = \vec{p}+r\cdot \cos(t)\cdot \vec{u} + r \cdot \sin(t)+\vec{v}
\]
gives a circle of radius $r$, centered at the tip of $\vec{p}$, lying
in the plane containing $\vec{u}$ and $\vec{v}$. Moreover, to produce
an ellipse, we write:
\[
\vec{g}(t) = \vec{p}+a\cdot \cos(t)\cdot \vec{u} + b \cdot \sin(t)+\vec{v}
\]

\begin{question}
  Consider the surface determined by all $x$, $y$ and $z$ such that:
  \[
  z^2=x^2+y^2
  \]
   Which of the following lines lie on the surface $x^2+y^2=z^2+1$?
  \begin{selectAll}% l(t) = (cos v,sin v, 0) + t(sin v,-cos v,1)% m(t) = (cos v,sin v, 0) + t(-sin v,cos v,1)
    \choice{$\vector{t\cos(t),\sin(t),t}$}
    \choice{$\vector{6\cos(t),6\sin(t),1}$}
    \choice[correct]{$\vector{t\cos(t),t\sin(t),t}$}
    \choice{$\vector{\cos(t),-2\sin(t),-2}$}
    \choice[correct]{$\vector{6\cos(t),6\sin(t),6}$}
    \choice[correct]{$\vector{-2\cos(t),-2\sin(t),-2}$}
  \end{selectAll}
  \begin{hint}
    Separate each line into its component functions: $x$, $y$, and
    $z$, and see if the equation defining the surface is valid for all
    $t$.
  \end{hint}
\end{question}


\end{document}
