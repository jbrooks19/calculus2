\documentclass{ximera}

\input{../preamble.tex}

\outcome{Understand what a slope field is.}
\outcome{Understand what a slope field tells us about a differential equation.}
\outcome{Define an autonomous differential equation.}
\outcome{Understand and use Euler's method.}
\outcome{Sketch slope fields.}
\outcome{Find equilibrium solutions.}

\title[Dig-In:]{Slope fields and Euler's method}

\begin{document}
\begin{abstract}
We describe numerical and graphical methods for understanding
differential equations.
\end{abstract}
\maketitle


\section{Slope fields}

We cannot (yet!) solve the differential equation
\[
y' = x+y
\]
However, from the equation alone, we can describe some facts about the
solution.

\begin{question}
  Consider the solution $y$ to the differential equation $y'=x+y$
  which passes through the point $(1,2)$.  Is $y$ increasing or
  decreasing at $x=1$?
  \begin{prompt}
  \begin{multipleChoice}
    \choice[correct]{increasing}
    \choice{decreasing}
  \end{multipleChoice}
  \end{prompt}
  \begin{hint}
    By definition, a solution to this differential equation which
    passes through $(1,2)$ must have $y(1)=2$, and
    \begin{align*}
      y'(1) &= 1+y(1)\\
      &=1+2\\
      &=3.
    \end{align*}
    This is positive, so the function is increasing at $x=1$.
  \end{hint}
\end{question}
In fact, we can say more. The differential equation
\[
y' = x+y
\]
tells us the \textbf{slope} of any solution passing through a given
point.
\begin{question}
  Consider the solution $y$ to the differential equation $y'=x+y$
  which passes through the point $(1,-2)$. What is the slope of the
  solution $y$ at $x=1?$
    \begin{hint}
    By definition, a solution to this differential equation which
    passes through $(1,-2)$ must have $y(1)=-2$, and
    \begin{align*}
      y'(1) &= 1+y(-1)\\
      &= 1-2=-1.
    \end{align*}
    Hence the slope is $-1$ at $x=1$.
    \end{hint}
    \begin{prompt}
      The slope of $y$ at $x=1$ is $\answer{-1}$.
    \end{prompt}
\end{question}

Given a differential equation, say $y'=x+y$, we can pick points in the
plane and compute what the slope of a solution at those points will
be. Repeating this process, we can generate a \textit{slope field}.
The slope field for the differential equation $y' = x+y$ looks like
this:
\begin{image}
{\def\length{sqrt(1+(x+y)^2)}
\begin{tikzpicture}
  \begin{axis}[
      xmin=-3, xmax=3,ymin=-3,ymax=3,domain=-3:3,view={0}{90},
      axis lines =center, xlabel=$x$, ylabel=$y$,
      every axis y label/.style={at=(current axis.above origin),anchor=south},
      every axis x label/.style={at=(current axis.right of origin),anchor=west},
      axis on top,
    ] 
    \addplot3 [penColor, quiver={u={1/\length}, v={(x+y)/(\length)},scale arrows=.2},samples=20] {0};
]  \end{axis}
\end{tikzpicture}}
\end{image}

Let's be explicit:
\begin{definition}
  A \dfn{slope field}, also called a \dfn{direction field}, is a
  graphical aid for understanding a differential equation, formed by:
  \begin{itemize}
	\item Choosing a grid of points.
	\item At each point, computing the slope given by the
          differential equation, using the $x$ and $y$-values of the
          point.
	\item At each point, drawing a short line segment with that
          slope.
\end{itemize}
\end{definition}

Here is the slope field for the differential equation $y'=x+y$, with a
few solutions of the differential equation also graphed.

\begin{image}
{\def\length{sqrt(1+(x+y)^2)}
\begin{tikzpicture}
  \begin{axis}[
      xmin=-3, xmax=3,ymin=-3,ymax=3,domain=-3:3,view={0}{90},
      axis lines =center, xlabel=$x$, ylabel=$y$,
      every axis y label/.style={at=(current axis.above origin),anchor=south},
      every axis x label/.style={at=(current axis.right of origin),anchor=west},
      axis on top,
    ] 
    \addplot3 [penColor, quiver={u={1/\length}, v={(x+y)/(\length)},scale arrows=.2},samples=20] {0};
    \addplot[penColor,very thick]{e^x-x-1};
    \addplot[penColor,very thick]{.2*e^x-x-1};
    \addplot[penColor,very thick]{-e^x-x-1};
    \addplot[penColor,very thick]{-.2*e^x-x-1};
    \addplot[penColor,very thick]{-3*e^x-x-1};
    \addplot[penColor,very thick]{2*e^x-x-1};
    \addplot[penColor2,very thick]{-x-1};
]  \end{axis}
\end{tikzpicture}}
\end{image}
Notice that the slope field suggests one solution to this differential
equation, which is a straight line.

\begin{question}
  Of the solutions shown in the slope field above, what is the formula
  for the linear solution?
  \begin{prompt}
    \[
    y = \answer{-1-x}
    \]
  \end{prompt}
\end{question}



\section{Autonomous differential equations}


Consider the following differential equations
\[
y' = x^3,\qquad y'=(y+1)(y-2), \qquad y'= x+y.
\]

The first differential equation, $y' = x^3$, is rather easy to solve,
we simply integrate both sides. This type of differential equation is
called a \dfn{pure-time differential equation}. Pure-time differential
equations express the derivative of the solution explicitly as a
function of an independent variable. We can symbolically describe a
pure-time differential equation as
\[
\dd[y]{x} = F(x).
\]

On the other hand, the second differential equation, $y'=(y+1)(y-2)$
does not involve the independent variable, $x$, at all!  Such
differential equations are called \textit{autonomous} differential
equations.

\begin{definition}
  A differential equation that does not involve the independent
  variable is called an \dfn{autonomous differential equation}. We can
  symbolically write
  \[
  \dd[y]{x} = F(y)
  \]
  to express an autonomous differential equation.
\end{definition}

Finally the third differential equation, $y'= x+y$, expresses $y'$ as
a function of both $y$ and the independent variable $x$. Such a
differential equation is called a \dfn{nonautonomous differential
  equation}. We can symbolically describe  a nonautonomous differential equation as
\[
\dd[y]{x} = F(x,y).
\]
\begin{question}
  Which of the following are autonomous differential equations?
  \begin{selectAll}
    \choice[correct]{$y' = y^2+y$}
    \choice{$y'=x-y$}
    \choice{$y'=x$}
    \choice[correct]{$y'=y$}
    \choice{$y'=-x/y$}
  \end{selectAll}
\end{question}

Since autonomous differential equations \textbf{only} depend on the
function's value their behavior does not depend on the independent
variable,

\begin{question}
Which of the five slope fields shown are for autonomous differential
equations?
\begin{selectAll}
\choice[correct]{
{\def\length{sqrt(1+(y^2+y)^2)}
\begin{tikzpicture}[framed,scale=.5,baseline=9ex]
  \begin{axis}[
      xmin=-3, xmax=3,ymin=-3,ymax=3,domain=-3:3,view={0}{90},
      axis lines =center, xlabel=$x$, ylabel=$y$,
      every axis y label/.style={at=(current axis.above origin),anchor=south},
      every axis x label/.style={at=(current axis.right of origin),anchor=west},
      axis on top,
    ] 
    \addplot3 [penColor, quiver={u={1/\length}, v={(y^2+y)/(\length)},scale arrows=.2},samples=20] {0};
]  \end{axis}
\end{tikzpicture}}}

\choice{{\def\length{sqrt(1+(x-y)^2)}
\begin{tikzpicture}[framed,scale=.5,baseline=9ex]
  \begin{axis}[
      xmin=-3, xmax=3,ymin=-3,ymax=3,domain=-3:3,view={0}{90},
      axis lines =center, xlabel=$x$, ylabel=$y$,
      every axis y label/.style={at=(current axis.above origin),anchor=south},
      every axis x label/.style={at=(current axis.right of origin),anchor=west},
      axis on top,
    ] 
    \addplot3 [penColor, quiver={u={1/\length}, v={(x-y)/(\length)},scale arrows=.2},samples=20] {0};
]  \end{axis}
\end{tikzpicture}}}

\choice{{\def\length{sqrt(1+(x)^2)}
\begin{tikzpicture}[framed,scale=.5,baseline=9ex]
  \begin{axis}[
      xmin=-3, xmax=3,ymin=-3,ymax=3,domain=-3:3,view={0}{90},
      axis lines =center, xlabel=$x$, ylabel=$y$,
      every axis y label/.style={at=(current axis.above origin),anchor=south},
      every axis x label/.style={at=(current axis.right of origin),anchor=west},
      axis on top,
    ] 
    \addplot3 [penColor, quiver={u={1/\length}, v={(x)/(\length)},scale arrows=.2},samples=20] {0};
]  \end{axis}
\end{tikzpicture}}}

\choice[correct]{{\def\length{sqrt(1+(y)^2)}
\begin{tikzpicture}[framed,scale=.5,baseline=9ex]
  \begin{axis}[
      xmin=-3, xmax=3,ymin=-3,ymax=3,domain=-3:3,view={0}{90},
      axis lines =center, xlabel=$x$, ylabel=$y$,
      every axis y label/.style={at=(current axis.above origin),anchor=south},
      every axis x label/.style={at=(current axis.right of origin),anchor=west},
      axis on top,
    ] 
    \addplot3 [penColor, quiver={u={1/\length}, v={(y)/(\length)},scale arrows=.2},samples=20] {0};
]  \end{axis}
\end{tikzpicture}}}

\choice{{\def\length{sqrt(1+(-x/y)^2)}
\begin{tikzpicture}[framed,scale=.5,baseline=9ex]
  \begin{axis}[
      xmin=-3, xmax=3,ymin=-3,ymax=3,domain=-3:3,view={0}{90},
      axis lines =center, xlabel=$x$, ylabel=$y$,
      every axis y label/.style={at=(current axis.above origin),anchor=south},
      every axis x label/.style={at=(current axis.right of origin),anchor=west},
      axis on top,
    ] 
    \addplot3 [penColor, quiver={u={1/\length}, v={(-x/y)/(\length)},scale arrows=.2},samples=20] {0};
]  \end{axis}
\end{tikzpicture}}}
\end{selectAll}
\begin{hint}
  If a differential equation is autonomous, then all of the slopes
  will be the same on each horizontal line.
\end{hint}
\begin{question}
Which of the five slope fields shown are for pure-time
differential equations?
\begin{prompt}
\begin{selectAll}
\choice{
{\def\length{sqrt(1+(y^2+y)^2)}
\begin{tikzpicture}[framed,scale=.5,baseline=9ex]
  \begin{axis}[
      xmin=-3, xmax=3,ymin=-3,ymax=3,domain=-3:3,view={0}{90},
      axis lines =center, xlabel=$x$, ylabel=$y$,
      every axis y label/.style={at=(current axis.above origin),anchor=south},
      every axis x label/.style={at=(current axis.right of origin),anchor=west},
      axis on top,
    ] 
    \addplot3 [penColor, quiver={u={1/\length}, v={(y^2+y)/(\length)},scale arrows=.2},samples=20] {0};
]  \end{axis}
\end{tikzpicture}}}

\choice{{\def\length{sqrt(1+(x-y)^2)}
\begin{tikzpicture}[framed,scale=.5,baseline=9ex]
  \begin{axis}[
      xmin=-3, xmax=3,ymin=-3,ymax=3,domain=-3:3,view={0}{90},
      axis lines =center, xlabel=$x$, ylabel=$y$,
      every axis y label/.style={at=(current axis.above origin),anchor=south},
      every axis x label/.style={at=(current axis.right of origin),anchor=west},
      axis on top,
    ] 
    \addplot3 [penColor, quiver={u={1/\length}, v={(x-y)/(\length)},scale arrows=.2},samples=20] {0};
]  \end{axis}
\end{tikzpicture}}}

\choice[correct]{{\def\length{sqrt(1+(x)^2)}
\begin{tikzpicture}[framed,scale=.5,baseline=9ex]
  \begin{axis}[
      xmin=-3, xmax=3,ymin=-3,ymax=3,domain=-3:3,view={0}{90},
      axis lines =center, xlabel=$x$, ylabel=$y$,
      every axis y label/.style={at=(current axis.above origin),anchor=south},
      every axis x label/.style={at=(current axis.right of origin),anchor=west},
      axis on top,
    ] 
    \addplot3 [penColor, quiver={u={1/\length}, v={(x)/(\length)},scale arrows=.2},samples=20] {0};
]  \end{axis}
\end{tikzpicture}}}

\choice{{\def\length{sqrt(1+(y)^2)}
\begin{tikzpicture}[framed,scale=.5,baseline=9ex]
  \begin{axis}[
      xmin=-3, xmax=3,ymin=-3,ymax=3,domain=-3:3,view={0}{90},
      axis lines =center, xlabel=$x$, ylabel=$y$,
      every axis y label/.style={at=(current axis.above origin),anchor=south},
      every axis x label/.style={at=(current axis.right of origin),anchor=west},
      axis on top,
    ] 
    \addplot3 [penColor, quiver={u={1/\length}, v={(y)/(\length)},scale arrows=.2},samples=20] {0};
]  \end{axis}
\end{tikzpicture}}}

\choice{{\def\length{sqrt(1+(-x/y)^2)}
\begin{tikzpicture}[framed,scale=.5,baseline=9ex]
  \begin{axis}[
      xmin=-3, xmax=3,ymin=-3,ymax=3,domain=-3:3,view={0}{90},
      axis lines =center, xlabel=$x$, ylabel=$y$,
      every axis y label/.style={at=(current axis.above origin),anchor=south},
      every axis x label/.style={at=(current axis.right of origin),anchor=west},
      axis on top,
    ] 
    \addplot3 [penColor, quiver={u={1/\length}, v={(-x/y)/(\length)},scale arrows=.2},samples=20] {0};
]  \end{axis}
\end{tikzpicture}}}
\end{selectAll}
\end{prompt}
\begin{hint}
  If a differential equation is pure-time, then all of the slopes
  will be the same on each vertical line.
\end{hint}
\begin{question}
Which of the five slope fields shown are for differential equations
that are neither autonomous nor pure-time?
\begin{prompt}
  \begin{selectAll}
\choice{
{\def\length{sqrt(1+(y^2+y)^2)}
\begin{tikzpicture}[framed,scale=.5,baseline=9ex]
  \begin{axis}[
      xmin=-3, xmax=3,ymin=-3,ymax=3,domain=-3:3,view={0}{90},
      axis lines =center, xlabel=$x$, ylabel=$y$,
      every axis y label/.style={at=(current axis.above origin),anchor=south},
      every axis x label/.style={at=(current axis.right of origin),anchor=west},
      axis on top,
    ] 
    \addplot3 [penColor, quiver={u={1/\length}, v={(y^2+y)/(\length)},scale arrows=.2},samples=20] {0};
]  \end{axis}
\end{tikzpicture}}}

\choice[correct]{{\def\length{sqrt(1+(x-y)^2)}
\begin{tikzpicture}[framed,scale=.5,baseline=9ex]
  \begin{axis}[
      xmin=-3, xmax=3,ymin=-3,ymax=3,domain=-3:3,view={0}{90},
      axis lines =center, xlabel=$x$, ylabel=$y$,
      every axis y label/.style={at=(current axis.above origin),anchor=south},
      every axis x label/.style={at=(current axis.right of origin),anchor=west},
      axis on top,
    ] 
    \addplot3 [penColor, quiver={u={1/\length}, v={(x-y)/(\length)},scale arrows=.2},samples=20] {0};
]  \end{axis}
\end{tikzpicture}}}

\choice{{\def\length{sqrt(1+(x)^2)}
\begin{tikzpicture}[framed,scale=.5,baseline=9ex]
  \begin{axis}[
      xmin=-3, xmax=3,ymin=-3,ymax=3,domain=-3:3,view={0}{90},
      axis lines =center, xlabel=$x$, ylabel=$y$,
      every axis y label/.style={at=(current axis.above origin),anchor=south},
      every axis x label/.style={at=(current axis.right of origin),anchor=west},
      axis on top,
    ] 
    \addplot3 [penColor, quiver={u={1/\length}, v={(x)/(\length)},scale arrows=.2},samples=20] {0};
]  \end{axis}
\end{tikzpicture}}}

\choice{{\def\length{sqrt(1+(y)^2)}
\begin{tikzpicture}[framed,scale=.5,baseline=9ex]
  \begin{axis}[
      xmin=-3, xmax=3,ymin=-3,ymax=3,domain=-3:3,view={0}{90},
      axis lines =center, xlabel=$x$, ylabel=$y$,
      every axis y label/.style={at=(current axis.above origin),anchor=south},
      every axis x label/.style={at=(current axis.right of origin),anchor=west},
      axis on top,
    ] 
    \addplot3 [penColor, quiver={u={1/\length}, v={(y)/(\length)},scale arrows=.2},samples=20] {0};
]  \end{axis}
\end{tikzpicture}}}

\choice[correct]{{\def\length{sqrt(1+(-x/y)^2)}
\begin{tikzpicture}[framed,scale=.5,baseline=9ex]
  \begin{axis}[
      xmin=-3, xmax=3,ymin=-3,ymax=3,domain=-3:3,view={0}{90},
      axis lines =center, xlabel=$x$, ylabel=$y$,
      every axis y label/.style={at=(current axis.above origin),anchor=south},
      every axis x label/.style={at=(current axis.right of origin),anchor=west},
      axis on top,
    ] 
    \addplot3 [penColor, quiver={u={1/\length}, v={(-x/y)/(\length)},scale arrows=.2},samples=20] {0};
]  \end{axis}
\end{tikzpicture}}}
\end{selectAll}
\end{prompt}
  \begin{hint}
  If a differential equation is neither autonomous nor pure-time, then the slope will change along horizontal lines and vertical lines.
\end{hint}
\end{question}
\end{question}
\end{question}


\begin{question}
Consider the differential equation $y' = (y+1)(2-y)$, whose slope
field is given below.  Which of the following statements appear to be
true?

\begin{image}
{\def\length{sqrt(1+((y+1)*(2-y))^2)}
\begin{tikzpicture}
  \begin{axis}[
      xmin=-3, xmax=3,ymin=-3,ymax=3,domain=-3:3,view={0}{90},
      axis lines =center, xlabel=$x$, ylabel=$y$,
      every axis y label/.style={at=(current axis.above origin),anchor=south},
      every axis x label/.style={at=(current axis.right of origin),anchor=west},
      axis on top,
    ] 
    \addplot3 [penColor, quiver={u={1/\length}, v={((y+1)*(2-y))/(\length)},scale arrows=.2},samples=20] {0};
]  \end{axis}
\end{tikzpicture}}
\end{image}

\begin{selectAll}
	\choice[correct]{The solution passing through the origin has $\lim_{x \to \infty} f(x) = 2$.}
	\choice{The solution passing through $(0,3)$ has $\lim_{x \to \infty} f(x) = \infty$.}
	\choice[correct]{The solution passing through $(0,-3)$ is always decreasing.}
	\choice[correct]{There are two solutions which are constant functions.}
	\choice{Every solution has a vertical tangent line at $x=0$.}
\end{selectAll}
\end{question}

Consider the autonomous differential equation:
\[
y'=(y+1)(2-y)
\]
The constant functions $f(x) = -1$ and $f(x) =2$ are solutions to this
differential equation.  In fact, for any autonomous differential
equation $y' = F(y)$, where $F$ is a function of $y$, if $F(c) = 0$
for any constant $c$, then $y = c$ will be a constant solution to the
differential equation.  These constant solutions are also known as
\dfn{equilibrium} solutions. We can witness these solutions if we
inspect the slope field:
\begin{image}
{\def\length{sqrt(1+((y+1)*(2-y))^2)}
\begin{tikzpicture}
  \begin{axis}[
      xmin=-3, xmax=3,ymin=-3,ymax=3,domain=-3:3,view={0}{90},
      axis lines =center, xlabel=$x$, ylabel=$y$,
      every axis y label/.style={at=(current axis.above origin),anchor=south},
      every axis x label/.style={at=(current axis.right of origin),anchor=west},
      axis on top,
    ] 
    \addplot3 [penColor, quiver={u={1/\length}, v={((y+1)*(2-y))/(\length)},scale arrows=.2},samples=20] {0};
    \addplot[penColor,very thick]{-1};
    \addplot[penColor,very thick]{2};
]  \end{axis}
\end{tikzpicture}}
\end{image}
\begin{question}
Consider the autonomous differential equation $y'=\cos(y)$.  Which of
the following are equilibrium solutions?
\begin{selectAll}
\choice{$\pi$}
\choice[correct]{$\frac{\pi}{2}$}
\choice{$0$}
\choice[correct]{$\frac{-\pi}{2}$}
\choice{$-\pi$}
\end{selectAll}
\begin{hint}
  Consider the slope field of $y'=\cos(y)$:
  \begin{image}
    {\def\length{sqrt(1+(cos(deg(y))^2)}
      \begin{tikzpicture}
        \begin{axis}[
            xmin=-3, xmax=3,ymin=-3,ymax=3,domain=-3:3,view={0}{90},
            axis lines =center, xlabel=$x$, ylabel=$y$,
            every axis y label/.style={at=(current axis.above origin),anchor=south},
            every axis x label/.style={at=(current axis.right of origin),anchor=west},
            axis on top,
          ] 
          \addplot3 [penColor, quiver={u={1/\length}, v={cos(deg(y)))/(\length)},scale arrows=.2},samples=20] {0};
    ]
        \end{axis}
\end{tikzpicture}}
\end{image}
\end{hint}
\end{question}

Finally, let's work an example problem:

\begin{example}
  Consider the differential equation:
  \[
  y' = y^2 - 3y+2
  \]
  Find all equilibrium solutions.
  \begin{explanation}
    First note that this differential equation is
    \wordChoice{\choice{pure-time}\choice[correct]{autonomous}\choice{nonautonomous}},
    and hence any solution where $y'=\answer[given]{0}$, will be an
    equilibrium solution. Hence we must solve the equation
    \[
    y^2 -3y + 2 = \answer[given]{0}.
    \]
    Write with me to factor
    \[
    y^2 -3y + 2 = (y-1)(\answer[given]{y-2}),
    \]
    hence $y= 1$ and $y=\answer[given]{2}$ are equilibrium solutions
    to this differential equation.
  \end{explanation}
\end{example}


%% In Figure~\ref{figure:slopeFieldExer}, we see a slope field for a
%% differential equation. If $f(2)=4$, what is your best guess for
%% $f(5)$?

%% $\answer[given]{3}$

%% In Figure~\ref{figure:slopeFieldExer}, we see a slope field for a
%% differential equation. If $f(1)=1.1$, what is your best guess for
%% $f(5)$?
%% $\answer[given]{3}$


%% {\def\length{sqrt(1+(-y^2+4*y-3)^2)}
%% \begin{tikzpicture}
%%   \begin{axis}[
%%       xmin=0, xmax=5,ymin=0,ymax=5,domain=0:5,view={0}{90},
%%       axis lines =center, xlabel=$x$, ylabel=$y$,
%%       every axis y label/.style={at=(current axis.above origin),anchor=south},
%%       every axis x label/.style={at=(current axis.right of origin),anchor=west},
%%       axis on top,
%%     ] 
%%     \addplot3 [penColor, quiver={u={1/\length}, v={(-y^2 +4*y - 3)/(\length)},scale arrows=.2},samples=20] {0};
%%   \end{axis}
%% \end{tikzpicture}}




















\section{Euler's Method}

In science and mathematics, finding exact solutions to differential
equations is not always possible.  We have already seen that slope
fields give us a powerful way to understand the qualitative features
of solutions.  Sometimes we need a more precise quantitative
understanding, meaning we would like numerical approximations of the
solutions.


Again, suppose you have set up the following differential equation
\[
y' = x+y
\]
If we know that $y$ solves this differential equation, and $y(0) = 1$,
how might we go about approximating $y(1)$?  One idea is to repeatedly
use linear approximation.

Let us approximate just using the two subintervals
\[
\left[0,\frac{1}{2}\right]\qquad\text{and}\qquad \left[\frac{1}{2},1\right].
\]
Since $y'(0) = 1$, we know that
\[
y\left(\frac{1}{2}\right)\approx 1 \cdot \frac{1}{2} + 1 = \frac{3}{2}
\]
by linear approximation.
\begin{image}
  \begin{tikzpicture}
    \begin{axis}[
        domain=0:1, xmin =0,xmax=1.1,ymax=3,ymin=-.2,
        width=6in,
        height=3in,
        %xticklabels={$1$,$1.5$,$2$, $2.5$, $3$},
        %% ytick style={draw=none},
        %% yticklabels={},
        axis lines=center, xlabel=$t$, ylabel=$y$,
            every axis y label/.style={at=(current axis.above origin),anchor=south},
            every axis x label/.style={at=(current axis.right of origin),anchor=west},
          ]
      \addplot [draw=penColor,ultra thick] plot coordinates {(0,1) (.5,1.5)};
    \end{axis}
\end{tikzpicture}
\end{image}
But now, we have
\[
y'\left(\frac{1}{2}\right) \approx \frac{1}{2}+\frac{3}{2} = 2
\]
so
\[
y(1) \approx  \frac{1}{2} \cdot 2 + y\left(\frac{1}{2}\right)\approx \frac{5}{2}
\]
\begin{image}
  \begin{tikzpicture}
    \begin{axis}[
        domain=0:1, xmin =0,xmax=1.1,ymax=3,ymin=-.2,
        width=6in,
        height=3in,
        %xticklabels={$1$,$1.5$,$2$, $2.5$, $3$},
        %% ytick style={draw=none},
        %% yticklabels={},
        axis lines=center, xlabel=$t$, ylabel=$y$,
            every axis y label/.style={at=(current axis.above origin),anchor=south},
            every axis x label/.style={at=(current axis.right of origin),anchor=west},
          ]
      \addplot [draw=penColor,ultra thick] plot coordinates {(0,1) (.5,1.5)};
      \addplot [draw=penColor,ultra thick] plot coordinates {(.5,1.5) (1,2.5)};
    \end{axis}
\end{tikzpicture}
\end{image}
Plotting our approximation with the actual solution we find:
\begin{image}
  \begin{tikzpicture}
    \begin{axis}[
        domain=0:1, xmin =0,xmax=1.1,ymax=3,ymin=-.2,
        width=6in,
        height=3in,
        %xticklabels={$1$,$1.5$,$2$, $2.5$, $3$},
        %% ytick style={draw=none},
        %% yticklabels={},
        axis lines=center, xlabel=$t$, ylabel=$y$,
            every axis y label/.style={at=(current axis.above origin),anchor=south},
            every axis x label/.style={at=(current axis.right of origin),anchor=west},
          ]
      \addplot [draw=penColor,very thick] plot coordinates {(0,1) (.5,1.5)};
      \addplot [draw=penColor,very thick] plot coordinates {(.5,1.5) (1,2.5)};
      \addplot [draw=penColor2,ultra thick] {2*e^x-x-1};
    \end{axis}
\end{tikzpicture}
\end{image}
This approximation could be improved by using more subintervals.  We
will now formalize the method of using repeated linear approximation
to approximate solutions to differential equations, and call it
\dfn{Euler's Method}.

\begin{definition}[Euler's Method]
  Consider a differential equation on the interval $[a,b]$
  \[
  y' = F(x,y)
  \]
  where $F$ is a function of two variables and it is given that $y(a)
  = y_0$. To approximate $y$ on the interval:
  \begin{itemize}
  \item Set $x_0=a$ and $y_0 = y(a)$.
  \item Decide either a step-size $\d x$ or how many subintervals $n$ you
    want to divide the interval $[a,b]$ into. Either way:
    \[
    n\cdot \d x = b-a \qquad\text{or}\qquad \frac{b-a}{\d x} = n
    \]
    and $x_k = k\cdot \d x + a$.
  \item Iteratively define
    \[
    y_{k+1} = y_k + F(x_k,y_k)\d x
    \]
    for $k=0,1,\dots,n-1$.
  \end{itemize}
  Then $y(x_k) \approx y_k$, and so $y(b) \approx y_{n}$.
\end{definition}

Let's try an example.

\begin{example}
Use Euler's method to approximate $y$ on $[0,2]$ where $\d x = .5$ for
the differential equation
\[
y'=x+y
\]
where $y(0) = 1$.
\begin{explanation}
Using the notation from the definition, we have
\[
F(x,y) = \answer[given]{x+y}
\]
Fill out the following table for Euler's method using $4$
subintervals.
\[
\begin{array}{l|l|l}
  k & x_k & y_k \\ \hline
  0 & 0   & 1 \\
  1 & 0.5 & \answer[given]{1.5} \\
  2 & \answer[given]{1} & \answer[given]{2.5}  \\
  3 & 1.5 & 4.25 \\
  4 & 2 & \answer[given]{7.125}
\end{array}
\]
Thus $y(2) \approx y_4 =\answer[given]{7.125}$.
\end{explanation}
\end{example}

\end{document}
