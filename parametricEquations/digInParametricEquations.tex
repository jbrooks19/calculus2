\documentclass{ximera}

\input{../preamble.tex}

\outcome{Sketch a parametric curve.}
\outcome{Eliminate a parameters of a parametric equation.}
\outcome{Represent a graph with parametric equations.}
\outcome{Find the derivative of a parametric curve.}

\title[Dig-In:]{Parametric equations}

\begin{document}
\begin{abstract}
Some abstract
\end{abstract}
\maketitle

\section{The idea of parametric equations}

Think back to when you first learned how to graph a function. I'm
pretty sure you used a so-called ``t-chart,'' and if $y = x^2$, I bet it
looked something like this:
\[
\begin{array}{c|c}
  x & y = x^2\\\hline
  0 & 0 \\
  1 & 1\\
  -1 & 1\\
  2 & 4\\
  -2 & 4
\end{array}
\]
With a parametric plot, both $x$ and $y$ are now functions of a third
parameter, we'll call it $t$.

%%
%% probably a three column t-chart goes here
%%

If $x=t$, then there isn't much difference between a regular plot a
the parametric plot.

On the other hand, with parametric functions, we can plot a function
that seems to fail the vertical line test!


(One input, two outputs)

\subsection{Famous parametric equations}

circles


\begin{align*}
  x(t) &= \cos(t)\\
  y(t) &= \sin(t)
\end{align*}
as $t$ runs from $0$ to $2\pi$.



lines
%% Unfortunately, we do vectors *after* parametric eqns.
%% Perhaps we can forshadow the idea of a vector.
\begin{align*}
  \l(t) &= \mathbf{point} + t\cdot \mathbf{direction}\\
  &= (a,b) + t \cdot(u,v)
\end{align*}
where the $\mathbf{direction}$ is imagined as starting at $(0,0)$ and going
to the point $(u,v)$. Another way of writing this is
\[
\l(t) = \left\{\begin{aligned}
  x(t) &= a + t\cdot u\\
  y(t) &= b + t\cdot v
\end{aligned}\right.
\]




\begin{question}
  Consider the parametric curve
  \begin{align*}
  x(t) &= t^2\\
  y(t) &=t^2
  \end{align*}
  Which of the following are the best description of this curve for
  $-\infty<t<\infty$?
  \begin{multipleChoice}
    \choice{It is a parabola.}
    \choice{It is a circle.}
    \choice{It is a square.}
    \choice{It is a line.}
    \choice[correct]{It is a ray.}
  \end{multipleChoice}
\end{question}


others?



\section{Calculus and parametric curves}







\end{document}
