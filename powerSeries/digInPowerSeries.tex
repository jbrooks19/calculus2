\documentclass{ximera}

\input{../preamble.tex}

\title[Dig-In:]{Power series}

\begin{document}
\begin{abstract}
  Infinte series can represent functions.
\end{abstract}
\maketitle

If we refuse to truncate a Talor polynomial, we call it a
\textit{power series}.

\begin{definition}
  A \dfn{power series} is a series of the form
  \[
  \sum_{k=0}^\infty a_k(x-c)^k
  \]
  where the $a_k$'s are the \dfn{coefficients} and $c$ is the
  \dfn{center}.
\end{definition}
\begin{align*}
           e^x &= 1 + x + \frac{x^2}{2!} + \frac{x^3}{3!} + \cdots\\
       \sin(x) &= x - \frac{x^3}{3!} + \frac{x^5}{5!} -\frac{x^7}{7!} + \cdots\\
       \cos(x) &= 1-\frac{x^2}{2!} + \frac{x^4}{4!} -\frac{x^6}{6!} + \cdots\\
 \frac{1}{1-x} &= 1+ x+ x^2 + x^3 + \cdots \qquad |x|< 1
\end{align*}
\end{document}
