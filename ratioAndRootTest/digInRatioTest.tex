\documentclass{ximera}

\input{../preamble.tex}

\outcome{Use the ratio test to determine if a series diverges or converges}

\title[Dig-In:]{The Ratio Test}

\begin{document}
\begin{abstract}
Some infinite series can be compared to geometric series
\end{abstract}
\maketitle

\begin{exploration}
Consider the infinite series

\[
\sum_0^\infty \frac{k}{2^k}
\]

Let $a_k = \frac{k}{2^k}$ be the sequence of terms of this series.

When $k$ is large, $a_{k+1}$ is pretty close to half of $a_{k}$.  The effect of the numerator increasing by $1$ is dwarfed by the effect of the denominator being doubled.

In fact we can formalize this by looking at the ratio of consecutive terms:

\[
\lim_{k \to \infty} \frac{a_{k+1}}{a_k} = \answer{\frac{1}{2}}
\]

Picking a very large $N$, we have $a_{k+1} \approx \frac{1}{2}a_k$ when $k>N$, we get the following approximations:

\begin{align*}
a_{N+1} &\approx \frac{1}{2} a_N\\
a_{N+2} &\approx \answer{\frac{1}{2^2}} a_N\\
a_{N+3} &\approx \answer{\frac{1}{2^3}} a_N\\
\vdots
\end{align*}

In other words, the tail of the sequence $(a_k)$ beginning with $N$ is ``approximately'' a geometric series with ratio $\frac{1}{2}$.

Does a geometric series with ratio $\frac{1}{2}$ converge or diverge?

\begin{multipleChoice}
	\choice[correct]{converge}
	\choice{diverge}
\end{multipleChoice}
 
 Given your answer above, do you suspect that the original sum $\sum_0^\infty \frac{k}{2^k}$ converges or diverges?
 
 \begin{multipleChoice}
	\choice[correct]{converge}
	\choice{diverge}
\end{multipleChoice}
 
\end{exploration}

The above exploration should motivate the following theorem.  The proof of this theorem is slightly beyond the scope of the course.

\begin{theorem}[The Ratio Test]
	Let $\sum a_k$ be an infinite series with positive terms.  If $r = \lim_{k \to \infty} \frac{a_{k+1}}{a_k}$ exists, then
	
	\begin{itemize}
		\item If $0 \leq r < 1$, then the series converges.
		\item If $r>1$, then the series diverges.
		\item If $r = 1$, then we learn nothing:  the series could diverge or converge.
	\end{itemize}
\end{theorem}

Note that this makes sense if you just use the heuristic that ``If the ratio test gives a limit of $r$, then the series is like a geometric series of ratio $r$''.  The case of $r=1$ is an edge case, and can go either way.

In the $4$ problems that follow we will see instances of each of the four possible behaviors:

\begin{itemize}
	\item The ratio test indicating convergence
	\item The ratio test indicating divergence
	\item The ratio test being inconclusive, but the series actually converges
	\item The ratio test being inconclusive, but the series actually diverges
\end{itemize}

It is important that examples illustrating the final two behaviors exist, because it shows that the ratio test really is inconclusive in the case $r=1$.

\begin{question}
	Consider 
	
	\[
	 \sum_4^\infty \frac{2^k}{k!}
	\]
	

	 $\lim_{k \to \infty} \frac{a_{k+1}}{a_k} = \answer{0}$	
	 
	 So the ratio test says that the series is
	 
	 \begin{multipleChoice}
		\choice[correct]{The ratio test says it is convergent}
		\choice{The ratio test says it is divergent}
		\choice{The ratio test gives no information in this case, but we know it is convergent through some other method}
		\choice{The ratio test gives no information in this case, but we know it is divergent through some other method}
	\end{multipleChoice}	
	
	\begin{hint}
		\begin{align*}
			\lim_{k \to \infty} \frac{a_{k+1}}{a_k} &= \lim_{k \to \infty} \frac{2^{k+1}}{(k+1)!} \frac{k!}{2^k}\\
				&=\lim_{k \to \infty} \frac{2}{k+1}\\
				&=0
	\end{align*}
	
	So the series is convergent by the ratio test.
	
	Note that this shows that $k!$ grows \textbf{much faster} than the exponential function $2^k$.
	\end{hint}
\end{question}

\begin{question}
	Consider 
	
	\[
	 \sum_1^\infty \frac{1}{k}
	\]
	

	 $\lim_{k \to \infty} \frac{a_{k+1}}{a_k} = \answer{1}$	
	 

	Which of the following best describes this series?
	 
	 \begin{multipleChoice}
		\choice{The ratio test says it is convergent}
		\choice{The ratio test says it is divergent}
		\choice{The ratio test gives no information in this case, but we know it is convergent through some other method}
		\choice[correct]{The ratio test gives no information in this case, but we know it is divergent through some other method}
	\end{multipleChoice}	
	
	\begin{hint}
		\begin{align*}
			\lim_{k \to \infty} \frac{a_{k+1}}{a_k} &= \lim_{k \to \infty} \frac{k+1}{k}\\
							&=1
		\end{align*}
	
	So the ratio test gives no information.
	
	However, we know that the harmonic series is divergent (we proved this using the integral test).
	\end{hint}
\end{question}

\begin{question}
	Consider 
	
	\[
	 \sum_1^\infty \frac{5^{1+k}}{k 2^{2k+1}}
	\]
	

	 $\lim_{k \to \infty} \frac{a_{k+1}}{a_k} = \answer{\frac{5}{4}}$	
	 
	 So the ratio test says that the series is
	 
	 \begin{multipleChoice}
		\choice{Convergent}
		\choice[correct]{Divergent}
		\choice{The ratio test gives no information in this case}
	\end{multipleChoice}
	
	\begin{hint}
		\begin{align*}
			\lim_{k \to \infty} \frac{a_{k+1}}{a_k} &= \lim_{k \to \infty} \frac{5^{k+2}}{(k+1)2^{2(k+1)+1}} \frac{k 2^{2k+1}}{5^{1+k}}\\
				&=\lim_{k \to \infty} \frac{5^(k+2)}{5^{1+k}} \frac{2^{2k+1}}{2^{2k+3}} \frac{k}{k+1}\\
				&=\lim_{k \to \infty} \frac{5}{4} \frac{k}{k+1}\\
				&=\frac{5}{4}
	\end{align*}
	
	So the series is divergent by the ratio test.
	\end{hint}
\end{question}

\begin{question}
	Consider 
	
	\[
	 \sum_1^\infty \frac{1}{k^2}
	\]
	

	 $\lim_{k \to \infty} \frac{a_{k+1}}{a_k} = \answer{1}$	
	 

	Which of the following best describes this series?
	 
	 \begin{multipleChoice}
		\choice{The ratio test says it is convergent}
		\choice{The ratio test says it is divergent}
		\choice[correct]{The ratio test gives no information in this case, but we know it is convergent through some other method}
		\choice{The ratio test gives no information in this case, but we know it is divergent through some other method}
	\end{multipleChoice}	
	
	\begin{hint}
		\begin{align*}
			\lim_{k \to \infty} \frac{a_{k+1}}{a_k} &= \lim_{k \to \infty} \frac{(k+1)^2}{k^2}\\
							&= \lim_{k \to \infty} \left( \frac{k+1}{k}\right)^2\\
							&= \lim_{k \to \infty} \left( 1+\frac{1}{k}\right)^2\\
							&= 1
		\end{align*}
	
	So the ratio test gives no information.
	
	However, we know that this series is convergent by the $p$ series test with $p=2$ (which ultimately derives from the integral test)
	\end{hint}
\end{question}



\end{document}


