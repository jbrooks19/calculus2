\documentclass{ximera}

\input{../preamble.tex}

\outcome{Use the root test to determine if a series diverges or converges.}

\title[Dig-In:]{The Root Test}

\begin{document}
\begin{abstract}
Some infinite series can be compared to geometric series.
\end{abstract}
\maketitle

In the Ratio Test section, we learned that it is powerful to recognize
that a series is ``approximately'' geometric, because we know how to
decide the convergence and divergence of geometric series.  If
\[
\lim_{k \to \infty} \frac{a_{k+1}}{a_k} = r,
\]
then the ``tail'' of the series looks like a geometric series of ratio
$r$, and follows the same convergence and divergence behavior.  The
\textit{root test} uses the same idea in a slightly different
situation.
\begin{theorem}[The Root Test]
  If $\sum^\infty a_k$ is an infinite series of positive terms, and $\lim_{k \to \infty} \sqrt[k]{a_k} = r$, then 
  \begin{itemize}
  \item If $0 \leq r < 1$, then the series converges.
  \item If $r>1$, then the series diverges.
  \item If $r = 1$, then we learn nothing:  the series could diverge or converge.
  \end{itemize}
\end{theorem}
Notice that the conclusion of the root test follows exactly the same
form as the ratio test.  It does so for exactly the same reason:
\begin{quote}
  If $\sqrt[k]{a_k} \approx r$ for large $k$, then $a_k \approx r^k$
  for large $k$, which exactly says that the tail of $a_k$ looks like
  a geometric series with ratio $r$.
\end{quote}
Again, we do not give a formal proof in this course (but if you are
interested you can find a proof online!)

\begin{example}
  Consider 
  \[
  \sum_{k=4}^\infty \frac{k^5}{k^k}
  \]
  Discuss the convergence of this series.
  \begin{explanation}
    We will attempt to use the root test. Write with me
    \[
    \lim_{k \to \infty} \sqrt[k]{a_k} = \answer[given]{0}	
    \]
    so the root test
	  \wordChoice{
	   \choice[correct]{says the series is convergent}
	   \choice{says the series is divergent}
	   \choice{gives no information in this case, but we know the series is convergent through some other method}
	   \choice{gives no information in this case, but we know the series divergent through some other method}
	 }.		
	  \begin{hint}
	    \begin{align*}
	      \lim_{k \to \infty} \sqrt[k]{a_k} &= \lim_{k \to \infty} \sqrt[k]{\frac{k^5}{k^k}}\\
	      &=\lim_{k \to \infty} \frac{(k^{\frac{1}{k}})^5}{k}\\
	      &=0 \textrm{ since $\lim_{k \to \infty} k^\frac{1}{k} = 1$}
	    \end{align*}
	    So the series is convergent by the root test.
	  \end{hint}
  \end{explanation}
\end{example}
  
  Generally the Root test is most useful when you have a lot of powers, and no factorials.  Anytime you see a factorial is a pretty good sign to try the Ratio test.







\end{document}


