\documentclass{ximera}

\input{../preamble.tex}

\title[Dig-In:]{Sequences}

\begin{document}
\begin{abstract}
  A function from positive integers to the real numbers is a sequence.
\end{abstract}
\maketitle

Let's get to the heart of the matter:

\begin{definition}
  A \dfn{sequence} (of things) is an ordered list of things.
\end{definition}

In our class, we're mostly talking about a sequence of real numbers.
For example, here is a sequence, meaning an ordered list of numbers:
\[
1,1, 2, 3, 5, 8, 13, 21, \ldots
\]
Note that numbers in the list can repeat.  And consider those little
dots at the end!  The dots ``\ldots'' signify that the list keeps
going, and going, and going, forever.

Sequences are written down in a few different, but equivalent,
ways; you might see a sequence written as
\begin{align*}
  & a_1, a_2,  a_3, \ldots, \\
  & a_n \\
  & \left(a_n\right)_{n \in \N}, \\
  & \left\{a_n\right\}_{n=1}^\infty, \\
  & \left\{f(n)\right\}_{n=1}^\infty, \quad \text{or} \\
  & \left(f(n)\right)_{n \in \N},
\end{align*}
depending on which author you read, here $\N=\{1,2,3,\ldots\}$.  Worse, depending on the
situation, the same author (and this author) might use various
notations for a sequence!  We will usually write $(a_n)$ if I want to
speak of the sequence as a whole and write $a_n$ if we are speaking of
a specific element in the sequence.


\begin{question}
  Considering the sequence
  \[
  1, 2, 4, 8, 16, \dots
  \]
  what number comes next?
  \begin{multipleChoice}
    \choice{$32$}
    \choice{$31$}
    \choice{$18$}
    \choice[correct]{there is no way to know}
  \end{multipleChoice}
  \begin{feedback}
    Sequences need to be defined by a ``rule.'' Without such a rule,
    it is difficult, if not impossible, to say which element comes
    next.
  \end{feedback}
\end{question}

\begin{example}
  Let $a_n = 2^{n-1}$.  Write down $a_1$, $a_2$, $a_3$, $a_4$, $a_5$,
  $a_6$.
  \begin{explanation}
    Just use the formula to see:
    \begin{align*}
      a_1 &= \answer[given]{1}\\
      a_2 &= \answer[given]{2}\\
      a_3 &= \answer[given]{4}\\
      a_4 &= \answer[given]{8}\\
      a_5 &= \answer[given]{16}\\
      a_6 &= \answer[given]{32}
    \end{align*}
  \end{explanation}
\end{example}


\begin{example}
  Consider a circle with $n$ points on it. Let $b_n$ be the maximum
  number of regions produced by connecting these points with
  chords. Write down $b_1$, $b_2$, $b_3$, $b_4$, $b_5$, $b_6$.
  \begin{explanation}
    There is only one way to solve this problem: start drawing
    pictures and counting regions.
    \begin{itemize}
      \item \begin{tikzpicture}      
            \tkzDefPoint(0,0){O} 
            \tkzDefPoint(1,0){A} 
            \tkzDrawCircle[color=penColor,very thick](O,A)
            \tkzDrawPoint[color=penColor,fill=penColor](A)
      \end{tikzpicture}
      \item \begin{tikzpicture}      
            \tkzDefPoint(0,0){O} 
            \tkzDefPoint(1,0){A} 
            \tkzDefPoint(-1,0){B}
            \tkzDrawCircle[color=penColor,very thick](O,A)
            \tkzDrawPoint[color=penColor,fill=penColor](A)
            \tkzDrawPoint[color=penColor,fill=penColor](B)
            \tkzDrawSegment[color=penColor](A,B)
      \end{tikzpicture}
      \item \begin{tikzpicture}      
            \tkzDefPoint(0,0){O} 
            \tkzDefPoint(1,0){A} 
            \tkzDefPoint(-.6,.8){B}
            \tkzDefPoint(-.6,-.8){C}
            \tkzDrawCircle[color=penColor,very thick](O,A)
            \tkzDrawPoint[color=penColor,fill=penColor](A)
            \tkzDrawPoint[color=penColor,fill=penColor](B)
            \tkzDrawPoint[color=penColor,fill=penColor](C)
            \tkzDrawSegment[color=penColor](A,B)
            \tkzDrawSegment[color=penColor](A,C)
            \tkzDrawSegment[color=penColor](B,C)
      \end{tikzpicture}
      \item \begin{tikzpicture}      
            \tkzDefPoint(0,0){O} 
            \tkzDefPoint(.6,.8){A} 
            \tkzDefPoint(-.6,.8){B}
            \tkzDefPoint(-.6,-.8){C}
            \tkzDefPoint(.6,-.8){D}
            \tkzDrawCircle[color=penColor,very thick](O,A)
            \tkzDrawPoint[color=penColor,fill=penColor](A)
            \tkzDrawPoint[color=penColor,fill=penColor](B)
            \tkzDrawPoint[color=penColor,fill=penColor](C)
            \tkzDrawPoint[color=penColor,fill=penColor](D)
            \tkzDrawSegment[color=penColor](A,B)
            \tkzDrawSegment[color=penColor](A,C)
            \tkzDrawSegment[color=penColor](A,D)
            \tkzDrawSegment[color=penColor](B,C)
            \tkzDrawSegment[color=penColor](B,D)
            \tkzDrawSegment[color=penColor](C,D)
      \end{tikzpicture}
        \item \begin{tikzpicture}      
            \tkzDefPoint(0,0){O} 
            \tkzDefPoint(1,0){A} 
            \tkzDefPoint(.27,.96){B}
            \tkzDefPoint(-.8,.6){C}
            \tkzDefPoint(-.8,-.6){D}
            \tkzDefPoint(.33,-.95){E}
            \tkzDrawCircle[color=penColor,very thick](O,A)
            \tkzDrawPoint[color=penColor,fill=penColor](A)
            \tkzDrawPoint[color=penColor,fill=penColor](B)
            \tkzDrawPoint[color=penColor,fill=penColor](C)
            \tkzDrawPoint[color=penColor,fill=penColor](D)
            \tkzDrawPoint[color=penColor,fill=penColor](E)
            \tkzDrawSegment[color=penColor](A,B)
            \tkzDrawSegment[color=penColor](A,C)
            \tkzDrawSegment[color=penColor](A,D)
            \tkzDrawSegment[color=penColor](A,E)
            \tkzDrawSegment[color=penColor](B,C)
            \tkzDrawSegment[color=penColor](B,D)
            \tkzDrawSegment[color=penColor](B,E)
            \tkzDrawSegment[color=penColor](C,D)
            \tkzDrawSegment[color=penColor](C,E)
            \tkzDrawSegment[color=penColor](D,E)
        \end{tikzpicture}
        \item \begin{tikzpicture}      
            \tkzDefPoint(0,0){O} 
            \tkzDefPoint(1,0){A} 
            \tkzDefPoint(0,1){B}
            \tkzDefPoint(-.8,.6){C}
            \tkzDefPoint(-.8,-.6){D}
            \tkzDefPoint(0,-1){E}
            \tkzDefPoint(.7,-.71){F}
            \tkzDrawCircle[color=penColor,very thick](O,A)
            \tkzDrawPoint[color=penColor,fill=penColor](A)
            \tkzDrawPoint[color=penColor,fill=penColor](B)
            \tkzDrawPoint[color=penColor,fill=penColor](C)
            \tkzDrawPoint[color=penColor,fill=penColor](D)
            \tkzDrawPoint[color=penColor,fill=penColor](E)
            \tkzDrawPoint[color=penColor,fill=penColor](F)
            \tkzDrawSegment[color=penColor](A,B)
            \tkzDrawSegment[color=penColor](A,C)
            \tkzDrawSegment[color=penColor](A,D)
            \tkzDrawSegment[color=penColor](A,E)
            \tkzDrawSegment[color=penColor](A,F)
            \tkzDrawSegment[color=penColor](B,C)
            \tkzDrawSegment[color=penColor](B,D)
            \tkzDrawSegment[color=penColor](B,E)
            \tkzDrawSegment[color=penColor](B,F)
            \tkzDrawSegment[color=penColor](C,D)
            \tkzDrawSegment[color=penColor](C,E)
            \tkzDrawSegment[color=penColor](C,F)
            \tkzDrawSegment[color=penColor](D,E)
            \tkzDrawSegment[color=penColor](D,F)
            \tkzDrawSegment[color=penColor](E,F)
            \end{tikzpicture}
    \end{itemize}
  \end{explanation}
\end{example}




numbers, like $f(x)=\sin (1/x)$.

A real-valued function with domain the natural numbers
$\N=\{1,2,3,\ldots\}$ is a \dfn{sequence}.

Other functions will also be regarded as sequences: the domain might
include $0$ alongside the positive integers, meaning that the
domain is the non-negative integers, $\Z^{\ge0}=\{0,1,2,3,\ldots\}$.  The range of the function is still
allowed to be the real numbers; in symbols, the function $f\colon
\N\to\R$ is a sequence.




Recall that the natural numbers $\N$ are the counting numbers $1, 2,
3, 4, \ldots$.  If we want our sequence to start at zero, we use
$\Z^{\ge 0}$ as the domain instead.  The fancy symbols $\Z^{\ge 0}$
refer to the non-negative integers, which include zero (since zero
is neither positive nor negative) and also positive integers (since
they certainly aren't negative).

To confuse matters further, some people---especially computer
scientists---might include zero in the natural numbers $\N$.
Mathematics is cultural.


\begin{warning}
  Usually the ``domain'' of a sequence is $\N$ and $\Z^{\ge 0}$.  But
  depending on the context, it may be convenient for a sequence to
  start somewhere else---perhaps with some negative number.  We
  shouldn't let the usual situation of $\N$ or $\Z^{\ge 0}$ get in the
  way of making the best choice for the problem at hand.
\end{warning}

As you can tell, there is a deep tension between precise definition
and a vague flexibility; as instructors, how we navigate that tension
will be a big part of whether we are successful in teaching the
course.  We need to invoke precision when we're tempted to be too
vague, and we need to reach for an extra helping of vagueness when the
formalism is getting in the way of our understanding.  It can be a
tough balance.




subsection{Defining sequences by giving a rule}

Just as real-valued functions were usually expressed by a formula, we
will most often encounter sequences that can be expressed by a
formula.  In the Introduction to this textbook, we saw the sequence
given by the rule $a_i=f(i)=1-1/2^i$.  Other examples are easy to
cook up, like
\begin{align*}
  a_i &={i\over i+1}, \\
  b_n &={1\over2^n}, \\
  c_n &=\sin(n\pi/6), \mbox{ or} \\
  d_i &={(i-1)(i+2)\over2^i}. \\
\end{align*}
Frequently these formulas will make sense if thought of either as
functions with domain $\R$ or $\N$, though occasionally the given
formula will make sense only for integers.  We'll address the
idea of a real-valued function ``filling in'' the gaps between the
terms of a sequence when we look at graphs in
%Section~\xrefn{section:graphs}.

\begin{warning}
  A common misconception is to confuse the sequence with the rule for
  generating the sequence.  The sequences $(a_n)$ and $(b_n)$ given by
  the rules $a_n = (-1)^n$ and $b_n = \cos (\pi \, n)$ are, despite
  appearances, different rules which give rise to the \textit{same}
  sequence.  These are just different names for the same object.
\end{warning}

Let's give a precise definition for ``the same'' when speaking of
sequences.  Compare this to equality for functions: two functions are
the same if they have same domain and codomain, and they assign the
same value to each point in the domain.

\youtube{https://www.youtube.com/watch?v=5f5-d3uiYxo}

\begin{definition}
  Suppose $(a_n)$ and $(b_n)$ are sequences starting at $1$.  These
  sequences are \dfn{equal}\index{sequence!equality} if for all
  natural numbers $n$, we have $a_n = b_n$.

  More generally, two sequences $(a_n)$ and $(b_n)$ are
  \dfn{equal} if they have the same initial index $N$, and for
  every integer $n \geq N$, the $n$th terms have the same value, that is,
  \[
  a_n = b_n \quad \mbox{for all $n \geq N$.}
  \]
\end{definition}
In other words, sequences are the same if they have the same set of
valid indexes, and produce the same real numbers for each of those
indexes---regardless of whether the given ``rules'' or procedures for
computing those sequences resemble each other in any way.

\subsection{Defining sequences using previous terms}
\label{subsection:recursive-definition}

Another way to define a sequence is \textit{recursively}, that is, by
defining the later outputs in terms of previous outputs.  We start by
defining the first few terms of the sequence, and then describe how
later terms are computed in terms of previous terms.

\begin{example}
Define a sequence recursively by
$$
a_1 = 1, \quad a_2 = 3, \quad a_3 = 10,
$$
and the rule that $a_n = a_{n-1} - a_{n-3}$.  Compute $a_5$.
\end{example}

\begin{explanation}
  First we compute $a_4$.  Substituting $4$ for $n$ in the rule $a_n = a_{n-1} - a_{n-3}$, we find
$$
a_4 = a_{4-1} - a_{4-3} = a_3 - a_1.
$$
But we have values for $a_3$ and $a_1$, namely $10$ and $1$, respectively.  Therefore $a_4 = 10 - 1 = 9$.

Now we are in a position to compute $a_5$.  Substituting $5$ for $n$ in the rule $a_n = a_{n-1} - a_{n-3}$, we find
$$
a_5 = a_{5-1} - a_{5-3} = a_4 - a_2.
$$
We just computed $a_4 = 9$; we were given $a_2 = 3$.  Therefore $a_5 = 9 - 3 = 6$.
\end{explanation}

\youtube{http://www.youtube.com/watch?v=Krqi7UGJV5o}

\begin{question}
  Consider the sequence $a_{n}$ defined recursively by the
  rule \[a_n = {a_{n-1}} {a_{n-2}} + 3 \, {a_{n-1}} - {a_{n-2}}\] and
  the facts that $a_0 = -3$ and $a_1 = 5$.  What is $a_4$?

    \begin{hint}
      We have been told the first two terms of the sequence, namely $a_0 = -3$ and $a_1 = 5$.
    \end{hint}
    \begin{hint}
      We also have a rule $a_n = {a_{n-1}} {a_{n-2}} + 3 \, {a_{n-1}} - {a_{n-2}}$ which lets us compute the third term $a_2$ using these first two terms $a_0$ and $a_1$.
    \end{hint}
    \begin{hint}
      To compute $a_2$, we set $n = 2$ in the recursive rule $a_n = {a_{n-1}} {a_{n-2}} + 3 \, {a_{n-1}} - {a_{n-2}}$.  This gives us $a_2 = {a_{2-1}} {a_{2-2}} + 3 \, {a_{2-1}} - {a_{2-2}}$.
    \end{hint}
    \begin{hint}
      Plugging $a_{2-1} = a_{1} = 5$ and $a_{2-2} = a_{0} = -3$ into the rule, we learn $a_2 = 3$.
    \end{hint}
    \begin{hint}
      To compute $a_3$, we set $n = 3$ in the recursive rule $a_n = {a_{n-1}} {a_{n-2}} + 3 \, {a_{n-1}} - {a_{n-2}}$, giving $a_3 = {a_{3-1}} {a_{3-2}} + 3 \, {a_{3-1}} - {a_{3-2}}$.
    \end{hint}
    \begin{hint}
      To evaluate that, we will have to know $a_{3-1} = a_{2}$, but we just found out that $a_{2} = 3$.
    \end{hint}
    \begin{hint}
      Plugging $a_{3-1} = 3$ and $a_{3-2} = a_{1} = 5$ into the rule, we find $a_3 = 19$.
    \end{hint}
    \begin{hint}
      To compute $a_4$, we set $n = 4$ in the recursive rule $a_n = {a_{n-1}} {a_{n-2}} + 3 \, {a_{n-1}} - {a_{n-2}}$, giving $a_4 = {a_{4-1}} {a_{4-2}} + 3 \, {a_{4-1}} - {a_{4-2}}$.
    \end{hint}
    \begin{hint}
      To evaluate that, we will have to know $a_{4-1} = a_{3}$, but we just found out that $a_{3} = 19$.
    \end{hint}
    \begin{hint}
      Plugging $a_{4-1} = 19$ and $a_{4-2} = a_{2} = 3$ into the rule, we learn $a_4 = 111$.
    \end{hint}
    \begin{hint}
      So we conclude $a_4 = 111$.
    \end{hint}

    \begin{multipleChoice}
      \choice[correct]{$111$}
      \choice{$176201$}
      \choice{$-40$}
      \choice{$-654016$}
      \choice{$2522$}
    \end{multipleChoice}

\end{question}


\section{Arithmetic sequences}


The first family we consider are the ``arithmetic'' sequences.  Here
is a definition.

\youtube{https://www.youtube.com/watch?v=WOF5kkVekUY}

%{Mathematically, the word \dfn{family}
%  does not have an entirely precise definition; a family of things is
%  a \dfn{collection} or a \dfn{set} of things, but family
%  also has a connotation of some sort of relatedness.} 

\begin{definition}
  An \dfn{arithmetic progression} (sometimes called an arithmetic
  sequence)\index{arithmetic progression} is a sequence where each
  term differs from the next by the same, fixed quantity.
\end{definition}

\begin{example}
  An example of an arithmetic progression is the sequence $(a_n)$ which begins 
  $$
  a_1 = 10, \quad a_2 = 14, \quad a_3 = 18, \quad a_4 = 22, \quad\ldots
  $$
  and which is given by the rule $a_n = 6 + 4 \, n$.  Each term differs
  from the previous by four.
\end{example}

In general, an arithmetic progression in which subsequent terms differ
by $m$ can be written as
$$
a_n = m \, (n-1) + a_1.
$$
Alternatively, we could describe an arithmetic progression
recursively, by giving a starting value $a_1$, and using the rule that
$a_{n} = a_{n-1} + m$.

%\marginnote{Why are arithmetic progressions called \textit{arithmetic?}  Note that every term is the \dfn{arithmetic mean}, that is, the \dfn{average}, of its two neighbors.}

An arithmetic progression can decrease; for instance,
$$
17,\quad  15,\quad  13,\quad  11,\quad  9, \quad\ldots
$$
is an arithmetic progression.

\begin{question}
  Which of the following could be the initial terms of an arithmetic progression?

    \begin{hint}
      In an arithmetic progression, there is a common difference between any two neighboring terms.
    \end{hint}
    \begin{hint}
      For instance, the difference between $-3$ and $-1$ is $-2$.
    \end{hint}
    \begin{hint}
      Can $3,  1,  -1,  -4,  -6,  -8 $ be the beginning of an arithmetic progression?  No, because the difference most between of those terms is $-2$, but $-1$ and $-4$ break the pattern, with difference $-3$, not $-2$.
    \end{hint}
    \begin{hint}
      Can $3,  1,  -1,  -3,  -4,  -6 $ be the beginning of an arithmetic progression?  No, because the difference between most of those terms is $-2$, but $-3$ and $-4$ break the pattern, with difference $-1$, not $-2$.
    \end{hint}
    \begin{hint}
      Can $3,  1,  -1,  -3,  -5,  -8 $ be the beginning of an arithmetic progression?  No, because the difference between most of those terms is $-2$, but $-5$ and $-8$ break the pattern, with difference $-3$, not $-2$.
    \end{hint}
    \begin{hint}
      Can $3,  1,  -1,  -3,  -5,  -7 $ be the beginning of an arithmetic progression?  Yes, because the difference between each neighboring pair of terms is $-2$.  For example, 
      $\left(-1\right) - \left(1\right) = \left(-3\right) - \left(-1\right) = \left(-5\right) - \left(-3\right) = -2$.
    \end{hint}

    \begin{multipleChoice}
      \choice[correct]{$3,\quad 1,\quad -1,\quad -3,\quad -5,\quad -7,\quad\ldots $}
      \choice{$3,\quad 0,\quad -2,\quad -4,\quad -6,\quad -8,\quad\ldots $}
      \choice{$3,\quad 1,\quad 0,\quad -2,\quad -4,\quad -6,\quad\ldots $}
      \choice{$3,\quad 1,\quad -1,\quad -4,\quad -6,\quad -8,\quad\ldots $}
      \choice{$3,\quad 1,\quad -1,\quad -3,\quad -4,\quad -6,\quad\ldots $}
    \end{multipleChoice}

\end{question}



\section{Geometric sequences}


The second family we consider are geometric progressions.

\youtube{https://www.youtube.com/watch?v=1z8QKFFU3Hc}

\begin{definition}
  A \dfn{geometric progression} (sometimes called a geometric
  sequence)\index{geometric progression} is a sequence where the ratio
  between subsequent terms is the same, fixed quantity.
\end{definition}

\begin{example}
  An example of a geometric progression is the sequence $(a_n)$ starting
  $$
  a_1 = 10, \quad a_2 = 30, \quad a_3 = 90, \quad a_4 = 270, \quad\ldots
  $$
  and given by the rule $a_n = 10 \cdot 3^{n-1}$.  Each term is three
  times the preceding term.
\end{example}

In general, a geometric progression in which the ratio between
subsequent terms is $r$ can be written as
$$
a_n = a_1 \cdot r^{n-1}.
$$
Alternatively, we could describe a geometric progression
recursively, by giving a starting value $a_1$, and using the rule that
$a_{n} = r \cdot a_{n-1}$.

\begin{remark}
Why are geometric progressions called \textit{geometric?}  Note that every term is the \dfn{geometric mean} of its two neighbors.  The geometric mean of two numbers $a$ and $b$ is defined to be $\sqrt{ab}$.

Of course, that raises another question: why is the geometric mean called \textit{geometric?}  One geometric interpretation of the geometric mean of $a$ and $b$ is this: the geometric mean is the side length of a square whose area is equal to that of the rectangle having side lengths $a$ and $b$.
\end{remark}

A geometric progression needn't be increasing.  For instance, in the following geometric progression
$$
\frac{7}{5}, \quad \frac{7}{10}, \quad \frac{7}{20}, \quad \frac{7}{40}, \quad \frac{7}{80}, \quad \frac{7}{160}, \quad\ldots
$$
the ratio between subsequent terms is one half, and each term is smaller than the previous.

\begin{question}
  Which of the following could be the initial terms of a geometric progression?
              
    \begin{hint}
      In a geometric progression, there is a common ratio between any two neighboring terms.
    \end{hint}
    \begin{hint}
      For instance, the ratio between $-64$ and $16$ is $-4$.
    \end{hint}
    \begin{hint}
      Can $1,  -4,  16,  -48,  192,  -768 $ be the beginning of a geometric progression?  No, because the ratio between most of those terms is $-4$, but $16$ and $-48$ break the pattern: the ratio $\displaystyle\displaystyle\frac{-48}{16}$ is -3, not -4.
    \end{hint}
    \begin{hint}
      Can $1,  -4,  16,  -64,  320,  -1280 $ be the beginning of a geometric progression?  No, because the ratio between most of those terms is $-4$, but $-64$ and $320$ break the pattern: the ratio $\displaystyle\displaystyle\frac{320}{-64}$ is -5, not -4.
    \end{hint}
    \begin{hint}
      Can $1,  -4,  16,  -64,  256,  -1280 $ be the beginning of a geometric progression?  No, because the ratio between most of those terms is $-4$, but $256$ and $-1280$ break the pattern: the ratio $\displaystyle\displaystyle\frac{-1280}{256}$ is -5, not -4.
    \end{hint}
    \begin{hint}
      Can $1,  -4,  16,  -64,  256,  -1024 $ be the beginning of a geometric progression?  Yes, because the ratio between each neighboring pair of terms is $-4$.  For example, 
      $\displaystyle\displaystyle\frac{16}{-4} = \displaystyle\displaystyle\frac{-64}{16} = \displaystyle\displaystyle\frac{256}{-64} = -4$.
    \end{hint}

    \begin{multipleChoice}
      \choice[correct]{$1,\quad -4,\quad 16,\quad -64,\quad 256,\quad -1024,\quad\ldots $}
      \choice{$1,\quad -5,\quad 20,\quad -80,\quad 320,\quad -1280,\quad\ldots $}
      \choice{$1,\quad -4,\quad 20,\quad -80,\quad 320,\quad -1280,\quad\ldots $}
      \choice{$1,\quad -4,\quad 16,\quad -48,\quad 192,\quad -768,\quad\ldots $}
      \choice{$1,\quad -4,\quad 16,\quad -64,\quad 320,\quad -1280,\quad\ldots $}
    \end{multipleChoice}


\end{question}


\section{Collatz}


Here is a fun sequence with which to amuse your friends---or distract
your enemies.  Let's start our sequence with $a_1 = 6$.  Subsequent
terms are defined using the rule
$$
a_n = \begin{cases} a_{n-1} / 2 & \mbox{ if $a_{n-1}$ is even, and } \\
3 \, a_{n-1} + 1 & \mbox{ if $a_{n-1}$ is odd.}
\end{cases}
$$
Let's compute $a_2$.  Since $a_1$ is even, we follow the instructions
in the first line, to find that $a_2 = a_1/2 = 3$. To compute $a_3$,
note that $a_2$ is odd so we follow the instruction in the second
line, and $a_3 = 3 \, a_2 + 1 = 3 \cdot 3 + 1 = 10$.  Since $a_3$ is
even, the first line applies, and $a_4 = a_3 / 2 = 10 / 2 = 5$.  But
$a_4$ is odd, so the second line applies, and we find $a_5 = 3 \cdot 5
+ 1 = 16$.  And $a_5$ is even, so $a_6 = 16 / 2 = 8$.  And $a_6$ is
even, so $a_7 = 8/4 = 4$.  And $a_7$ is even, so $a_8 = 4 / 2 = 2$,
and then $a_9 = 2/2 = 1$.  Oh, but $a_9$ is odd, so $a_{10} = 3 \cdot
1 + 1 = 4$.  And it repeats.  Let's write down the start of this sequence:
$$
6,\quad %1 
3,\quad %2
10,\quad  %3
5,\quad  %4
16,\quad  %5
8,\quad  %6
4,\quad  %7
2,\quad  %8
1,\quad  %9
4,\quad %10
2,\quad %11
1,\quad %12
\overbrace{4,\quad %10
2,\quad %11
1,}^{\mbox{repeats}}\quad %12
4,\quad %10
\ldots
$$
What if we had started with a number other than six?  What if we set
$a_1 = 25$ but then we used the same rule?  In that case, since $a_1$
is odd, we compute $a_2$ by finding $3 \, a_1 + 1 = 3 \cdot 25 + 1 =
76$.  Since $76$ is even, the next term is half that, meaning $a_3 =
38$.  If we keep this up, we find that our sequence begins
\begin{align*}
&25,\quad 76,\quad 38,\quad 19,\quad 58,\quad 29,\quad 88,\quad 44,\quad 22,\quad 11,\quad 34,\quad 17,\quad 52,\quad 26, \\
&13,\quad 40,\quad 20,\quad 10,\quad 5,\quad 16,\quad 8,\quad 4,\quad 2, \quad 1, \quad \ldots
\end{align*}
and then it repeats ``4, 2, 1, 4, 2, 1, \ldots'' just like before.

If you think you have an argument that answers the Collatz conjecture, I challenge you to try your hand at the $5x+1$ conjecture, that is, use the rule
\[
a_n = \displaystyle\begin{cases} a_{n-1} / 2 & \mbox{ if $a_{n-1}$ is even, and } \\
5 \, a_{n-1} + 1 & \mbox{ if $a_{n-1}$ is odd.}
\end{cases}
\]

Does this always happen?  Is it true that no matter which positive
integer you start with, if you apply the half-if-even, $3x+1$-if-odd
rule, you end up getting stuck in the ``4, 2, 1, \ldots'' loop?  That
this is true is the \dfn{Collatz conjecture}; it has been
verified for all starting values below $5 \times 2^{60}$.  Nobody has
found a value which doesn't return to one, but for all anybody knows
there \textit{might} well be a very large initial value which doesn't
return to one; nobody knows either way.  It is an unsolved
problem


This is not the last unsolved problem we will
  encounter in this course.  There are many things which humans do not
  understand. in mathematics.



\end{document}
