\documentclass{ximera}

\input{../preamble.tex}

\outcome{Recognize a geometric series.}
\outcome{Recognize a telescoping series.}
\outcome{Compute the sum of a geometric series.}
\outcome{Compute the sum of a telescoping series.}

\title[Dig-In:]{Series}

\begin{document}
\begin{abstract}
A series is summation of a sequence.
\end{abstract}
\maketitle


Given the sequence $(a_n) = (1/2^n) = 1/2, 1/4, 1/8, \ldots$, consider
the following sums:
\[
\begin{array}{ccccc}
a_1				&=& 1/2					 &=& 1/2\\
a_1+a_2		&=& 1/2+1/4			 &=& 3/4\\
a_1+a_2+a_3 &=& 1/2+1/4+1/8  &=& 7/8\\
a_1+a_2+a_3+a_4 &=& 1/2+1/4+1/8+1/16 & =& 15/16
\end{array}
\]
In general, we can show that $$a_1+a_2+a_3+\cdots +a_n =
\frac{2^n-1}{2^n} = 1-\frac{1}{2^n}.$$ Let $S_n$ be the sum of the
first $n$ terms of the sequence $(1/2^n)$. From the above, we see
that $S_1=1/2$, $S_2 = 3/4$, etc. Our formula at the end shows that
$S_n = 1-1/2^n$.

Now consider the following limit: $\lim_{n\to\infty}S_n =
\lim_{n\to\infty}\big(1-1/2^n\big) = 1$. This limit can be interpreted
as saying something amazing: \emph{the sum of \emph{all} the terms of
  the sequence $(1/2^n)$ is 1.}

This example illustrates some interesting concepts that we explore in
this section. We begin this exploration with some definitions.

\begin{definition}
Let $(a_n)$ be a sequence.
\begin{enumerate}
\item The sum $\sum_{n=1}^\infty a_n$ is an \dfn{infinite series} (or,
  simply \dfn{series}).
\item Let $S_n = \sum_{i=1}^n a_i$; the sequence $(S_n)$ is the
  sequence of $n^\text{th}$ \dfn{partial sums} of $(a_n)$.
\item If the sequence $(S_n)$ converges to $L$, we say the series
  $\sum_{n=1}^\infty a_n$ \dfn{converges} to $L$, and we write
  $\sum_{n=1}^\infty a_n = L$.
\item If the sequence $(S_n)$ diverges, the series $\sum_{n=1}^\infty
  a_n$ \dfn{diverges}.
\end{enumerate}
\end{definition}

Using our new terminology, we can state that the series
$\sum_{n=1}^\infty 1/2^n$ converges, and $\sum_{n=1}^\infty 1/2^n =
1.$

We will explore a variety of series in this section. We start with two
series that diverge, showing how we might discern divergence.

\begin{example}
\begin{enumerate}
\item Let $(a_n) = (n^2)$. Show $\sum_{n=1}^\infty a_n$ diverges.
\item Let $(b_n) = ((-1)^{n+1})$. Show $\sum_{n=1}^\infty b_n$
  diverges.
\end{enumerate}
\begin{explanation}
\begin{enumerate}
\item Consider $S_n$, the $n^\text{th}$ partial sum.
  \begin{align*} S_n &= a_1+a_2+a_3+\cdots+a_n \\		
    &= 1^2+2^2+3^2\cdots +
    n^2.  \intertext{By Theorem \ref{thm:summation}, this is} &=
    \frac{n(n+1)(2n+1)}{6}.
  \end{align*}
  Since $\lim_{n\to\infty}S_n = \infty$, we conclude that the series
  $\sum_{n=1}^\infty n^2$ diverges. It is instructive to write
  $\sum_{n=1}^\infty n^2=\infty$ for this tells us \emph{how} the series
  diverges: it grows without bound.
  A scatter plot of the sequences $(a_n)$ and $(S_n)$ is given in Figure
  \ref{fig:series1}(a). The terms of $(a_n)$ are growing, so the terms
  of the partial sums $(S_n)$ are growing even faster, illustrating that
  the series diverges.
  
\item Consider some of the partial sums $S_n$ of $(b_n)$:
  \begin{align*}
    S_1 &= 1\\
    S_2 &= 0\\
    S_3 &= 1\\
    S_4 &= 0
  \end{align*}
  This pattern repeats; we find that $S_n = \left(\begin{array}{cc} 1  & n\ \text{ is odd}\\
    0  & n\  \text{ is even}
  \end{array}\right..$
  As $(S_n)$ oscillates, repeating 1, 0, 1, 0, $\ldots$, we conclude
  that $\lim_{n\to\infty}S_n$ does not exist, hence $\sum_{n=1}^\infty
  (-1)^{n+1}$ diverges.
  
  A scatter plot of the sequence $(b_n)$ and the partial sums $(S_n)$
  is given in Figure \ref{fig:series1}(b). When $n$ is odd, $b_n =
  S_n$ so the marks for $b_n$ are drawn oversized to show they
  coincide.

\mtable{.6}{Scatter plots relating to Example \ref{ex_series1}.}{fig:series1}{%
\begin{tabular}{c}
\myincludegraphics{figures/figseries1a}\\[10pt]
(a)\\[15pt]
\myincludegraphics{figures/figseries1b}\\[10pt]
(b)
\end{tabular}
}																					
\end{enumerate}
\end{explanation}
\end{example}

While it is important to recognize when a series diverges, we are
generally more interested in the series that converge. In this section
we will demonstrate a few general techniques for determining
convergence; later sections will delve deeper into this topic.

\section{Geometric series}

One important type of series is a \textit{geometric series}.

\definition{def:geom_series}{Geometric Series}
{A \textbf{geometric series} is a series of the form 
$$\sum_{n=0}^\infty r^n = 1+r+r^2+r^3+\cdots+r^n+\cdots$$
Note that the index starts at $n=0$, not $n=1$.%
\index{series!geometric}\index{geometric series}
}

We started this section with a geometric series, although we dropped the first term of $1$. One reason geometric series are important is that they have nice convergence properties.

\theorem{thm:geom_series}{Convergence of Geometric Series}
{Consider the geometric series $\sum_{n=0}^\infty r^n$.
\begin{enumerate}
\item		The $n^\text{th}$ partial sum is: $S_n = \frac{1-r\,^{n+1}}{1-r}$.
\item		The series converges if, and only if, $|r| < 1$. When $|r|<1$, 
\index{series!geometric}\index{geometric series}\index{convergence!of geometric series}\index{divergence!of geometric series}
$$\sum_{n=0}^\infty r^n = \frac{1}{1-r}.$$
\end{enumerate}
}

According to Theorem \ref{thm:geom_series}, the series $\sum_{n=0}^\infty \frac{1}{2^n} = 1+\frac12+\frac14+\cdots$ converges, and $\sum_{n=0}^\infty \frac{1}{2^n} = \frac{1}{1-1/2} = 2.$ This concurs with our introductory example; while there we got a sum of 1, we skipped the first term of 1.\\
%\clearpage

\example{ex_series2}{Exploring geometric series}{
Check the convergence of the following series. If the series converges, find its sum.\\

\noindent 1. $\sum_{n=2}^\infty \left(\frac34\right)^n$\qquad 2. $\sum_{n=0}^\infty \left(\frac{-1}{2}\right)^n$ \qquad 3. $\sum_{n=0}^\infty 3^n$ 
}
{\begin{enumerate}
\item		Since $r=3/4<1$, this series converges. By Theorem \ref{thm:geom_series}, we have that
$$\sum_{n=0}^\infty \left(\frac34\right)^n = \frac{1}{1-3/4} = 4.$$ However, note the subscript of the summation in the given series: we are to start with $n=2$. Therefore we subtract off the first two terms, giving:
$$\sum_{n=2}^\infty \left(\frac34\right)^n = 4 - 1 - \frac34 = \frac94.$$
This is illustrated in Figure \ref{fig:series2a}.
\mfigure{.35}{Scatter plots relating to the series in Example \ref{ex_series2}.}{fig:series2a}{figures/figseries2a}

%\mfigure{.8}{Scatter plots relating to the series of Example \ref{ex_series2} part 1.}{fig:series2a}{figures/figseries2a}

\item	Since $|r| = 1/2 < 1$, this series converges, and by Theorem \ref{thm:geom_series},
$$\sum_{n=0}^\infty \left(\frac{-1}{2}\right)^n = \frac{1}{1-(-1/2)} = \frac23.$$
The partial sums of this series are plotted in Figure \ref{fig:series2}(a). Note how the partial sums are not purely increasing as some of the terms of the sequence $((-1/2)^n)$ are negative.

%\mfigure{.55}{Scatter plots relating to the series of Example \ref{ex_series2} part 2.}{fig:series2b}{figures/figseries2b}

\item		Since $r>1$, the series diverges. (This makes ``common sense''; we expect the sum $$1+3+9+27 + 81+243+\cdots$$ to diverge.) This is illustrated in Figure \ref{fig:series2}(b).

%\mfigure{.3}{Scatter plots relating to the series of Example \ref{ex_series2} part 3.}{fig:series2c}{figures/figseries2c}
\mtable{.67}{Scatter plots relating to the series in Example \ref{ex_series2}.}{fig:series2}{%
\begin{tabular}{c}
%\myincludegraphics{figures/figseries2a}\\[10pt]
%(a)\\[15pt]
\myincludegraphics{figures/figseries2b}\\[10pt]
(a)\\[15pt]
\myincludegraphics{figures/figseries2c}\\[10pt]
(b)
\end{tabular}
}
\end{enumerate}
\vskip-\baselineskip
}\\


\example{ex_series3}{Telescoping series}{
Evaluate the sum $\sum_{n=1}^\infty \left(\frac1n-\frac1{n+1}\right)$.
\index{series!telescoping}\index{telescoping series}}
{It will help to write down some of the first few partial sums of this series.
\begin{align*}
S_1 &=	\frac11-\frac12 & & = 1-\frac12\\
S_2 &=	\left(\frac11-\frac12\right) + \left(\frac12-\frac13\right) & & = 1-\frac13\\
S_3 &=	\left(\frac11-\frac12\right) + \left(\frac12-\frac13\right)+\left(\frac13-\frac14\right) & &= 1-\frac14\\
S_4 &=	\left(\frac11-\frac12\right) + \left(\frac12-\frac13\right)+\left(\frac13-\frac14\right) +\left(\frac14-\frac15\right)& &= 1-\frac15
\end{align*}
Note how most of the terms in each partial sum are canceled out! In general, we see that $S_n = 1-\frac{1}{n+1}$. The sequence $(S_n)$ converges,  as $\lim_{n\to\infty}S_n = \lim_{n\to\infty}\left(1-\frac1{n+1}\right) = 1$, and so we conclude that $\sum_{n=1}^\infty \left(\frac1n-\frac1{n+1}\right) = 1$. Partial sums of the series are plotted in Figure \ref{fig:series3}.
\mfigure{.75}{Scatter plots relating to the series of Example \ref{ex_series3}.}{fig:series3}{figures/figseries3}
}\\

The series in Example \ref{ex_series3} is an example of a \sword{telescoping series}. Informally, a telescoping series is one in which the partial sums reduce to just a finite number of terms. The partial sum $S_n$ did not contain $n$ terms, but rather just two: 1 and $1/(n+1)$.\index{series!telescoping}\index{telescoping series}

When possible, seek a way to write an explicit formula for the $n^\text{th}$ partial sum $S_n$. This makes evaluating the limit $\lim_{n\to\infty} S_n$ much more approachable. We do so in the next example.\\

%\noindent\textbf{Note on notation:} Most of the series we encounter will start with $n=1$. For ease of notation, we will often write $\sum a_n$ instead of writing $\sum_{n=1}^\infty a_n$.\\



\example{ex_series4}{Evaluating series}{
Evaluate each of the following infinite series.\\

\noindent 1. $\sum_{n=1}^\infty \frac{2}{n^2+2n}$ \qquad 2. $\sum_{n=1}^\infty \ln\left(\frac{n+1}{n}\right)$
}
{\begin{enumerate}
\item		We can decompose the fraction $2/(n^2+2n)$ as $$\frac2{n^2+2n} = \frac1n-\frac1{n+2}.$$ (See Section \ref{sec:partial_fraction}, Partial Fraction Decomposition, to recall how  this is done, if necessary.)

Expressing the terms of $(S_n)$ is now more instructive:
\footnotesize
\begin{align*}
S_1 &= 1-\frac13 &&= 1-\frac13\\
S_2 &= \left(1-\frac13\right) + \left(\frac12-\frac14\right) &&= 1+\frac12-\frac13-\frac14\\
S_3 &= \left(1-\frac13\right) + \left(\frac12-\frac14\right)+\left(\frac13-\frac15\right) &&= 1+\frac12-\frac14-\frac15\\
S_4 &= \left(1-\frac13\right) + \left(\frac12-\frac14\right)+\left(\frac13-\frac15\right)+\left(\frac14-\frac16\right) &&= 1+\frac12-\frac15-\frac16\\
S_5 &= \left(1-\frac13\right) + \left(\frac12-\frac14\right)+\left(\frac13-\frac15\right)+\left(\frac14-\frac16\right)+\left(\frac15-\frac17\right) &&= 1+\frac12-\frac16-\frac17\\
\end{align*}
\normalsize

We again have a telescoping series. In each partial sum, most of the terms cancel and we obtain the formula $S_n = 1+\frac12-\frac1{n+1}-\frac1{n+2}.$ Taking limits allows us to determine the convergence of the series:
$$\lim_{n\to\infty}S_n = \lim_{n\to\infty} \left(1+\frac12-\frac1{n+1}-\frac1{n+2}\right) = \frac32,\quad \text{so } \sum_{n=1}^\infty \frac1{n^2+2n} = \frac32.$$
This is illustrated in Figure \ref{fig:series4}(a).
%\mfigure{.3}{Scatter plots relating to the series of Example \ref{ex_series4} part 1.}{fig:series4a}{figures/figseries4a}
\mtable{.5}{Scatter plots relating to the series in Example \ref{ex_series4}.}{fig:series4}{%
\begin{tabular}{c}
\myincludegraphics{figures/figseries4a}\\[10pt]
(a)\\[15pt]
\myincludegraphics{figures/figseries4b}\\[10pt]
(b)
\end{tabular}
}
%\drawexampleline

\item		We begin by writing the first few partial sums of the series:

\begin{align*}
S_1 &= \ln\left(2\right) \\
S_2 &= \ln\left(2\right)+\ln\left(\frac32\right) \\
S_3 &= \ln\left(2\right)+\ln\left(\frac32\right)+\ln\left(\frac43\right) \\
S_4 &= \ln\left(2\right)+\ln\left(\frac32\right)+\ln\left(\frac43\right)+\ln\left(\frac54\right) 
\end{align*}
At first, this does not seem helpful, but recall the logarithmic identity: $\ln x+\ln y = \ln (xy).$ Applying this to $S_4$ gives:
$$S_4 = \ln\left(2\right)+\ln\left(\frac32\right)+\ln\left(\frac43\right)+\ln\left(\frac54\right) = \ln\left(\frac21\cdot\frac32\cdot\frac43\cdot\frac54\right) = \ln\left(5\right).$$

We can conclude that $(S_n) = \big(\ln (n+1)\big)$. This sequence  does not converge, as $\lim_{n\to\infty}S_n=\infty$. Therefore  $\sum_{n=1}^\infty  \ln\left(\frac{n+1}{n}\right)=\infty$; the series diverges. Note in Figure \ref{fig:series4}(b) how the sequence of partial sums grows slowly; after 100 terms, it is not yet over 5. Graphically we may be fooled into thinking the series converges, but our analysis above shows that it does not.
%\mfigure{.35}{Scatter plots relating to the series of Example \ref{ex_series4} part 2.}{fig:series4b}{figures/figseries4b}
\end{enumerate}
\vskip-1.5\baselineskip
}\\

%\enlargethispage{3\baselineskip}
We are learning about a new mathematical object, the series. As done before, we apply ``old'' mathematics to this new topic.

\theorem{thm:series_prop}{Properties of Infinite Series}
{Let \quad$\sum_{n=1}^\infty a_n = L$,\quad  $\sum_{n=1}^\infty b_n = K$, and let $c$ be a constant.
\begin{enumerate}
\item  Constant Multiple Rule: $\sum_{n=1}^\infty c\cdot a_n = c\cdot\sum_{n=1}^\infty a_n = c\cdot L.$\index{Constant Multiple Rule!of series}
\item		Sum/Difference Rule: $\sum_{n=1}^\infty \big(a_n\pm b_n\big) = \sum_{n=1}^\infty a_n \pm \sum_{n=1}^\infty b_n = L \pm K.$
\index{series!properties}\index{Sum/Difference Rule!of series}
\end{enumerate} 
}



















\end{document}
