\documentclass{ximera}

\input{../preamble.tex}

\title[Dig-In:]{Surface area}

\begin{document}
\begin{abstract}
We compute surface area.
\end{abstract}
\maketitle

%% Adoped from APEX

We have already seen how a curve $y=f(x)$ on $[a,b]$ can be revolved
around an axis to form a solid. Instead of computing this solid's
volume, we now consider its surface area.


We begin as we have in the previous sections: we partition the
interval $[a,b]$ with $n$ subintervals, where the $i\,^{\text{th}}$
subinterval is $[x_i,x_{i+1}]$.

On each subinterval, we can approximate the curve $y=f(x)$ with a
straight line that connects $f(x_i)$ and $f(x_{i+1})$ as shown in
Figure \ref{fig:surface_intro} (a). Revolving this line segment about
the $x$-axis creates part of a cone (called the \textit{frustum} of a
cone) as shown in Figure \ref{fig:surface_intro} (b). The surface area
of a frustum of a cone is
\[
2\pi\cdot\text{ length }\cdot\text{average of the two radii $R$ and
  $r$}.
\]

The length is given by $L$; we use the material just covered by arc
length to state that $$L\approx \sqrt{1+\fp(c_i)}\dx_i$$ for some
$c_i$ in the $i\,^\text{th}$ subinterval. The radii are just the
function evaluated at the endpoints of the interval. That is, $$R =
f(x_{i+1})\quad \text{and}\quad r = f(x_i).$$

Thus the surface area of this sample frustum of the cone is
approximately
$$2\pi\frac{f(x_i)+f(x_{i+1})}2\sqrt{1+\fp(c_i)^2}\dx_i.$$

Since $f$ is a continuous function, the Intermediate Value Theorem
states there is some $d_i$ in $[x_i,x_{i+1}]$ such that $\ds f(d_i) =
\frac{f(x_i)+f(x_{i+1})}2$; we can use this to rewrite the above
equation as
\[
2\pi f(d_i)\sqrt{1+\fp(c_i)^2}\dx_i.
\]
Summing over all the
subintervals we get the total surface area to be approximately
$$\text{Surface Area}\approx \sum_{i=1}^n 2\pi f(d_i)\sqrt{1+\fp(c_i)^2}\dx_i,$$
which is a Riemann Sum. Taking the limit as the subinterval lengths go to zero gives us the exact surface area, given in the following Key Idea.

\keyidea{idea:surface_area}{Surface Area of a Solid of Revolution}
{Let $f$ be differentiable on an open interval containing $[a,b]$ where $\fp$ is also continuous on $[a,b]$. \index{integration!surface area}\index{surface area!solid of revolution}
	\begin{enumerate}
	\item	The surface area of the solid formed by revolving the graph of $y=f(x)$, where $f(x)\geq0$, about the $x$-axis is
	$$\text{Surface Area} = 2\pi\int_a^b f(x)\sqrt{1+\fp(x)^2}\ dx.$$
	\item	The surface area of the solid formed by revolving the graph of $y=f(x)$ about the $y$-axis, where $a,b\geq0$, is
	$$\text{Surface Area} = 2\pi\int_a^b x\sqrt{1+\fp(x)^2}\ dx.$$
	\end{enumerate}
}

(When revolving $y=f(x)$ about the $y$-axis, the radii of the resulting frustum are $x_i$ and $x_{i+1}$; their average value is simply the midpoint of the interval. In the limit, this midpoint is just $x$. This gives the second part of Key Idea \ref{idea:surface_area}.)\\

\example{ex_sa1}{Finding surface area of a solid of revolution}{
Find the surface area of the solid formed by revolving $y=\sin x$ on $[0,\pi]$ around the $x$-axis, as shown in Figure \ref{fig:sa1}.
\mfigure{.4}{Revolving $y=\sin x$ on $[0,\pi]$ about the $x$-axis.}{fig:sa1}{figures/figsa1}
}
{The setup is relatively straightforward. Using Key Idea \ref{idea:surface_area}, we have the surface area $SA$ is:
\begin{align*}
SA  &=	2\pi\int_0^\pi \sin x\sqrt{1+\cos^2x}\ dx \\
		&=	-2\pi\frac12\left.\left(\sinh^{-1}(\cos x)+\cos x\sqrt{1+\cos^2x}\right)\right|_0^\pi \\
		&= 2\pi\left(\sqrt{2}+\sinh^{-1} 1\right) \\
		&\approx 14.42\ \text{units}^2.
\end{align*}
The integration step above is nontrivial, utilizing an integration method called Trigonometric Substitution. 

It is interesting to see that the surface area of a solid, whose shape is defined by a trigonometric function, involves both a square root and an inverse hyperbolic trigonometric function.
}\\

\example{ex_sa2}{Finding surface area of a solid of revolution}{
Find the surface area of the solid formed by revolving the curve $y=x^2$ on $[0,1]$ about:
		\begin{enumerate}
		\item		the $x$-axis
		\item		the $y$-axis.
		\end{enumerate}
\mtable{.55}{The solids used in Example \ref{ex_sa2}.}{fig:sa2}{%
\begin{tabular}{c}
\myincludegraphics{figures/figsa2a}\\
(a)\\[15pt]
\\
\myincludegraphics{figures/figsa2b}\\
(b)
\end{tabular}
}
}
{\begin{enumerate}
	\item		The integral is straightforward to setup:
	\begin{align*}
	SA &= 2\pi\int_0^1 x^2\sqrt{1+(2x)^2}\ dx.
	\intertext{Like the integral in Example \ref{ex_sa1}, this requires Trigonometric Substitution.}
		&= \left.\frac{\pi}{32}\left(2(8x^3+x)\sqrt{1+4x^2}-\sinh^{-1}(2x)\right)\right|_0^1\\
		&=\frac{\pi}{32}\left(18\sqrt{5}-\sinh^{-1}2\right)\\
		&\approx 3.81\ \text{units}^2.
	\end{align*}
	The solid formed by revolving $y=x^2$ around the $x$-axis is graphed in Figure \ref{fig:sa2} (a).
	
	\item	 Since we are revolving around the $y$-axis, the ``radius'' of the solid is not $f(x)$ but rather $x$. Thus the integral to compute the surface area is:
	\begin{align*}
	SA &= 2\pi\int_0^1x\sqrt{1+(2x)^2}\ dx.
		\intertext{This integral can be solved using substitution. Set $u=1+4x^2$; the new bounds are $u=1$ to $u=5$. We then have }
		&=	\frac{\pi}4\int_1^5 \sqrt{u}\ du \\
		&= \left.\frac{\pi}{4}\frac23 u^{3/2}\right|_1^5\\
		&= \frac{\pi}6\left(5\sqrt{5}-1\right)\\
		&\approx 5.33\ \text{units}^2.
	\end{align*}
 The solid formed by revolving $y=x^2$ about the $y$-axis is graphed in Figure \ref{fig:sa2} (b).	
\end{enumerate}

\enlargethispage{3\baselineskip}
\vskip-1.5\baselineskip
}\\
%\clearpage


% Adapted from Guichard/Mike Wills material
\section{Surface Area}{}{}
\label{sec:surface area}
\nobreak

\begin{example} We compute the surface area of a sphere of radius $r$.
The sphere can be obtained by rotating the graph of
  $\ds f(x)=\sqrt{r^2 - x^2}$ about the $x$-axis.
The derivative $f'$ is $\ds -x/\sqrt{r^2-x^2}$, so the surface area is
given by
$$\eqalign{
A&=2\pi \int_{-r }^r \sqrt{r^2 - x^2}\sqrt{1+{x^2\over r^2-x^2}}\,dx \\
&=2\pi \int_{-r }^r \sqrt{r^2 - x^2}\sqrt{r^2\over r^2-x^2}\,dx \\
&=2\pi \int_{-r }^r r\,dx=2\pi r\int_{-r }^r 1\,dx=4\pi r^2 \\}$$
\vskip-10pt\end{example}




\end{document}
