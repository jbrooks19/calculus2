\documentclass{ximera}

\input{../preamble.tex}

\title[Dig-In:]{Power series}

\begin{document}
\begin{abstract}
  Infinite series can represent functions.
\end{abstract}
\maketitle

We have seen that Taylor polynomials are good polynomial approximations to a function near a point.  The higher the degree, the better the approximation.  When we let the degree go to infinity, we obtain Taylor series.

\begin{definition}
	The \dfn{Taylor series} of a function $f$ centered at $a$ is the following power series:
	
	\[
	\sum_{k=0}^\infty \frac{f^{(k)}(a)}{k!} = f(a)+f'(a)(x-a)+\frac{f''(a)}{2!}(x-a)^2+\frac{f^{3}(a)}{3!}(x-a)^3+\dots
	\]
	
	A Taylor series centered at $0$ is called a \dfn{Maclaurin series}.
\end{definition}

\begin{example}
	Let's try to find the Taylor series of some familiar functions.
	
	Let $f(x) = \sin(x)$
	
	The
	
	\begin{align*}
		f(0) =  \answer{0}\\
		f'(0) = \answer{1}\\
		f''(0) = \answer{0}\\
		f^{3}(0) = \answer{-1}\\
		f^{4}(0) = \answer{0}\\
		f^{5}(0) = \answer{1}\\
		\vdots
	\end{align*}
	
	Letting $c_k$ be the coefficient of $x^k$ in the Maclaurin series, we then have that
	
	\begin{align*}
		c_0 = \answer{0}\\
		c_1 = \answer{1}\\
		c_2 = \answer{0}\\
		c_3 = \answer{-\frac{1}{6}}\\
		c_4 = \answer{0}\\
		c_5 = \answer{\frac{1}{120}}
	\end{align*}
	
	Thus the Maclaurin series for sine is 
	
	\[
	x-\frac{x^3}{3!}+\frac{x^5}{5!}-\frac{x^7}{7!}+... = \sum_{k=1}^\infty \frac{(-1)^{k+1}}{(2k+1)!} x^{2k+1}
	\]
	
	Which of the following is the radius of convergence of this power series?
	\begin{multipleChoice}
		\choice{$0$}
		\choice{$1$}
		\choice{$\pi$}
		\choice[correct]{$\infty$}
	\end{multipleChoice}
	
	\begin{hint}
		Using the ratio test, we have that 
		
		\begin{align*}
			r &= \lim_{k \to \infty} \left| \frac{x^{2x+3}}{(2k+3)!} \frac{(2k+1)!}{x^{2k+1}} \right|\\
			&= |x|^2 \lim_{k \to \infty} \frac{1}{(2k+3)(2k+2)}\\
			&=0
		\end{align*}
		
		So for every value of $x$, the series converges, and so the radius of convergence is infinite.
	\end{hint}
	
\end{example}

\begin{example}
	The Taylor series centered at $a$ of a function defined by a power series centered at $a$ is just the power series.
	
	The last sentence may not sound like it is saying very much, but it actually has some content.
	
	\begin{question}

	For instance, you should recognize that
	
	\[
	f(x) = 1+x+x^2+x^3+x^4+...
	\]
	
	is a geometric series with ratio $x$.  Therefore it converges when $|x|<1$, and is equal to 
	
	\[
	f(x) = \answer{\frac{1}{1-x}}
	\]
	
	
	\begin{hint}
		$f(x) = \frac{1}{1-x}$
	\end{hint}	
	
		Notice that the function $f$ has a much larger natural domain than its Taylor series does!

	
	\end{question}
	
	
	Using the fact that $1+x+x^2+x^3+x^4+...$ must be the taylor series for $f$, you can compute $f^k(0)$ without actually taking any derivatives.
	
	\begin{question}
	
	\[
	f^{3}(0) = \answer{6}
	\]
	
	\begin{hint}
		Since all of the Taylor coefficients are $1$, we have $1 = \frac{f^{(k)}(0)}{k!}$, so it must be that $f^{(k)}(0) = k!$.  Thus the third derivative of $f$ at $0$ is $6$.
	\end{hint}
	\end{question}
	
	\begin{question}
	To get a little trickier, let $g(x) = \frac{7x^2}{1-x}$.  Use Taylor series to figure out what $g''(0)$ is.
	
	\[
	g^{(3)}(0) = \answer{56}
	\]
	\end{question}
	
	\begin{hint}
		We can get the Taylor series for $g$ from the Taylor series for $f$ by just multiplying through by $7x^2$.
		
		\[
		g(x) = 7x^2+7x^3+7x^4+\dots
		\]
		
		Thus $7 = \frac{g^{(3)}(0)}{3!}$, so $g^{(3)}(0) = 56$
	\end{hint}
\end{example}

\begin{warning}
	Since we designed Taylor polynomials to approximate functions, you might guess that the Taylor series of a function is equal to the function (at least on the interval of convergence for the Taylor series).  This is \textbf{false}.
\end{warning}

Here is an unsatisfying example:

\begin{example}
	\[
	f(x) = \begin{cases}
		1 \textrm{ if $|x|<1$}\\
		0 \textrm{ else}
		\end{cases}
	\]
	
	Then the Maclaurin series of $f$ is $\answer{1}$.
	
	The interval of convergence of the Maclaurin series is
	
	\begin{multipleChoice}
		\choice{$0$}
		\choice{$1$}
		\choice[correct]{$\infty$}
	\end{multipleChoice}
	
	So does the Maclaurin series of $f$ equal $f$ on its interval of convergence?
	
	\begin{multipleChoice}
		\choice{Yes}
		\choice[correct]{No}
	\end{multipleChoice}
\end{example}

A much more satisfying example is the following

\begin{example}
	Let 
	
	\[
	f(x) = \begin{cases}
		e^{-1/x^2} \textrm{ if $x \neq 0$}\\
		0 \textrm{ if $x = 0$}
	\end{cases}
	\]
	
	It turns out that $f$ is infinitely differentiable everywhere, but all of its derivatives vanish at $x=0$.  It "goes to $0$" faster than any polynomial, and so no polynomial term "detects" it.
	
	It is within your power to show that $f$ is infinitely differentiable everywhere, and to prove that $f^{(k)}(0) = 0$.  This is quite involved, and we will not do it here.  If you have the gumption, and the willpower, it would make a fantastic exercise.
\end{example}

How can we \textbf{prove} that a given function is actually represented by its Taylor series?

\begin{example}
	The Maclaurin series for $e^x$ is 
	\[
	1+x+\frac{x^2}{2!} +\frac{x^3}{3!}+\dots
	\]
	
	You can show that this power series has an infinite radius of convergence.
	
	We want to know whether $e^x = 1+x+\frac{x^2}{2!} +\frac{x^3}{3!}+\dots$ for every real number $x$.
	
	Another way of phrasing this is that we want the remainder
	
	\[
	R_n(x) = e^x - (1+x+\frac{x^2}{2!} +\frac{x^3}{3!}+\dots+\frac{x^{n}}{n!})
	\]
	
	to satisfy 
	
	\[
	\lim_{n \to \infty} R_n(x) = 0 \textrm{ for all real numbers $x$} 
	\]
	
	We learned in a previous section that
	
	\[
	R_n(x) = \frac{f^{(n+1)}(c)}{(n+1)!}x^{n+1}
	\]
	
	for some real number $c$ between $0$ and $x$.  Note that $c$ depends on  both $n$ and $x$.
	
	  We know $f^{(n+1)}(c) = e^c$, since all the derivatives of $e^x$ are just $e^x$.
	
	Thus 
	
	\[
	R_n(x) = \frac{e^c}{(n+1)!}x^{n+1}
	\]
	
	So how large can $R_n(x)$ be?  If $c$ is between $0$ and $x$, then  $e^c$ is between $1$ and $e^{|x|}$, so
	
	\[
	\left| R_n(c)\right| \leq \frac{e^{|x|}}{(n+1)!}|x|^{n+1}
	\]
	
	Thus 
	\[
	\lim_{n \to \infty}\left| R_n(c)\right| = \lim_{n \to \infty} \frac{e^{|x|}}{(n+1)!}|x|^{n+1} = 0
	\]
	
	proving that the Maclaurin series for $e^x$ converges to $e^x$ everywhere!
	

\end{example}

\end{document}
