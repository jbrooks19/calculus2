\documentclass{ximera}

\input{../preamble.tex}


\outcome{State the definition of a vector.}
\outcome{Work with vectors in two or three dimensions. }
\outcome{Multiply vectors by scalars.}
\outcome{Add and subtract vectors.}
\outcome{Calculate the magnitude of a vector.}
\outcome{Find unit vectors.}
\outcome{Use vectors in applied settings.}
\outcome{Find the equations for planes.}
\outcome{Give the equation for a sphere or ball.}

\title[Dig-In:]{Vectors}

\begin{document}
\begin{abstract}
  Vectors are lists of numbers that denote direction and magnitude.
\end{abstract}
\maketitle


\section{The idea of vectors}

So far, we have mostly studied functions which take single numbers as
their inputs and output either individual numbers or ordered-pairs (as
in the case of parametric and polar functions).  Now we set the stage
for the study of functions that accept lists of numbers as inputs and
lists of numbers as outputs. When we want to keep track of more than
one number at a time, we use a \textit{vector}.

\begin{definition}
  A \dfn{vector} is something with ``direction'' and ``magnitude.''
\end{definition}

We write vectors typographically as either boldfaced letters or
numbers (like $\vec{v}$ or $\vec{w}$) or decorated with an arrow hat
(like $\arrowvec{v}$ or $\arrowvec{w}$).  We often visualize a vector as
an arrow to explicitly show its direction and magnitude:
\begin{image}
  \begin{tikzpicture}
	\begin{axis}[
            xmin=-1,xmax=5,ymin=-1,ymax=3,
            clip=false,
            axis lines=center,
            %ticks=none,
            unit vector ratio*=1 1 1,
            xlabel=$x$, ylabel=$y$,
            %ytick={-2,-1,...,7},
	    %xtick={-2,-1,...,10},
	    grid = major,
            every axis y label/.style={at=(current axis.above origin),anchor=south},
            every axis x label/.style={at=(current axis.right of origin),anchor=west},
          ]
          \addplot[very thick,penColor,->] plot coordinates {(1,1) (4,2)};
        \end{axis}
\end{tikzpicture}
\end{image}
Two vectors are equal when they have the same direction and magnitude.
\begin{question}
  True or False: Given $\vec{v}$ and $\vec{w}$ in the diagram below
  \begin{image}
  \begin{tikzpicture}
	\begin{axis}[
            xmin=-1,xmax=5,ymin=-1,ymax=4,
            clip=false,
            axis lines=center,
            %ticks=none,
            unit vector ratio*=1 1 1,
            xlabel=$x$, ylabel=$y$,
            %ytick={-2,-1,...,7},
	    %xtick={-2,-1,...,10},
	    grid = major,
            every axis y label/.style={at=(current axis.above origin),anchor=south},
            every axis x label/.style={at=(current axis.right of origin),anchor=west},
          ]
          \addplot[very thick,penColor,->] plot coordinates {(1,2) (4,3)};
          \addplot[very thick,penColor2,->] plot coordinates {(0,0) (3,1)};
          \node[above] at (axis cs:2.5, 2.5) [penColor] {$\vec{v}$};
          \node[above] at (axis cs:1.5, .5) [penColor2] {$\vec{w}$};

        \end{axis}
\end{tikzpicture}
\end{image}
  we have that $\vec{v}=\vec{w}$.
  \begin{prompt}
  \begin{multipleChoice}
    \choice[correct]{true}
    \choice{false}
  \end{multipleChoice}
  \end{prompt}
\end{question}

Given two points in the plane, the vector $\vec{v}$ whose tip is at $(a,b)$ and
whose tail is at $(c,d)$ is $\vec{v}=\vector{a-c,b-d}$
\begin{image}
  \begin{tikzpicture}
	\begin{axis}[
            xmin=-1,xmax=5,ymin=-1,ymax=3,
            clip=false,
            axis lines=center,
            ticks=none,
            unit vector ratio*=1 1 1,
            xlabel=$x$, ylabel=$y$,
            %ytick={-2,-1,...,7},
	    %xtick={-2,-1,...,10},
	    %grid = major,
            every axis y label/.style={at=(current axis.above origin),anchor=south},
            every axis x label/.style={at=(current axis.right of origin),anchor=west},
          ]
          \addplot[very thick,penColor,->] plot coordinates {(1,1) (4,2)};
          \node[above] at (axis cs:4, 2) [penColor] {$\mathrm{tip}=(a,b)$};
          \node[below] at (axis cs:1, 1) [penColor] {$\mathrm{tail}=(c,d)$};
          \node[below right] at (axis cs:2.5, 1.5) [penColor] {$\vec{v}=\vector{a-c,b-d}$};
         \end{axis}
\end{tikzpicture}
\end{image}

\begin{question}
  What vector has its tip at $(1,2)$ and its tail at $(4,3)$?
  \begin{prompt}
    \[
    \vec{v} = \vector{\answer{-3},\answer{-1}}
    \]
  \end{prompt}
  
  \begin{feedback}
     \begin{image}
       \begin{tikzpicture}
	 \begin{axis}[
            xmin=-4,xmax=5,ymin=-2,ymax=4,
            clip=false,
            axis lines=center,
            %ticks=none,
            unit vector ratio*=1 1 1,
            xlabel=$x$, ylabel=$y$,
            ytick={-2,-1,...,4},
	    xtick={-4,-3,...,5},
	    grid = major,
            every axis y label/.style={at=(current axis.above origin),anchor=south},
            every axis x label/.style={at=(current axis.right of origin),anchor=west},
          ]
          \addplot[very thick,penColor,->] plot coordinates {(4,3) (1,2)};
          \addplot[very thick,penColor2,->] plot coordinates {(0,0) (-3,-1)};
          \node[above] at (axis cs:2.5, 2.5) [penColor] {$\vec{v}$};
          \node[above] at (axis cs:-1.5, -.5) [penColor2] {$\vec{v}$};

         \end{axis}
       \end{tikzpicture}
     \end{image}
  \end{feedback}
\end{question}


Since vectors are determined only by their direction and magnitude,
notation such as
\[
\vector{a,b,c}
\]
completely describes a vector, since we assume the tail is at the
origin. We should point out that there is other common notation for
vectors including:
\[
\begin{bmatrix}
  a\\
  b\\
  c
\end{bmatrix}, \quad
\begin{bmatrix}
  a & b & c
\end{bmatrix},
\quad
(a,b,c).
\]


\section{Operations on vectors}



\begin{definition}
The \dfn{dimension} of a vector is the number of entries. Each
individual entry of a vector is called a \dfn{component}.
\end{definition}
\begin{question}
  What is the dimension of the vector 
  \[
  \vector{3,4,1,-4}?
  \]
  \[
  \text{Dimension} = \answer{4}
  \]
  \begin{question}
    What are the components of the vector $\vector{1,2,3}$?
    \begin{prompt}
      \begin{itemize}
      \item The $x$-component is $\answer{1}$.
      \item The $y$-component is $\answer{2}$.
      \item The $z$-component is $\answer{3}$.
      \end{itemize}
    \end{prompt}
  \end{question}
\end{question}

We can add vectors of the same dimension together by component-wise
addition:

\begin{question}
  \[
  \vector{1,2,3}+ \vector{-1,2,2} =
  \vector{\answer{0},\answer{4},\answer{5}}
  \]
\end{question}
Now let us investigate the geometry of addition of vectors:
\[
\vec{v} = \vector{1  , 2} \qquad \vec{w} = \vector{3  , 1 }
\]
Then
\[
\vec{v}+\vec{w} = \vector{4 , 3}
\]
\begin{image}
  \begin{tikzpicture}
    \begin{axis}[
        xmin=-1,xmax=5,ymin=-1,ymax=4,
            axis lines=center,
            %ticks=none,
            unit vector ratio*=1 1 1,
            xlabel=$x$, ylabel=$y$,
            ytick={-2,-1,...,7},
	    %yticklabels={$0.5$,$1$,$1.5$,$2$},
	    xtick={-2,-1,...,10},
	    %xticklabels={$0.5$,$1$,$1.5$,$2$},
	    grid = major,
            every axis y label/.style={at=(current axis.above origin),anchor=south},
            every axis x label/.style={at=(current axis.right of origin),anchor=west},
          ]
          \addplot[very thick,penColor,->] plot coordinates {(1,2) (4,3)};
          \addplot[very thick,penColor2,->] plot coordinates {(0,0) (1,2)};
          \addplot[ultra thick,penColor3,->] plot coordinates {(0,0) (4,3)};

           \node[left] at (axis cs:.5, 1) [penColor2] {$\vec{v}$};
           \node[above] at (axis cs:2.5, 2.5 ) [penColor] {$\vec{w}$};
           \node[below right] at (axis cs:2, 1.5 ) [penColor3] {$\vec{v}+\vec{w}$};
    \end{axis}
\end{tikzpicture}
\end{image}

If we place the tail of the vector $\vec{w}$ at the tip of the vector
$\vec{v}$, then the sum $\vec{v}+\vec{w}$ connects the tail of
$\vec{v}$ to the tip of $\vec{w}$

\begin{question}
  Consider the following diagram:
  \begin{image}
  \begin{tikzpicture}
    \begin{axis}[
        xmin=0,xmax=5,ymin=-1,ymax=4,
        axis lines=center,
            %ticks=none,
            unit vector ratio*=1 1 1,
            xlabel=$x$, ylabel=$y$,
            ytick={-2,-1,...,7},
	    %yticklabels={$0.5$,$1$,$1.5$,$2$},
	    xtick={-2,-1,...,10},
	    %xticklabels={$0.5$,$1$,$1.5$,$2$},
	    grid = major,
            every axis y label/.style={at=(current axis.above origin),anchor=south},
            every axis x label/.style={at=(current axis.right of origin),anchor=west},
          ]
          \addplot[very thick,penColor,->] plot coordinates {(2,3) (1,0)};
          \addplot[very thick,penColor2,->] plot coordinates {(4,1) (1,0)};
          \addplot[very thick,penColor3,->] plot coordinates {(4,1) (2,3)};
          
          \node[left] at (axis cs:1.5, 1.5 ) [penColor] {$\vec{a}$};
          \node[below] at (axis cs:2.5, .5) [penColor2] {$\vec{b}$};
          \node[above right] at (axis cs:3, 2 ) [penColor3] {$\vec{c}$};
    \end{axis}
  \end{tikzpicture}
  \end{image}
  Which equation is represented by the diagram above?
  \begin{multipleChoice}
    \choice{$\vec{a} + \vec{b} = \vec{c}$}
    \choice[correct]{$\vec{a} + \vec{c} = \vec{b}$}
    \choice{$\vec{b} + \vec{c} = \vec{a}$}
\end{multipleChoice}
\end{question}



We can also multiply vectors by a \dfn{scalar} (a number), by
multiplying each component by the scalar:

\begin{question}
  \[
  4\cdot \vector{2 , 4 , 0 , 1} = \vector{\answer{8}, \answer{16} , \answer{0} , \answer{4}}
  \]	
\end{question}

\begin{question}
  True or False: Multiplying a vector by a nonzero scalar will not
  change the direction of the vector.
  \begin{multipleChoice}
    \choice{true}
    \choice[correct]{false}
  \end{multipleChoice}
  \begin{feedback}
    Multiplying a vector by a positive scalar $s$ will not change the
    direction of the vector:
    \begin{image}
      \begin{tikzpicture}
	 \begin{axis}[
            xmin=-1,xmax=7,ymin=-1,ymax=3,
            clip=false,
            axis lines=center,
            %ticks=none,
            unit vector ratio*=1 1 1,
            xlabel=$x$, ylabel=$y$,
            ytick={-1,0,...,3},
	    xtick={-1,0,...,7},
	    grid = major,
            every axis y label/.style={at=(current axis.above origin),anchor=south},
            every axis x label/.style={at=(current axis.right of origin),anchor=west},
          ]
          \addplot[ultra thick,penColor,->] plot coordinates {(0,0) (3,1)};
          \addplot[very thick,penColor2,->] plot coordinates {(0,0) (6,2)};
          \node[above] at (axis cs:1.5, .5) [penColor] {$\vec{v}$};
          \node[below] at (axis cs:4, 1.25) [penColor2] {$s\cdot\vec{v}$};

         \end{axis}
       \end{tikzpicture}
    \end{image}
    However, if we multiply a vector by a \textit{negative} scalar,
    call it $-s$, then the direction will change:
    \begin{image}
      \begin{tikzpicture}
	 \begin{axis}[
            xmin=-7,xmax=4,ymin=-3,ymax=2,
            clip=false,
            axis lines=center,
            %ticks=none,
            unit vector ratio*=1 1 1,
            xlabel=$x$, ylabel=$y$,
            ytick={-3,-2,...,2},
	    xtick={-7,-6,...,4},
	    grid = major,
            every axis y label/.style={at=(current axis.above origin),anchor=south},
            every axis x label/.style={at=(current axis.right of origin),anchor=west},
          ]
          \addplot[very thick,penColor,->] plot coordinates {(0,0) (3,1)};
          \addplot[very thick,penColor2,->] plot coordinates {(0,0) (-6,-2)};
          \node[above] at (axis cs:1.5, .5) [penColor] {$\vec{v}$};
          \node[below] at (axis cs:-3, -1) [penColor2] {$-s\cdot\vec{v}$};

         \end{axis}
       \end{tikzpicture}
    \end{image}
  \end{feedback}
\end{question}

\section{Magnitude and vectors}

We can visualize vectors in dimension $1$, $2$, or $3$ as directed arrows.


\begin{question}
	Which vector is depicted by the following picture?
	
\begin{image}
  \begin{tikzpicture}
	\begin{axis}[
            domain=(-1:6),
            clip=false,
            axis lines=center,
            %ticks=none,
            unit vector ratio*=1 1 1,
            xlabel=$x$, ylabel=$y$,
            ytick={-2,-1,...,7},
	    %yticklabels={$0.5$,$1$,$1.5$,$2$},
	    xtick={-2,-1,...,10},
	    %xticklabels={$0.5$,$1$,$1.5$,$2$},
	    grid = major,
            every axis y label/.style={at=(current axis.above origin),anchor=south},
            every axis x label/.style={at=(current axis.right of origin),anchor=west},
          ]
         % \addplot[very thick,penColor2!50!white,->,>=stealth'] plot coordinates {(2,1) (6,4)};
          \addplot[very thick,penColor2,->] plot coordinates {(0,0) (2,3)};
          \addplot[color=penColor,fill=penColor] coordinates{(6,6)};  %% closed hole
         % \addplot[color=penColor,dashed] ({2+4*x},{1+3*x});
        \end{axis}
\end{tikzpicture}
\end{image}
	
	\[
	\vector{2,3}
	\]
\end{question}

\begin{question}
	What is the length of the vector 
	\[
	\vector{2,3}
	\]
	\[
	\text{Length}  = \answer{\sqrt{13}}
	\]
	
	\begin{hint}
	  The length is $\sqrt{2^2+3^2} = \sqrt{13}$
	\end{hint}
\end{question}

You were able to find the answer to the question above because you are
used to working with $2$ and $3$ dimensional objects.  We make the
following definition in $n$ dimensions.

\begin{definition}
	Let $\vec{v} = \langle v_1, v_2, v_3, \dots, v_n \rangle \in
        \R^n$ is an $n$ dimensional vector.  Then the \dfn{length} or
        \dfn{magnitude} of $\vec{v}$ is denoted by $|\vec{v}|$ and is
        defined by	
	\[
	|\vec{v}| = \sqrt{v_1^2+v_2^2+v_3^2+\dots+v_n^2}
	\]
\end{definition}

We can understand the operations of scalar multiplication and vector
addition from this perspective as well:

Let

\[
\vec{v} = \vector{1  , 2 }
\]

Then 

\[
2\vec{v} =  \vector{2  , 4 }
\]

If we graph both of these, we see that $2\vec{v}$ points in the same
direction as $\vec{v}$, but is twice as long.  In other words it
\textit{scales} the vector by a factor of $2$ (which explains why we
call numbers ``scalars'' in this context).


\begin{image}
  \begin{tikzpicture}
	\begin{axis}[
            domain=(-1:2),
            clip=false,
            axis lines=center,
            %ticks=none,
            unit vector ratio*=1 1 1,
            xlabel=$x$, ylabel=$y$,
            ytick={-2,-1,...,7},
	    %yticklabels={$0.5$,$1$,$1.5$,$2$},
	    xtick={-2,-1,...,10},
	    %xticklabels={$0.5$,$1$,$1.5$,$2$},
	    grid = major,
            every axis y label/.style={at=(current axis.above origin),anchor=south},
            every axis x label/.style={at=(current axis.right of origin),anchor=west},
          ]
          \addplot[very  thick,penColor,->,>=stealth'] plot coordinates {(0,0) (2,4)};
          \addplot[ thick,penColor2,->,>=stealth'] plot coordinates {(0,0) (1,2)};
        \end{axis}
\end{tikzpicture}
\end{image}


\begin{observation}
  If $c$ is a positive constant, and $\vec{v}$ is a vector, then
  vector $c\vec{v}$ points in the same direction as $\vec{v}$, but its
  length is scaled by a factor of $c$.  If $c$ is negative, then
  $c\vec{v}$ points in the opposite direction of $\vec{v}$, and its
  length is scaled by a factor of $|c|$.
\end{observation}

\begin{definition}
	A \dfn{unit vector} is a vector of length $1$.
\end{definition}

BADBBAD
Unit vector of any angle
BADBAD

\begin{question}
	Find a unit vector $\vec{u}$ which points in the same direction as the vector $\vec{v} = \langle 2,1,3,7,1\rangle$ .
	
	\[
	\vector{
	\answer{2/8},
	\answer{1/8},
	\answer{3/8},
	\answer{7/8},
	\answer{1/8}}
	\]
	
	\begin{hint}
		Scaling the vector $\vec{v}$ by the reciprocal of its length should result in a length $1$ vector which points in the same direction
	\end{hint}
	
	\begin{hint}
		$|\vec{v}| = \sqrt{2^2+1^2+3^2+7^2+1^2} = \sqrt{64} = 8$
	\end{hint}
	
	\begin{hint}
		Thus the vector $ \vec{u} = \vector{\frac{2}{8},  \frac{1}{8},\frac{3}{8},\frac{7}{8},\frac{1}{8}}$ points in the same direction of $\vec{v}$, but is of unit length.
	\end{hint}
\end{question}

\begin{theorem}
  If $\vec{v}$ is a nonzero vector, then the unit vector which
  points in the same direction as $\vec{v}$ is
  $\frac{1}{|\vec{v}|} \vec{v}$
\end{theorem}


\section{Some three dimensional geometry}

In three dimensions we have three coordinates axes, the $x$, $y$, and $z$ axis:

\begin{image}
\begin{tikzpicture}
\begin{axis}[
  view={35}{15},
  axis lines=center,
  width=15cm,height=15cm,
  xtick={-10,-5,5,10},ytick={-10,-5,5,10},ztick={-10,-5,5,10},
  minor tick={-12,-11,...,12},
  xmin=-11,xmax=11,ymin=-11,ymax=11,zmin=-11,zmax=11,
 % xlabel={$x$},ylabel={$y$},zlabel={$z$},
]


\node [above right] at (axis cs:10,0,0) {$x$};
\node [above right] at (axis cs:0,10,0) {$y$};
\node [above right] at (axis cs:0,0,10) {$z$};


\end{axis}
\end{tikzpicture}
\end{image}

Together, these form three coordinate planes (The $(x,y)$-plane, the $(x,z)$-plane, and the $(y,z)$-plane).

\begin{question}
	The $yz$-plane corresponds to which of the following equations?
	
	\begin{multipleChoice}
		\choice[correct]{$x=0$}
		\choice{$y=0$}
		\choice{$z=0$}
	\end{multipleChoice}
	
	\begin{hint}
	  Every point on the $yz$ axis has $x=0$, so this is the answer.
	\end{hint}
\end{question}

\begin{question}
	Which of the following most accurately describes the solution
        set of $y=2$ in $\R^3$?
	
	\begin{multipleChoice}
		\choice{A plane parallel to the $xy$ plane}
		\choice[correct]{A plane parallel to the $xz$ plane}
		\choice{A plane parallel to the $yz$ plane}
	\end{multipleChoice}

\begin{hint}
	$y=2$ consists of all those points where $y=2$, but $x$ and
  $z$ are allowed to be anything.  This corresponds to a plane which
  is parallel to the $xz$ plane ($y=0$).
\end{hint}
	
\end{question}

\begin{question}
	The equation $(x-1)^2+y^2+(z+2)^2 = 4$ has a solution set in
        $\R^3$ which is a sphere.  What is the center and
        radius of this sphere?
	
\[
\text{Radius} = \answer{2}
\]

\[
\text{Center} = \left(\answer{1},\answer{0}, \answer{-2} \right)
\]

\begin{hint}
	$(x-1)^2+y^2+(z+2)^2$ is the square of the distance from
  $(1,0,-2)$ to $(x,y,z)$.  If the square of the distance is $4$, then
  the distance is $2$.  Since the solution set of this equation is all
  points which are a distance of $2$ away from $(1,0,-2)$, then this
  is a sphere of radius $2$ centered at $(1,0,-2)$
\end{hint}
\end{question}

\end{document}
